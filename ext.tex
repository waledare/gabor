\section{Multivariate Extension} \label{sec:extension}
We propose the following extension to \svn from Chapter 1 to a multivariate setting. Let $\{X\}_{t \ge 0}$ be a $d$-dimensional vector of prices satisfying the stochastic integral equation:
\begin{align}
  X_t = X_0 + \int^t_0 b(s) \D s + \int^t_0 \sigma(s) \D W_s, \qquad  t \ge 0,\notag
  \label{}
\end{align}
where $\{W_s\}_{t \ge 0}$ is an $m$-dimensional standard Brownian motion; $b$ is  $\real^d$-valued, adapted, and \cadlag; $\sigma$ is $\real^d \times \real^m$-valued, adapated, and continuous; and $X_0$ is known at time 0. Let 
\begin{align}
  \Sigma (t) := \sigma(t)\sigma'(t), \qquad t \ge  0\notag
  \label{}
\end{align}
where $\sigma'(t)$ is the transpose of $\sigma(t)$. Our aim is to obtain an estimate for $\Sigma$ on the basis of $n$ discretely and synchronously  observed price vectors $\{X_1,\cdots,X_n\}$ in some fixed time interval. The observation interval is normalized to \domain without loss of generality. We assume that the prices are observed at equidistant intervals given by 
\begin{align}
\Delta_n : = 1/n.
\end{align}
Let $\{g_{h,k}, \tghk\}_{h,k \in \ints}$ denote  a pair of dual Gabor frames generated under the assumptions of  Lemma \eqref{le:gabor}; let  $\Theta_n := \{(h,k) \in \ints^2 : |h| \le H_n \text{ and }|k| \le K_0\}$, where $H_n$ is  an increasing sequence in  $n$;  and let $K_0= \lceil (1 + |s| + |r|)/b \rceil$ be constant. Intuitively,  $H_n$ represents the length of the frequency domain expansion for each time window; whereas, $K_0$, which is a constant because of the finite time domain support of the volatility function, captures the number of frame elements used to tile the time axis for each frequency domain shift.    
We propose to estimate the spot co-volatility matrix in $[0,1]$ using $\hat{\Sigma}_n$ defined component-wise for $1 \le u,v \le d$ as follows:
\begin{align}
  &\Svn^{u,v}(t) = \sum_{(h,k) \in \Theta_n} \cnhk^{u,v}\;g_{h,k}(t),\qquad \forall t \in [0,1],  \label{eq:Sigma}
\end{align}
where 
\begin{align}
  &\cnhk^{u,v} = \sum_{i =0}^{n-1} \btghki (X_u(t_{i+1}) - X_u(t_i))(X_v(t_{i+1}) - X_v(t_i)).
  \label{}
\end{align}
We conjecture that \Svn is consistent for $\Sigma$ and that it converges in the mean integrated error sense at the same rate of convergence as \svn (more precisely, the same order of convergence. So actual rate modulo a constant factor which we conjecture to be equal to the number $m$ of driving Brownian motions). A rigorous proof of this conjecture will be given and further substantiated with simulations.
