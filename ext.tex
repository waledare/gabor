\section{Extensions}
\subsection{Estimating the spot co-volatility matrix}
We propose the following extension to \svn from Chapter 1 to a multivariate setting. Let $\{X\}_{t \ge 0}$ be a $d$-dimensional vector of log-prices satisfying the stochastic integral equation:
\begin{align}
  X_t = X_0 + \int^t_0 \mu(t, X_s) \D s + \int^t_0 \sigma(t) \D W_s, \qquad  t \ge 0,\notag
  \label{}
\end{align}
where $\{W_s\}_{t \ge 0}$ is an $m$-dimensional standard Brownian motion; $\mu$ is  $\real^d$-valued, continuous, and locally bounded in both variables; $\sigma$ is $\real^d \times \real^m$-valued, continuous and locally bounded in time.   Let 
\begin{align}
  \Sigma (t) := \sigma(t)\sigma'(t), \qquad t \ge  0\notag
  \label{}
\end{align}
where $\sigma'(t)$ is the transpose of $\sigma(t)$. Our aim is to obtain an estimate for $\Sigma$ on the basis of $n$ discretely and synchronously  observed price vectors $\{X_1,\cdots,X_n\}$ in some fixed time interval $[0,T]$, where $T < \infty$. With very little loss of generality we assume that the prices are observed at equidistant intervals given by 
\begin{align}
\Delta_n : = T/n.
\end{align}
Let $\{g_{h,k}, \tghk\}_{h,k \in \ints}$ be a pair of dual Gabor frames generated as in Lemma \eqref{le:gabor}, and $\Theta_n := \{(h,k) \in \ints^2 : |h| \le H_n \text{ and }|k| \le K_0\}$ with $H_n$ an increasing sequence $n$ and $K_0= \lceil (T + |s| + |r|)/b \rceil$. 
We propose to estimate the spot co-volatility matrix in $[0,T]$ using $\hat{\Sigma}_n$ defined component-wise for $1 \le u,v \le d$ as follows:
\begin{align}
  &\Svn^{u,v}(t) = \sum_{(h,k) \in \Theta_n} \cnhk^{u,v}\;g_{h,k}(t), \text{ where}\label{eq:Sigma}\\
  &\cnhk^{u,v} = \sum_{i =0}^{n-1} \btghki (X_u(t_{i+1}) - X_u(t_i))(X_v(t_{i+1}) - X_v(t_i)).
  \label{}
\end{align}
We conjecture that \Svn is consistent for $\Sigma$ and that it converges in the mean integrated error sense at the same rate of convergence as \svn (more precisely, the same order of convergence. So actual rate modulo a constant factor which we conjecture to be equal to the number $m$ of driving Brownian motions). A rigorous proof of this conjecture will be given and further substantiated with simulations.
