\subsection{Asymptotic properties} \label{sec:deviation}
In this section we obtain an estimate of the rate of convergence of  the Gabor frame estimator on compact intervals of the real line. We will take \domain as the prototype for such intervals. The usual way to think about  $\sigma^2$ is as a stochastic process, but it is just as natural to think of it as a random element. A random element is an extension  of the familiar concept of a random variable to include  situations where the state space can be any metric space $E$. Because our interest is in studying the convergence of the estimator on \domain,  $E$ will be set equal to \Ltwo, the set of real-valued, square integrable function on \domain. Now, due to the path continuity assumption on $\sigma^2$, we will restrict our attention to \czero,  the set of real-valued, continuous  functions on \domain equipped with the \Ltwo norm. So, given  an outcome $\omega$ in the sample space  $\Omega$, the restriction of realized volatility $\sigma^2(\omega)$ to \domain is a  continuous function defined on \domain. Now, with regards smoothness, \state  is  a very diverse class, comprising    functions that are infinitely differentiable,  those that are nowhere differentiable, and everything in-between. For a random volatility coefficient it may very well be the case that for some outcome $\omega$, realized volatility $\sigma^2(\omega)$ is very smooth with finite derivatives of all orders, whereas for outcome $\omega'$,  $\sigma^2(\omega')$  is very rough with kinks everywhere. For instance, with probability one, the one-dimesional Brownian motion  maps outcomes $\omega$  to continuous functions in  \state, but it is  almost surely  the case that none of these functions will be  differentiable. 

Now,  a good estimator should  yield successively better approximations with increasing observation frequency, regarless of the degree of smoothness  of the realized volatility function. But in all probability, the \emph{rate} of convergence of the approximation would depend on the smoothness of the realized volatility, with faster rates achieved for smooth functions. That is, for outcomes $\omega, \omega'$ in $\Omega$, if $\sigma^2(\omega)$ is  smoother as a function of time than $\sigma^2(\omega')$ then the number of observations required to achieve a given level of accuracy when $\sigma^2(\omega)$ is realized should not exceed  the required number when $\sigma^2(\omega')$ is realized.  So, while an estimator might eventually converge regardless of the regularity of volatility, the convergence rate may be outcome- or state-dependent. 

To develop an asymptotic theory for the Gabor frame estimator which can account for state-dependency in convergence rates, we characterize the effective  state space of the volatility coefficient, viewed as a random element in \state, according to a smoothness criterion.
A simple way to achieve this characterization is via the H\"older continuity criterion. Let $0 < \alpha \le 1$, a function $f$ in \state is said to be H\"older continuous with exponent $\alpha$ if  there is a finite constant $K$ such that whenever $x$ and $y$ are distinct numbers in \domain then 
\begin{align}
  \homo{f} := \frac{\vert f(x) - f(y)\vert}{\vert x - y \vert^\alpha} \le K. 
  \label{eq:holder}
\end{align}
The set of \holder continuous functions with exponent $\alpha$ is denoted by \calpha. The \holder class with $\alpha = 1$ is the familiar class of Lipschitz continuous functions. The \holder classes admit a natural ordering relation whereby if $\alpha$ is larger than  $\beta$ then every function that is \holder-continuous with exponent $\alpha$ is also  \holder-continuous with exponent $\beta$. Note that with regards regularity (smoothness), the ordering is reversed: the smoother the function the larger the \holder exponent. Consequently, the Lipschitz class ($ \alpha = 1$) is contained in every \holder class, and it is also the class with the smoothest (most regular) functions. As mentioned earlier, Brownian paths are nowhere differentiable, but using \holder classes the regularity of Brownian paths can be further qualified; this is a consequence of  the well-known L\'evy's Modulus of Continuity Theorem \citep[Theorem I.10.2]{Williams2000}, which states that, with probability one, no Brownian path is \holder-continuous with exponent larger than 1/2 on \domain.   
Now for a fixed exponent $\alpha$, the following norm may be defined on \calpha:
\begin{align}
  \hono{f} := \sup_{t \in \domain} |f(t)| + \homo{f},\notag
  \label{}
\end{align}
where \homo{f} is defined in \eqref{eq:holder}. The norm is obviously well-defined since $f$ is a continuous function defined on a compact set. Now using \hono{\cdot}  the state space may be characterized in a such a way that functions with similar regularity propoerties can be grouped together. We accomplish this via \holder balls: A \holder ball  of radius $c > 0$ is given by:
\begin{align}
  \hball{\alpha}{c}  := \{f \in \calpha : \hono{f} \le c\}.\notag
  \label{}
\end{align}
With this device it is possible to obtain convergence rates that take into account the regularity of realized volatility. While this is already quiet satisfying,  it is also of some interest to achieve  flexibility with regards the drift coefficient. Where the drift is concerned, regularity or even continuity for that matter is irrelevant; what is key is pathwise boundedness. The natural way to achieve flexibility in this respect is to group realized drift according to membership in  balls of radius $c$ in $\Linf$, that is,   
\begin{align} 
  \uball{c} := \{f \in \Linf: \Vert f \Vert_\infty \le c \}.
\end{align}
In this way we are able to  characterize the sample space according to the regularity of realized volatility  and the boundedness of realized drift.  This leads to the consideration of the asymptotic behavior of the estimator when  $\mu \in \uball{c}$ and $\sigma^2 \in  \hball{\alpha}{c}$ for  $c, \alpha > 0$\footnote{It is not essential to use different $c$'s for $\mu$ and $\sigma^2$.}. We denote such events by $\eball{\alpha}{c}$, i.e., 
\begin{align}
  \eball{\alpha}{c}  := \{ \omega : \mu(\omega) \in \uball{c}\} \cap \{\omega: \sigma^2(\omega) \in \hball{\alpha}{c}\}.\notag
  \label{}
\end{align}
So, an outcome $\omega$  is in \eball{\alpha}{c} if realized drift is caught between $-c$ and $c$ on \domain; realized volatility $\sigma^2(\omega)$  is \holder continuous with exponent $\alpha$,  and $\Vert \sigma^2(\omega) \Vert_\alpha \le c$. Note that the implication of the last statement is that there is some $c' \le c$ such that realized volatility is cought between $0$ and $c'$.     

We are now only left with the task of making the  obvious  modification to the usual  integrated mean square error criterion :
\begin{align}
  R_n(\alpha, c) := \e[\Vert \svn - \sigma^2\Vert^2\indvol],\notag
  \label{}
\end{align} 
where  $\indvol$ is the indicator function of $\eball{\alpha}{c}$, $\Vert \cdot \Vert$ is the \Ltwo norm, and $n$ is the observation frequency. Note that if $\eball{\alpha}{c} = \Omega$,  the expression above will just be the usual integrated mean square error criterion. By restricting the  volatility and the drift according to  events \eball{\alpha}{c}, the asymptotic properties of the estimator may be studied with full flexibility. That is we may  obtain results of the form  $\limsup_{n \to \infty}\tilde{n}_{n,\alpha,c} R_n(\alpha,c) < \infty$, where the rate may vary for different values of $\alpha$ and $c$. 

Much like the usual integrated mean square error risk, $R_n(\alpha,c)$ admits a decomposition in terms of an integrated square bias component and an integrated variance component. To see this, note the following:
\begin{align}
  R_n(\alpha, c ) & = \e \int_0^1  \{(\svn(t) - \sigma^2(t))\indvol\}^2 \D t \\ 
  & = \int_0^1 \e[ \{(\svn(t) - \sigma^2(t))\indvol\}^2] \D t \label{eq:fubini}\\ 
& = \int_0^1 \e[ (\svn(t) - \sigma^2(t))\indvol]^2 \D t \notag \\
& \quad + \int_0^1 \var[\svn(t)\indvol] \D t. \label{eq:msedecomp}
\end{align}
The equality in \eqref{eq:fubini} results from an interchange of the expectectation and integration operators justified by Fubini's theorem. The decomposition in line  \eqref{eq:msedecomp} results from the usual mean square error decomposition  into a square bias and a variance component for each  $t \in \domain$. The two summands in the last line are the bias and variance components and will be denoted $B_n^2(\alpha,c)$ and $V_n(\alpha,c)$, respectively; we obtain estimates for their rates of convergence below.
\begin{prop} \label{pr:consistency}
  Let $\{g, \tg\}$ be pair of dual Gabor generators constructed as in Lemma \eqref{le:gabor}.   Suppose the conditions in  Assumption \eqref{as:vol}  hold. If $g$ is Lipschitz continuous and  
  \begin{align}
    H_n^2 \Delta_n  + H_n^{-\alpha} \log H_n= o(1) \notag
    \label{}
  \end{align}
   then $R_n(\alpha,c)$ converges to 0, with 
  \begin{align}
    & B_n^2(\alpha,c)  = O(H_n^2\Delta_n  + H_n^{-2\alpha} \log^2 H_n)\notag \\
    & V_n(\alpha,c)  = O(H_n^2 \Delta_n),
    \label{}
  \end{align}
  where  $\Delta_n = 1/n$ is the step size, and $H_n$ is the order of magnitude of the number of estimated frame coefficients. 
\end{prop}
\proof{ See the appendix.}
\begin{remark}\mbox{}
  \begin{enumerate}  
    \item First, the above bounds are remarkably similar to those achievable using an orthonormal basis such as wavelets \citep{GenonCatalot1992}. The variance component is slower by a factor of $H_n$. This comes about because the vectors in a frame need not be orthogonal. The bias term is slower by a logarithmic factor. Intuitively, the logarithmic term shows up because we are expanding \sv using a frame, which may be thought of as containing some redundant term. Overall, the rate of convergence of the integrated mean square error differ only by logarithmic term. In practical implementations, this is an insignificant price to pay for the added flexibility and robustness gained by using frames. 
    \item Second, this result shows that the variance component of the MISE does not depend on the smoothness properties of either $\sigma^2$ and $g$.  
  \end{enumerate}
\end{remark}
\begin{prop}\label{pro:finite}
  Suppose the  price process is specified as in \eqref{eq:contsemimartingale}.  Assume the following conditions hold:
  \begin{enumerate}[label=\emph{(}\roman*\emph{)}]
    \item 
  the drift of $X$ satisfy with probability 1: $M^* : = \sup_{t \le 1} \vert b_t \vert $ is a finite valued random variable;
\item let $H_n \uparrow \infty$ such that 
  \begin{align}
    H_n n^{-1/2} \log(n) = o(1).\notag
    \label{}
  \end{align}
  \end{enumerate}
  Then
  \svnx converges in \Ltwo to \sv in probability.
\end{prop}
\begin{proof}
\begin{align}
  \svnx - \sigma^2(t) & = \sum_{(h,k) \in \Theta_n} (\cnhk\ - \chk)\;g_{h,k}(t)\notag\\
  & \quad -\sum_{(h,k) \not\in \Theta_n} \chk\;g_{h,k}(t),\label{eq:rone} 
\end{align}
  where 
\begin{align}  
  \cnhk &= \sum_{i =0}^{n-1} \btghki (X_{t_{i+1}} - X_{t_i})^2 \text{ and }\notag\\  
  \chk &= \int^1_0 \btghks \sigma^2(s) \D s. \notag
\end{align}
We tackle the summands in \eqref{eq:rone} in turn starting with the first one. But first let 
\begin{align}
  M_i := \int_{t_i}^{t_{i+1}} b(s) \D s, \quad \text{and} \quad  S_i := \int_{t_i}^{t_{i+1}} \sigma(s) \D W_s, \notag
\end{align}
and note that since $X_{t_{i+1}} - X_{t_i} = M_i + S_i$, it follows that
\begin{align}
  (X_{t_{i+1}} - X_{t_i})^2 &= M_i^2  
  + 2M_iS_i +   S_i^2.\notag 
\end{align}
So, \eqref{eq:rone} may be written as 
\begin{align}
  &\svnx - \sv(t) = B_{1,n}(t) + B_{2,n}(t) + B_{3,n}(t) + B_{4,n}(t), \notag
\end{align}
where
\begin{align}
  &B_{1,n}(t) :=  \sumt g_{h,k}(t) \left(\sum_{i=0}^{n-1} \btghki S_i^2  - \chk\right), \notag\\
  &B_{2,n}(t) := 2 \sumt g_{h,k}(t)\left(\sum_{i=0}^{n-1} \btghki S_i M_i \right), \notag\\
  &B_{3,n}(t) := \sumt g_{h,k}(t)\left( \sum_{i=0}^{n-1} \btghki M_i^2 \right), \notag\\
  &B_{4,n}(t) := - \sum_{\substack{(h,k) \not\in \Theta_n }}g_{h,k}(t)\chk. 
  \label{}
\end{align}
We will estimate the summands starting with $B_{4,n}(t)$. Note the following:
\begin{align}
  \sum_{\substack{(h,k) \not\in \Theta_n }}g_{h,k}(t)\chk &=  \sum_{\substack{(h,k) \not\in \Theta_n }}g_{h,k}(t)\inner{\sigma^2}{ \tghk} \notag\\ 
&\le c\modc{\tghk}{1/H_n}\log H_n  + c\modc{\sigma^2}{1/H_n} \log H_n\notag,
\end{align}
where the last  line follows from Theorem \eqref{th:fourone}. It from the \holder continuity of $\sigma$ that  that $\modc{\sigma^2}{1/H_n} \le c H_n^{-\alpha}$. Furthermore, by Lemma \eqref{lem:modtg} and the Lipschitz continuity of $g$ we have $\modc{\tghk}{1/H_n} \le c H^{-1}$.  So,  
\begin{align}
 B_{4,n}(t) =  O( H_n^{-\alpha}\log H_n).
  \label{eq:B4}
\end{align}
Note the generic use of the constant  $c$. In the sequel, we will use $c$ to denote the amalgamation of various constants resulting from multiple steps; this should be harmless since  constants are not asymptotically relevant. 

We now obtain an estimate for  $B_{3,n}(t)$. Note the following:
\begin{align}
  M_i^2 & = \left(\int_{t_i}^{t_{i+1}} b(s) \D s\right)^2 \notag\\
  & \le \left(M^* n^{-1}\right)^2 
  \label{eq:m}
\end{align}
 Now since  $g_{h,k}$ and \tghk  are bounded independently of $h$ and $k$, and $n\Delta_n = 1$, we have
\begin{align}
  B_{3,n}(t) = O( H_n \Delta_n).
  \label{}
\end{align}
  Now consider the following, 
\begin{align}
  S_i & = \int_{t_i}^{t_{i+1}} \sigma(s) \D W_s\notag \\
  & = O (\sqrt{n^{-1} \log(n)}),\label{eq:rthree}
\end{align}
for sufficiently large $n$ by \levy's modulus of continuity theorem. Hence
\begin{align}
  M_i S_i = O(\sqrt{n^{-3} \log(n)}).\label{eq:ms}
\end{align}
Since  $g_{h,k}$ and \tghk  are bounded independently of $h$ and $k$, and $n\Delta_n = 1$, we have
\begin{align}
  B_{2,n}(t) = O( H_n \sqrt{n^{-1} \log(n)}).
  \label{}
\end{align}
Now we tackle the final piece $B_{1,n}(t)$. Let 
\begin{align}
  A  := \sum_{i=0}^{n-1}\btghki S_i^2 -\int_{0}^{1} \sigma^2(s) \btghks \D s.
\end{align}
We will first obtain an upper bound for $A$; we proceed by adding and subtracting $\sumin\int_{t_i}^{t_{i+1}}\btghki \sigma^2(s)  \D s$  from  $A$ to yield: 
\begin{align}
  A & = \sumin\btghki \left(S_i^2 -\int_{t_i}^{t_{i+1}} \sigma^2(s) \D s\right)\notag \\
  &\quad +  \sumin\left(\int_{t_i}^{t_{i+1}} \sigma^2(s) \{\btghki - \btghks\} \D s\right) \notag\\
  &=:A_{1} + A_{2} \notag.
  \label{}
\end{align}
We obtain estimates in turn for the summands. By linearity of expectation 
\begin{align}
  A_{2} &= \sumin \int_{t_i}^{t_{i+1}} \sigma^2(s)  \{\btghki - \btghks\} \D s\notag\\
  &\le c\modc{\tghk}{\Delta_n},\notag
  \label{}
\end{align}
where $\modc{\tghk}{\Delta_n}$ is the modulus of continuity of $\tghk$ on an interval of length $\Delta_n$. By  Lemma \eqref{lem:modtg} and the Lipschitz continuity of $g$ we have, 
\begin{align}
  A_{2} \le c\modc{g}{\Delta_n} \le c \Delta_n.\notag 
  \label{}
\end{align}

Now, we obtain an estimate for $A_{1}$. First, let 
\begin{align}
  T_c = \inf \{ t >0 : \sigma^2(t) > c \}\notag.
  \label{}
\end{align}
By the regularity assumption on $\sigma$, $\p(T_c > t)$ can be made arbitrarily small by taking $c$ sufficiently large.  Hence, by the usual stopping time argument, it may be assumed that $\sigma^2(t)$ is bounded above by $c$ almost surely.  Now, let $D_i: \Omega\times[0,1] \to \real$ for $i = 0, \cdots, n-1$ be defined as follows:  
\begin{align}
&D_{i}(t) := \btghki\left(\int_{t_i}^{t} \sigma(u)\D W_u\right)\mathbbm{1}_{(t_i, t_{i+1}]}(t).
  \label{eq:Di}\\
&D_0(0) := 0.
\end{align}
So, $D_i(t)$ is 0 on $[0,1]$ except when $t$ is in $(t_i, t_{i+1}]$.


Now, using the integration by parts formula for semimartingales, we may write
\begin{align}
  S^2_i - \int^{t_{i+ 1}}_{t_i} \sigma^2(s) \D W_s = 2\int_{t_i}^{t_{i+1}}\left(\int_{t_i}^{s} \sigma(u)\D W_u\right)\sigma(s) \D W_s\notag
  \label{}
\end{align}
so that 
\begin{align}
  \e(\vert A_1 \vert )  &= 2\,\e\left[\sumin\int_{t_i}^{t_{i+1}}\btghki\left(\int_{t_i}^{s} \sigma(u)\D W_u\right)\sigma(s) \D W_s\right ]\notag \\
  & \le 2\,\e\left[\left\vert\int_0^{1}\sumin D_i(s) \sigma(s) \D W_s\right\vert \right].\notag 
  \label{}
\end{align}
Now using the fact that $\int_0^{t}\sumin D_i(s) \sigma(s) \D W_s$ is a  martingale, we may make an appeal to the BDG inequality to yield:
\begin{align}
  \e(\vert A_1 \vert) & \le c\e\left[\left\vert\int_0^{1}\left(\sumin D_i(s) \sigma(s)\right)^2 \D s\right\vert^{1/2} \right]\notag \\
  & \le c\e\left[\left\vert\int_0^{1}\sumin \{D_i(s) \sigma(s)\}^2 \D s\right\vert^{1/2} \right]\notag,
  \notag
  \label{}
\end{align}
where the last line follows because $D_i (s) D_j(s) = 0$ whenever $i \not= j$. Now if  we define $D^*_i := \sup_{t_i <s \le t_{i+1}} D_i(s)$, and use the fact that $\sigma^2$ is less than  $c$ before $T_c$ then
\begin{align}
  \e(\vert A_1\vert ) & \le c\e\left[\left\vert\sumin \Delta_n(D_i^*)^2 \right\vert^{1/2} \right] \notag\\
  & \le c\e\left[\left\vert\sumin \Delta_n(D_i^*)^2 \right\vert \right] \notag\\
  & \le c\Delta_n\sumin\e[(D_i^*)^2]   
  \label{eq:ra1}
\end{align}
where $c$ is a generic constant representing the bound on $\sigma^2$ and the BDG constant.  Note  from the definition of  $D_i$  that it  is itself a martingale, so we may bound  $D^*_i$  with yet another application of the BDG inequality. That is
\begin{align}
  \e((D^*_i)^2) &\le \e\left(\int^{t_{i+1}}_{t_i} \sigma^2(s) \D s\right) \notag\\
  &\le c \Delta_n. 
  \label{}
\end{align}
Hence, given any $\varepsilon > 0$,  $\p( \sup_{t \in [0,1]} \vert B_{j,n}(t) \vert > \varepsilon)  = O(H_n n^{-1}) + \p(T_c < 1)$.

Collecting the estimates for $B_{j,n}(t)$ for $j =1,\cdots,4$, it is easily seen that $\vert \svnx - \sigma^2(t) \vert$ tends to zero in probability uniformly  for all $t \in \domain$. 
\end{proof}

\begin{comment}
\begin{remark}\mbox{}
  \begin{enumerate}
    \item The adaptive approach is not the only way to specify this estimator. We could have instead said something like the volatility paths are all Holder continuous with exponent at least $\alpha$ and are all contained in a Holder ball of at most $c$. This would give us a minimal rate. The holder ball is actually not too bad either. 
    \item The approach taken here is a little more theoretical and frankly does not add much more in terms approachability of friendliness. But it is the more mature approach. It lacks usefulness but it is precise.
    \item The difficulties here are of serveral order. 1. The thing to be estimated \sv is itself random. 2. the realisations of the thing to be estimated are paths not single numbers. 3. the paths of the thing to be estimated can vary very widely with regards smoothness. For example, the Brownian process can produce paths of anysmoothness imaginable; of course some of this paths have zero probability. The point is there is no single smoothness criterion that captures all the possibilities adequately. The best we can do is to restrict the type of processes to be considered by say something like only processes whose paths is Holder continuous with exponent at least $\alpha > 0$, but this will exclude the Brownian motion as a possible driver. In fact all non-deterministic semimartingales.
    \item the adaptive approach seems like the less of two evils.
    \item Under the assumption that \sv is compactly supported, could it be taken  for granted that the modulus of continuity is proportional to the step size? 
    \item Zigmonds theorem says $|\hat{f}(\rho)| \le (1/2)\omega(\pi/|\rho|)$. This works for $2\pi$ periodic functions defined on the entire real line. It possibly holds when the subset of the real line considered is compact.
  \end{enumerate}
\end{remark}
\end{comment}




