\subsection{Finite activity \levy jumps}
In order to demonstrate that the global estimator of spot volatility is consistent, we will proceed in stages.  First suppose the price process specified in all generality in \eqref{eq:generalsemimartingale} experiences at most a finite number of jump in any finite time interval. That is we assume that $X$ has finite activity jumps, which is equivalent to $\nu$ being finite on the complement of   $\{0\}$. The finite activity assumption also implies that the price process may be expressed as  
\begin{align}
  X_t :=  \int_0^t b_s d s + \int_0^t \sigma_s dW_s +  \sum^{N_t}_{ i = 1} Y_i,
  \label{jumpfa}
\end{align}
where jump sizes $Y_i$ are \iid with distribution $F$, and $N$ is a Poisson process with intensity $\lambda$, independent of each $Y_i$. Under this conditions, we have the following:

\begin{prop}\label{pro:finite}
  Suppose the  price process is specified as in \eqref{jumpfa}. Let $\{g, \tg\}$ be pair of dual Gabor generators satisfying the conditions of Lemma \eqref{le:gabor} such that $g$ is Lipschitz continuous on the unit interval. Further assume the following conditions hold:
  \begin{enumerate}[label=\emph{(}\roman*\emph{)}]
    \item 
  the drift of $X$ satisfies with probability 1:
  \begin{align}
    \limsup_{\Delta_n \to 0} \frac{M^*}{(\Delta_n \log(1/\Delta_n))^{1/2}} \le  C< \infty, \notag
    \label{}
  \end{align}
where $M^* : = \sup_{1 \le i <n} \vert \int^{t_{i+1}}_{t_i} b_s \D s\vert$;
\item the diffusion coefficient satisfies with probability 1:  $\int^T_0 \sigma^2 \D s < \infty$ and 
  \begin{align}
\limsup_{\Delta_n \to 0} \frac{S^*}{\Delta_n} \le B <\infty, \notag
    \label{}
  \end{align}
  where $S^* := \sup_{1 \le i <n} \vert \int^{t_{i+1}}_{t_i} \sigma^2_s \D s\vert$;
\item and   $u_n \downarrow 0 $ is a sequence in $n$ such that  
\begin{align}
    \lim_{\Delta_n \to 0} \frac{\Delta_n \log (1/\Delta_n)}{u_n} = 0.\notag
    \label{}
  \end{align}
\item If $H_n \uparrow \infty$ satisfies 
  \begin{align}
    H_n (n^{-1} \log(n))^{1/2} = o(1)\notag
    \label{}
  \end{align}
  \end{enumerate}
  then
  \jvn as defined in \eqref{eq:jumpvolestimator}  converges in \Ltwo in probability to \sv.
\end{prop}
\begin{proof}
  First note that, there is a finite valued random variabe $\Lambda$ such that $C + M + 1 \le \Lambda$. Now, 
  let $X^c$ denote the continuous part of $X$ so that $X = X^c + J$, and 
  \begin{align}
  X^c_t = \int^t_0 b_s \D s + \int^t_0 \sigma_s \D W_s,
    \label{eq:contpart}
  \end{align}
  for $t$ in \domain. Denote
  \begin{align}
    & \svn{X^c}:=\sum_{(h,k) \in \Theta_n} \dnhk\;g_{h,k}(t), \qquad \forall t \in [0,1], \text{ where}\\
  &\dnhk := \sum_{i =0}^{n-1} \btghki (\Delta X^c_i)^2.  \notag
    \label{}
  \end{align}
We have
\begin{align}
  \int_0^1 & (\jvn  - \sv(t))^2\D t \notag\\
  & \le  2 \int_0^1  (\jvn  - \svn{X^c})^2\D t +   2\int_0^1  (\svn{X^c} - \sv(t) )^2\D t.
\end{align}
That the second summand on the right converges to 0 in probability is a result of Proposition (1.1). 
Now note that

\begin{align}
  \jvn{X}  - \svn{X^c}(t) = \sum_{(h,k) \in \Theta_n} (\anhk - \dnhk)\;g_{h,k}(t),\notag
  \label{}
\end{align}
and 
\begin{align}
  \anhk - \dnhk = \sum_{i =0}^{n-1} \btghki\{ (\dx)^2 \indx- (\Delta X^c_i)^2\}.\notag
  \label{}
\end{align}
By Theorem 3.1 of \cite{Mancini2009}, for almost all outcomes, there is $n'$ such that for all $n \ge n'$ 
\begin{align}
  \indx =\indn. \notag
  \label{}
\end{align}
Hence for almost all outcomes and sufficiently large $n$  
\begin{align}
   \anhk - \dnhk &\le \sum_{i =0}^{n-1} \btghki  (\Delta X^c_i)^2  \indnc\notag\\
& \le  c \Lambda \Delta_n \log(1/\Delta_n)\sum_{i =0}^{n-1}\indnc \notag \\
  \label{}
\end{align}
Hence for almost all outcomes,
\begin{align}
  \int^1_0\{\jvn  - \svn{X^c}\}^2 \D t & \le c \Lambda^2 H_n^2 \Delta_n^2\log^2(1/\Delta_n) N_1^2\notag\\
  & = O(H_n^2 \Delta_n^2\log^2(1/\Delta_n))\notag \\
  & \to 0.
  \label{}
\end{align}
\end{proof}
