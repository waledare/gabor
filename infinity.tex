\begin{comment} \subsection{Infinite activity \levy jumps}
We now turn to the case of a price process specified in full generality by \eqref{eq:generalsemimartingale}, that is the price process is a sum of a continuous and a discontinuous process with possibly  infinite activity. The infinite activity assumption is equivalent to  the statement that $\nu$ assigns  infinite measure to the complement of the singleton containing zero.    The following is the consistency Proposition in this more general framework:
\end{comment}
We now prove consistency for the estimator when the price process admits both large and small jumps. That is 
\begin{align}
  X_t = X_0 + X^c_t + J^l_t + J^s_t. \notag
  \label{}
\end{align}
where $J^l_t := (xI_{\{\vert x \vert > 1\}}) \ast \mu_t$ and  $J^s_t := (xI_{\{\vert x \vert \le 1\}}) \ast (\mu - \nu)_t$.
\begin{comment}
First, we make some obvious statements:
\begin{lem}\label{lem:est}
  Let $X$ be an \ito semimartingale meetig the requirements of Assumption \ref{as:vol} and \ref{as:nu}. Let $\{T_m, F^m\}$ be a localizing sequence for $(b, \sigma, F)$. Denote $(X^c_m)_t := X^c_{t \wedge T_m}$,  $(J^l_m)_t := J^l_{t \wedge T_m}$, and $(J^s_m)_t := J^s_{t \wedge T_m}$. Then,
  \begin{enumerate}
    \item $\e((\dxc_m)^p) = O(\Delta_n^{p/2}), \qquad p \in \{2, 4\}$,
    \item $\e((\djt_m)^2) = O(\Delta_n)$,
    \item $\e(\vert \djl_m \vert ) = O(\Delta_n)$.
  \end{enumerate}
\end{lem}
\begin{proof}
That $\e((\dxc_m)^p) = O(\Delta_n^{p/2})$  for $p = 2$ and 4  holds has already been demonstrated in Proposition \ref{pro:finite}. Now, since $J^s$ is a local martingale, we have by  the BDG inequality that,   for $t \in (t_i, t_{i + 1}]$, 
\begin{align}
\e( ( J^s_{t\wedge T_m} - J^s_{t_{i}\wedge T_m})^2) &= \e(  [J^s_{t\wedge T_m} - J^s_{t_{i}\wedge T_m}]) \notag \\
& = \e(\int_{t_i \wedge T_m}^{t\wedge T_m}\int_{\{ \vert x \vert \le 1\}} x^2 F_t(dx) dt)\notag \\
& \le \e(\int_{t_i\wedge T_m}^{t\wedge T_m}\int_\real (x^2 \wedge \vert x \vert ) F_t(dx) dt)\notag \\
& \le  \int_{t_i\wedge T_m}^{t\wedge T_m}\int_\real (x^2 \wedge \vert x \vert) F^m(dx) dt\notag\\
%& \le  (t - t_i)\int_\real x^2 F^m(dx) \notag\\
& =  O(\Delta_n)\notag.
\end{align}
Similarly, 
\begin{align}
\e( \vert J^l_{t\wedge T_m} - J^l_{t_{i}\wedge T_m} \vert) &\le \e( (\vert x\vert I_{\{\vert x \vert > 1\}}) \ast \mu((t_i\wedge T_m, t_{i+1}\wedge T_m] ) \notag\\
& = \e(\int_{t_i\wedge T_m}^{t\wedge T_m}\int_{\{ \vert x \vert > 1\}} \vert x\vert F_t(dx) dt)\notag \\
& \le  \int_{t_i\wedge T_m}^{t\wedge T_m}\int_\real (x^2 \vee \vert x \vert) F^m(dx) dt\notag\\
& =  O(\Delta_n)\notag.
  \label{}
\end{align}
\end{proof}
\end{comment}
\begin{comment}
First, we state some obvious results.
\begin{lem}
 Let the price process  $X$ be specified as in  \eqref{eq:semimartingale}. Then,
 \begin{enumerate}
   \item $(x^2 \wedge 1) \ast \mu$ is locally integrable.
   \item $(\vert x \vert I_{\{\vert x \vert > 1\}}) \ast \mu$ is locally integrable.
 \end{enumerate}
\end{lem}
The first statement follows because $(x^2 \wedge 1) \ast \mu_t$ is dominated by $ (x^2I_{\{\vert x \vert \le 1\}}) \ast \mu_t$ and $ (I_{\{\vert x \vert > 1\}}) \ast \mu_t$, which  are increasing processes with bounded jumps and, therefore, locally integrable.    
\end{comment}
We now give the main result of the paper.
\begin{prop} \label{pro:infinity}
  Let the price process  $X$ be specified as in  \eqref{eq:semimartingale}. We assume that the requirements of Assumption \ref{as:vol} and \ref{as:nu} are met. Let $\{g, \tg\}$ be pair of dual Gabor generators satisfying the conditions of Lemma \eqref{le:gabor} with $g$  Lipschitz continuous on the unit interval. Let  $\{H^n\}$ be an increasing sequence and   $\{u_n\}$  a decreasing sequence statisfying $u_n = O(\Delta_n^\beta)$ with   $0 <\beta<1$. If  
  \begin{align}
    &u_n^{-1/2}(H^n)^2\Delta_n^{1/2} = o(1), \notag\\
    &(H^n)^2u^{1/2}_n = o(1)
    \label{eq:smallo}
  \end{align} 
  \begin{comment}and   that $\nu$ satisfies 
   \begin{align}
    (x^2 \wedge u_n^{1/2}) \ast \nu_1 = o((H^n)^{-2}).
    \label{eq:smallo}
  \end{align}
  \end{comment}
  then  \jvn defined in \eqref{eq:jumpvolestimator} converges in \Ltwo in probability to \sv.
\end{prop}
\begin{proof}
  \begin{comment}  We wish to show that the random variable
  $\int_0^1 (\jvn - \sigma^2(t))^2\D t$ tends to zero in probability. The regularity conditions on $X$ and $\sigma^2$ imply that  $\sup_{t \in [0,1]} (\jvn - \sigma^2(t))^2$ is a random variable and that the previous claim would follow as soon as $\sup_{t \in [0,1]} (\jvn - \sigma^2(t))^2$ is shown  to converge to  zero in probability.\end{comment}
  We argue along the lines of Theorem 4 of  \cite{Mancini2009}. First,  consider the following decomposition of the process $X$:
  \begin{align}
    &X = X^f + J^s\label{eq:xj},\\
    &X^f = X^c + J^l\label{eq:xjc},
  \end{align}
  where 
    $X^c_t = \int^t_0 b_s \D s + \int^t_0 \sigma_s \D W_s$, 
    $J^l_t = (xI_{\vert x \vert > 1}) \ast \mu_t,$
    and $J^s_t = (xI_{\vert x \vert \le  1} )\ast (\mu - \nu)_t$. By localization, it is enough to assume $\sigma^4$ and  $b^4$   are integrable. 
    Let $t$ be a point in the unit interval, then 
    \begin{align}
      \jvn -  \sigma^2_t &= \sumt (\ceen{X} - \cee) \ghk(t) - \sumnt \cee \ghk(t),
      \label{eq:open}
    \end{align}
    with $\ceen{X}$ and $\cee$ defined by \eqref{eq:jumpvolestimator} and \eqref{eq:chk}, respectively. The last term tends to zero, almost surely, in \Ltwo as $n \to \infty$ because Gabor frames converge unconditionally. 
    
     To obtain a bound on the first item on the right of \eqref{eq:open}, we may use \eqref{eq:xj} to write
    \begin{align}
      \sumt &(\ceen{X} - \cee)\ghk(t)   = \sumt (w_{h,k} + x_{h,k} + y_{h,k} + z_{h,k}) \ghk(t),\label{eq:summands} 
    \end{align}
    where 
    \begin{align}
      &w_{h,k} :=  \sumin \btghki  (\dxf)^2 \indxff - \int^{1}_{0} \sigma^2(s) \btghks \D s \notag\\
      &x_{h,k} :=  \sumin \btghki(\dxf)^2 (\indx - \indxff) \notag\\
      &y_{h,k}  := 2 \sumin \btghki\dxf \djt \indx \notag \\
      &z_{h,k} := \sumin \btghki(\djt)^2\indx.
      \label{}
    \end{align}
    By Lemma \ref{lem:finite}, if $\delta > 0$ then %\begin{align}  
      $\p(\sup_{t \in [0,1]}\vert \sumt w_{h,k} \ghk(t) \vert > \delta) \to 0$ %\label{eq:w} \end{align} 
    as $n$ tends to infinity.  It remains to show that the last three terms on the right of \eqref{eq:summands} converge to zero in probability. Starting with the second summand, denote $A_i := \{(\dx)^2 \le u_n\}$,  $B_i   := \{(\dxf)^2 \le 4 u_n\}$ and note that $I_{A_i} - I_{B_i} = I_{A_i \cap B_i^c} - I_{A^c_i\cap B_i}$.  Hence, we may write 
    \begin{align}
      \sumt x_{h, k} \ghk(t) = \sumt \sumin \btghki (x^{i, 1} - x^{i, 2}) \ghk(t)  \notag
      \label{}
    \end{align}
   where $ x^{i, 1}  := (\dxf)^2 I_{A_i \cap B_i^c}$ and  $x^{i, 2} := (\dxf)^2I_{A^c_i\cap B_i}$.  
    It is now easily verified using the  reverse triangle inequality that   $A_i \cap B_i^c \subset \{\vert \djt \vert > u_n^{1/2}\}$. So that, 
    \begin{align}
      (\dxf)^2 I_{A_i \cap B_i^c} &\le (\dxf)^2 I_{\{(\djt)^2 > u_n\}}\\
       & \le 2 (\dxc)^2 I_{\{(\djt)^2 > u_n\}}
       + 2 (\djl)^2 I_{\{(\djt)^2 > u_n\}}\notag\\
       & =: v_i + w_i. 
      \label{eq:longineq}
    \end{align}
    It thus follows that   
    \begin{align}
      \sumt \sumin \btghki x^{i,1} \ghk(t) \le \sumt \left(\sumin \btghki (v_i + w_i) \right) \ghk(t). \notag 
  \end{align}
    \begin{comment}
    where \begin{align} &v_n :=  2c H^n \Lambda n^{-1} \log(n) \sumin I_{\{(\djt)^2 > u_n\}} \notag \\ & w_n: = c H^n \sumin  (\djl)^2 I_{\{(\djt)^2 > u_n\}}\notag \end{align}  where $c$ is a sufficiently large constant, and $\Lambda$ is a finite-valued random variable satisfying  $\Lambda \ge  \sup_{t \in \domain} \vert b(t)\vert  + C $, where $C^{1/2}$ is the finite-valued random variable from Lemma \ref{lem:mylevy}.  Let  $\delta > 0$ be given, put  $x_n(t)  :=  \sumt\left(\sumin  \btghki(\dxf)^2 I_{A_i \cap B_i^c})\right)\ghk(t)$  and note that 
    \begin{align}
      \p &\left(  \sup_{t \in \domain}\vert x_n (t) \vert  > \delta\right) \le \p(v_n > \delta/2) + \p(w_n > \delta/2) \notag. 
      \label{}
    \end{align}
  Now let  $\varepsilon > 0$ be given and note that because  $\Lambda$ is almost surely finite,   there is a sufficiently large $K > 0$ such that $\p(\Lambda > K) \le \varepsilon/2$. Hence, 
\end{comment}
We  proceed by using \holder's inequality and \eqref{eq:secmo}   to write
\begin{align} 
  \e(v_i) & \le c (\e( (\dxc)^4))^{1/2} \p( (\djt)^2 > u_n)^{1/2} \notag \\ 
  & \le c u_n^{-1/2} \e( (\dxc)^4)^{1/2} \e( (\djt)^2)^{1/2} \notag \\ 
  & \le c u_n^{-1/2}\Delta_n^{3/2} \label{eq:vi}.
\end{align}
Hence,  by Markov's inequality and the boundedness of $\ghk$ \begin{align} \sup_{t \in \domain}\sumt \sumin \btghki v_i \ghk(t) = O_P(u_n^{-1/2}H^n \Delta_n^{1/2}),\end{align} 
which by assumption tends to zero in probability.

As for the term involving $w_i$,   recall  that because $\mu$ is a Poisson random measure, if $A$ and  $B$ are disjoint measurable sets  in $\real^+\times \real$    then $\mu(A)$ is independent of $\mu(B)$. Using this fact, we may write  given $\eta > 0$    
\begin{align} 
  \p( \sup_{t \in \domain}\vert &\sumt \sumin \btghki w_i \ghk(t)\vert > \eta) \notag \\
  &\le \p\left(\cup_i\{\mu( (t_i, t_{i + 1}] \times \{\vert x \vert > 1\}) > 0   , (\djt)^2 > u_n\}\right) \notag \\ 
  & \le n \p(\mu( [0, t_1] \times \{\vert x \vert > 1\}) > 0 ) E( ( J^s_{t_1})^2) u_n^{-1} \notag \\ 
  & \le c \Delta_n u_n^{-1}\notag,
\end{align} 
which clearly tends to zero in $n$. This concludes the demonstration that $ \sumt \sumin \btghki x^{i,1} \ghk(t)$ tends to zero in probability.  
  \begin{comment}
\p(v_n > \delta /2) & \le 2cKH^n\delta^{-1}E(n^{-1} \log(n) \sumin I_{\{(\djt)^2 > u_n\}}) + \p(\Lambda > K) \notag  \\ & = 2cK H^n\delta^{-1}\log(n) \p( (\Delta_{1/n}J^s)^2 > u_n)  + \varepsilon/2 \notag \\ &\le  2cK H^n\delta^{-1}\log(n) E((\Delta_{1/n} J^s)^2))u_n^{-1}   + \varepsilon/2 \notag \\ &\le 2cKH^n\delta^{-1} \log(n) n^{-1}\kappa u_n^{-1} +  \varepsilon/2\label{eq:asabove} \end{align} where $\kappa := E((\Delta_1 J^s)^2)) < \infty$.  Obviously there is a large enough $n$ such that the first expression above is less than or equal to $\varepsilon/2$.
\end{comment}
\begin{comment} Moreover, because $\delta >  0$,  \begin{align} \p(w_n > \delta/2) & \le \p\left(\cup_i\{I_{\{\vert x \vert > 1\}}\ast \mu( (t_i, t_{i + 1}] \times \real) > 0   , (\djt)^2 > u_n\}\right) \notag \\ & \le n \p(\mu( [0, 1/n] \times \{\vert x \vert > 1\}) > 0 ) E( (\Delta_1 J^s)^2) u_n^{-1} \notag \\ & \le c n^{-1} \kappa u_n^{-1}\notag,\end{align} which clearly tends to zero in $n$. 
\end{comment}
To tackle the term $ \sumt \sumin \btghki x^{i,2} \ghk(t)$, we start with the following definitions:
\begin{align}
&\Omega_n^1 := \{ \omega: \vert\Delta_i X^c(\omega)\vert <   1-  2u_n^{1/2} , \text{ for all }i < n   \},\notag \\
%\{\omega : \vert \dxf(\omega) \vert > 2u^{1/2}_n, \forall  i < n\},\notag \\ 
&\Omega_n^2 := \{\omega : \mu(\omega,  (t_i, t_{i+1}] \times \{\vert x \vert > 1\} ) \le 1, \forall i <n  \},\notag 
\end{align}
$\forall n \in \nats$. These sets are clearly measurable. Denote 
\begin{align}
  \Omega_n   := \Omega_n^1 \cap \Omega_n^2. \label{eq:myman}
\end{align}
Since there can be at most a finite number of jumps  larger than 1  in magnitude    on \domain,  and  $1 - 2u_n^{1/2} \uparrow 1$ while $ \Delta_iX^c \downarrow  0 $  uniformly  on \domain, it follows that $ \p(\Omega_n) \to 1$ as $n \to \infty$. Now note that 
\begin{align}
  \notag
  A_i^c \cap B_i \cap \Omega_n & \subset  \{ (\dxc + \djt)^2 > u_n \} \notag\\
  &\subset \{(\dxc)^2 > u_n/4\} \cup \{(\djt)^2 > u_n/4\}\notag .
  \label{}
\end{align}
Hence, by successive applications of \holder and Markov inequalities,   
\begin{align}
  \notag
 \e(&(\dxf)^2 I_{A_i^c \cap B_i \cap \Omega_n} ) = \e((\dxc)^2 I_{A_i^c \cap B_i \cap \Omega_n} )\notag \\
 &\le \e((\dxc)^2 I_{\{(\dxc)^2 > u_n/4\}}) + \e((\dxc)^2 I_{\{(\djt)^2 > u_n/4\}}) \label{} \notag\\
 & \le c \Delta_n^{3/2} u_n^{-1/2}.\notag
\end{align}
Let $\eta$ be a given positive number; it is now clear that
\begin{align}
  \p(\sup_{t \in \domain} \vert \sumt\sumin  \btghki x^{i,2} \ghk(t)\vert  > \eta) \le  \p(\Omega_n^c)  + c u^{-1/2}_nH^n\Delta_n^{1/2}, \notag
  \label{}
\end{align}
which tends to zero.  This completes the demonstration that \begin{align}\label{eq:x} \p(\sup_{t \in [0,1]}\vert \sumt x_{h,k} \ghk(t) \vert > \eta) \to 0. \end{align}
Now we show the third summand in  \eqref{eq:summands} tends to zero. First, denote $C_i := \{( \djt )^2 \le 4 u_n\}$, $p_{h,k} : = 2 \sumin \btghki\dxf \djt I_{A_i \cap C_i}$, and $q_{h,k} := 2 \sumin \btghki\dxf \djt I_{A_i \cap C_i^c}$. Clearly, 
\begin{align} 
  \sumt y_{h,k} \ghk(t)  = \sumt (p_{h,k} + q_{h,k}) \ghk(t).\notag
\end{align}
Treating the term involving $q_{h,k}$ first, note that by the reverse triangle inequality, we may  write $A_i \cap C_i^c \subset \{ u_n^{1/2} < \vert \dxf \vert  \} \subset \{ u_n^{1/2}/2 < \vert \dxc \vert  \} \cup  \{ u_n^{1/2}/2 < \vert \djl \vert  \} =: G^1_i \cup G^2_i\notag$. So that 
\begin{align}
  \dxf \djt I_{A_i \cap C_i^c} &\le \dxf \djt  (I_{G^1_i} +  I_{G^2_i}) \notag\\
  & \le \dxc \djt(I_{G^1_i} +  I_{G^2_i}) + \djl \djt(I_{G^1_i} +  I_{G^2_i}) \notag \\
  & =: \gamma_i^1 + \gamma_i^2 + \gamma_i^3 + \gamma^4_i.\notag 
\end{align}
Hence, \begin{align} \notag \sumt q_{h,k} \ghk(t) \le \sumt (\sumin \btghki (\gamma_i^1 + \gamma_i^2 + \gamma_i^3 + \gamma^4_i)) \ghk(t). \end{align} We show in turn that each summand converges to zero. First, observe that
\begin{align}
  \e(\gamma_i^1) &\le  \e( (\dxc  I_{G^1_i})^2)^{1/2} \e( (\djt)^2)^{1/2} \notag\\
  &\le  \e( (\dxc)^4)^{1/4}  E(I_{G^1_i})^{1/4} \e( (\djt)^2)^{1/2} \notag\\
  & \le c \Delta_n^{1/2} (u^{-1/2}_n \Delta_n^{1/2}) \Delta_n^{1/2} \notag \\
  & \le c u^{-1/2}_n \Delta_n^{3/2}.
  \label{}
\end{align}
Hence, given positive $\eta$, 
\begin{align} 
  \p( \sup_{t \in \domain}\vert \sumt \sumin \btghki \gamma^1_i \ghk(t)\vert > \eta) \le cH^n (u^{-1}_n\Delta_n)^{1/2}. \label{eq:g1} \end{align}
Secondly, we have 
\begin{align}
  \e(\gamma_i^2) & =  \e(\dxc \djt I_{G^2_i}) \notag\\
  & \le \e((\dxc)^2 I_{G^2_i})^{1/2} \e((\djt)^2)^{1/2}\notag \\
  &\le \e((\dxc)^4)^{1/4} \p(\djl > u_n^{1/2}/2)^{1/4} \e((\djt)^2)^{1/2}\notag\\
  & \le  c u^{-1/8}_n \Delta_n^{5/4}. \notag
\end{align}
So that given positive $\eta$,
\begin{align} 
  \label{eq:g2}
  \p( \sup_{t \in \domain}\vert \sumt \sumin \btghki \gamma^2_i \ghk(t)\vert > \eta) \le cH^n (u^{-1/2}_n\Delta_n)^{1/4}.  \end{align}
Moreover, 
\begin{align}
\p( \sup_{t \in \domain}\vert &\sumt \sumin \btghki \gamma^3_i  \ghk(t)\vert > \eta) \notag \\
&\le \p(    \cup_i\{ \mu( (t_i, t_{i+1}] \times \{\vert x \vert > 1 \} ) > 0,  (\dxc)^2 > u_n/4\}) \notag \\ & \le c \Delta_n u_n^{-1}. \label{eq:g3}\end{align}
Finally,
\begin{align}
\p( \sup_{t \in \domain}\vert &\sumt \sumin \btghki \gamma^4_i  \ghk(t)\vert > \eta) \notag \\
&\le \p(    \cup_i\{ \mu( (t_i, t_{i+1}] \times \{\vert x \vert > 1 \} ) > 0,  (\djt)^2 > u_n/4\}) \notag \\ & \le c \Delta_n u_n^{-1}.\label{eq:g4}\end{align}
\begin{comment}

$\Omega_q := \{\omega : \mu(\omega, \domain  \times \{\vert x \vert > 1\} ) \le q \}$, for     $q \in \nats$. It is easily seen that $\p(\Omega_q) \to 1$ as $q \to \infty$. Set $\Omega(n, q) := \Omega_n \cap \Omega_q$. Now note that   $\e(\gamma^3_i I_{\Omega(n,q)}) \le q \e(\djt I_{G^1_i}) \le c q \Delta_n^{3/2} u_n^{-1/2}$. So that given positive $\eta$,
\begin{align} 
  \p( \sup_{t \in \domain}&\vert \sumt \sumin \btghki \gamma^3_i \ghk(t)\vert > \eta) \notag \\ &\le \p(\Omega(n,q)^c) + cqH^n (u^{-1}_n\Delta_n)^{1/2}, \notag \end{align}
which can be made arbitrarily small by  choosing $n,q$ large enough and letting $n \to \infty$.  Similarly note that $\e(\gamma^4_i I_{\Omega(n,q)}) \le q \e(\djt I_{G^2_i}) \le c q \Delta_n u_n^{-1/4}$. So that 
\begin{align} 
  \p( \sup_{t \in \domain}\vert \sumt \sumin \btghki \gamma^4_i \ghk(t)\vert > \eta) \le \p(\Omega(n,q)^c) + cqH^n u^{-1/4}_n\Delta_n, \notag \end{align}
which can be made as small as desired. This completes the demonstration  that 
\begin{align}
  \p(\sup_{t \in \domain} \vert \sumt q_{h,k} \ghk(t) \vert > \eta) \to 0.\notag
  \label{}
\end{align}
%\begin{comment}
Now, arguing as in Theorem 4.1 of \cite{Mancini2009}, note that on $ A_i \cap C_i^c$, it is the case that  $2u_n^{1/2} - \vert \dxf \vert < \vert \djt\vert - \vert \dxf\vert \le \vert \dx \vert \le u_n^{1/2}$, so that $u_n^{1/2} < \vert \dxf \vert < \vert \djl \vert + \vert \dxc \vert $. In turn, the last inequality implies that either $\vert \djl \vert > u^{1/2}_n/2$ or  $\vert \dxc \vert > u^{1/2}_n/2$. Now, for sufficiently large $n$, it is almost surely never the case that $\vert \dxc \vert > u^{1/2}_n/2$ for some $i$,  $0 \le  i  \le n -1$. Hence,  for positive $\delta$, \begin{align} \p(\vert&\sumt q_{h,k} \ghk(t) \vert > \delta/2) \notag \\ &\le \p(    \cup_i\{ \mu( (t_i, t_{i+1}] \times \{\vert x \vert > 1 \} ) > 0,  (\djt)^2 > u_n\}) \notag \\ & \le c n^{-1} \kappa u_n^{-1}.\end{align}
\end{comment}
%\end{comment}
We conclude by  reference to the estimates in \eqref{eq:g1}, \eqref{eq:g2}, \eqref{eq:g3}, and \eqref{eq:g4} that  $\sup_{t \in \domain} \vert \sumt q_{h,k} \ghk (t) \vert $ tends to zero in probability.

We now show that $\sumt p_{h,k} \ghk (t)$ tends to zero uniformly  in probability.
To that end, let $\Psi_n := \{\omega: \vert \dxc(\omega) \vert > u_n^{1/2} \text{ for some } i < n\}$.  It now follows by Markov's inequality that 
\begin{align}
  \p(\Psi_n ) &\le \sumin \p(\vert \dxc \vert > u_n^{1/2}) \notag \\
  & \le u_n^{-3/2(1 - \beta) } \sumin \e( (\dxc)^{ 3/(1 - \beta)} ) \notag\\
  & \le  c \Delta_n^{1/2}.
  \label{}
\end{align}
Hence, $\p(\Psi_n ) \to 0$. On $A_i \cap C_i \cap \Psi_n^c$, it is easily seen that $\vert \djl \vert - \vert \dxc + \djt \vert < \vert \dx \vert \le u_n^{1/2}$, so that $\vert \djl\vert \le u^{1/2}_n + \vert \dxc \vert + \vert \djt\vert$. It is therefore the case that  $ \vert \djl \vert  = O(u^{1/2}_n)$. Let $r_{h,k} :=  2 \sumin \btghki\dxc \djt I_{A_i \cap C_i \cap \Psi_n^c}$ and $s_{h,k} := 2 c u_n^{1/2}\sumin \btghki \djt I_{A_i \cap C_i \cap \Psi_n^c}$. Then given  $\delta > 0$ and $\varepsilon > 0$, 
\begin{align}
  &\p(\sup_{t \in \domain}\vert\sumt p_{h,k} \ghk(t) \vert > \delta)\notag \\&  \le \p(\Psi_n) + \p(\sup_{t \in \domain} \vert\sumt r_{h,k} \ghk(t) \vert > \delta/2)  \notag \\
  &\quad + \p(\sup_{t \in \domain} \vert\sumt s_{h,k} \ghk(t) \vert > \delta/2).
  \label{eq:phk}
\end{align}
Now consider that $\sumt r_{h,k} \ghk(t) \le  c H^n \sumin \vert \dxc \djt  I_{A_i \cap C_i \cap \Psi^c} \vert$; this implies that  
\begin{align}
  \p(&\vert\sumt r_{h,k} \ghk(t) \vert > \delta/2) \le \p(c H^n  \sumin \vert \dxc \djt I_{A_i \cap C_i \cap \Psi^c} \vert  > \delta/2)\notag\\& \le \p\left( \left(\sumin (\dxc)^2\right)^{1/2} \left(\sumin ( \djt I_{A_i \cap C_i \cap \Psi^c})^2\right)^{1/2} > \delta (2H^nc)^{-1}\right).\notag
  \label{}
\end{align}
We now use the well-known fact  that  $\sumin (\dxc)^2 (t)$ converges to $\int_0^t\sigma^2(s)\D s$ in probability uniformly on compact intervals \citep[Theorem II.22]{Protter2004}. That is, there is a sufficiently large $N$   and $C$ such that if $n$ is larger than or equal to $N$ then   $ \p(\vert (\sumin (\dxc)^2)^{1/2}  - (\int_0^1\sigma^2(s) \D s)^{1/2}\vert > C ) \le  \varepsilon/12$, and because integrated volatility is almost surely finite, there is a sufficiently large $K$ satisfying  $K/2 > C$ such that $\p( \int_0^1\sigma^2(s) \D s > K/2) \le \varepsilon/12.$ Hence, we may write
\begin{align}
\p(&\vert\sumt r_{h,k} \ghk(t) \vert > \delta/2) \notag \\%&\le \p\left(\sumin ( \djt I_{A_i \cap C_i})^2 > \delta^2 (4K H^nc)^{-2}\right) + \varepsilon/2\notag\\
&\le  \p\left( (x^2I_{\{\vert x\vert \le 1 \wedge 2 u_n^{1/2}\}})\ast\mu_1 > \delta^2 (K H^nc)^{-2}\right) + \varepsilon/6\notag\\
&\le  c (H^n)^{2}\e\left( (x^2I_{\{\vert x\vert \le 1 \wedge 2 u_n^{1/2}\}})\ast\mu_1  \right) + \varepsilon/6\notag\\
&\le c(H^n)^{2} (x^2I_{\{\vert x\vert \le 1 \wedge 2 u_n^{1/2}\}})\ast\nu_1   + \varepsilon/6\notag 
%&\le c(H^n)^{2} u^{1/2}_n   + \varepsilon/2\notag
  \label{}
\end{align}
which for sufficiently large $n$ is  less than $\varepsilon/3$ by Assumption \ref{as:nu} and \eqref{eq:smallo}.
Now it is easily seen that for sufficiently large $c$
\begin{align}
  \p(&\vert \sumt s_{h,k} \ghk(t) \vert > \delta/2) \le \p( \sumin  \djt I_{A_i \cap C_i \cap \Psi^c} > ( c H^nu_n^{1/2})^{-1}\delta) \notag\\
  &\le  c u_n (H^n)^2 \e\left( (x^2I_{\{\vert x\vert \le 1 \wedge 2 u_n^{1/2}\}})\ast\mu_1  \right)\notag \\
  & \le   c u_n (H^n)^2  (x^2I_{\{\vert x\vert \le 1 \wedge 2 u_n^{1/2}\}})\ast\nu_1
  \label{}
\end{align}
which, as above, is  eventually less than $\varepsilon/3$. Hence, each summand on the right hand side of \eqref{eq:phk} tends to zero. This concludes the demonstration that
\begin{align}
  \notag
  \p(\sup_{t \in [0,1]}\vert \sumt y_{h,k} \ghk(t) \vert > \delta) \to 0.\label{eq:y}
\end{align}
We  now tackle the last remaining summand in \eqref{eq:summands}.  Note that we may write   $z_{h,k} = a_{h,k} + b_{h,k}$ with $a_{h,k} :=  \sumin \btghki(\djt)^2I_{A_i \cap C_i}$ and  $b_{h,k} := \sumin \btghki(\djt)^2I_{A_i \cap C_i^c}$. Then for positive $\delta$ 
\begin{align}
  \p(\sup_{t \in \domain}&\vert \sumt z_{h,k} \ghk(t) \vert   > \delta)  \le \p(\sup_{t \in \domain}\vert \sumt a_{h,k} \ghk(t) \vert > \delta/2)\notag \\ 
  & \quad + \p(\sup_{t \in \domain}\vert \sumt b_{h,k} \ghk(t) \vert > \delta/2)\notag.
\end{align}
Now consider the event $\Omega_n$ from \eqref{eq:myman}, and note that  $A_i \cap C_i^c \cap \Omega_n \subset  \{\mu ( (t_i, t_{i + 1}] \times \{\vert x \vert > 1\} > 0\} \cap C_i^c$. Hence,   
\begin{align}
  \p(&\vert \sumt b_{h,k} \ghk(t) \vert > \delta/2) \notag \\ &\le
\p\left(\cup_i\{I_{\{\vert x \vert > 1\}}\ast \mu( (t_i, t_{i + 1}] \times \real) > 0   , (\djt)^2 > 4u_n\}\right)  + \p(\Omega_n^c)\notag \\
&\le n \p(I_{\{\vert x \vert > 1\}}\ast \mu( [0, t_1] \times \real) > 0) \e(( J^s_{t_1})^2)(4u_n)^{-1} + \p(\Omega_n^c)\notag \\
&\le c \Delta_n u^{-1}_n + \p(\Omega_n^c).
\end{align}
which can be made as small as desired. Now consider
\begin{align}
  \p(&\vert \sumt a_{h,k} \ghk(t) \vert > \delta/2) \notag \\ &\le \p( \sumin (\djt)^2 I_{\{\vert \djt \vert \le 2u_n^{1/2} \}} > \delta (2 c H^n)^{-1} )\notag\\
  &\le c H^n\e\left( x^2I_{\{\vert x\vert \le 1 \wedge 2 u_n^{1/2}\}}\ast\mu_1  \right)\notag \\
  &\le   c H^n (x^2I_{\{\vert x\vert \le 1 \wedge 2 u_n^{1/2}\}}\ast\nu)_1\notag \\
  &\le c H^n u^{1/2}_n \notag
\notag
\end{align}
which can be made arbitrarily small by the constraints on $H^n$. This completes the demonstration that  
\begin{align}
  \p(\sup_{t \in [0,1]}\vert \sumt z_{h,k} \ghk(t) \vert > \delta) \to 0.\label{eq:z}
\end{align}
\begin{comment}
Meanwhile, on $A_i \cap C_i$, it is easily seen that $\vert \djl \vert - \vert \dxc + \djt \vert < \vert \dx \vert \le u_n^{1/2}$, so that
\begin{align}
  \notag
  \dxf \djt I_{A_i \cap C_i} &\le \dxf \djt I_{\{\vert \djl \vert \le 4 u^{1/2}_n\}}\\
   &\le \dxc \djt I_{\{\vert \djl \vert \le 4 u^{1/2}_n\}} +  \djl \djt I_{\{\vert \djl \vert \le 4 u^{1/2}_n\}}\notag\\
   & =: \theta_i^1 + \theta_i^2\notag. 
  \label{}
\end{align}
Arguing as above, it is easily verified that $\e(\theta_i^1)  \le c \Delta_n^{5/4} u_n^{-1/8}$  so that 
\begin{align} 
  \p( \sup_{t \in \domain}\vert \sumt \sumin \btghki \theta^1_i \ghk(t)\vert > \eta) \le cH^n (u^{-1/2}_n\Delta_n)^{1/4} \notag \end{align}
Similarly,  $\e(\theta_i^2 I_{\Omega(n,q)}) \le q c \Delta_n u^{-1/4}_n$ so that  
\begin{align} 
  \p( \sup_{t \in \domain}\vert \sumt \sumin \btghki \theta^2_i &\ghk(t)\vert > \eta) \notag \\ &\le \p(\Omega(n,q)^c) + cH^n (u^{-1/2}_n\Delta_n)^{1/4} \notag \end{align}
Hence, 
\begin{align}
  \p(\sup_{t \in \domain} \vert \sumt p_{h,k} \ghk(t) \vert > \eta) \to 0.\notag
  \label{}
\end{align}
This completes the demonstration  that 
\begin{align}
  \p(\sup_{t \in \domain} \vert \sumt y_{h,k} \ghk(t) \vert > \eta) \to 0.\notag
  \label{}
\end{align}

$\vert \djl\vert \le u^{1/2}_n + \vert \dxc \vert + \vert \djt\vert \le 4u^{1/2}_n$. So that  On the other hand, $ \vert \djl \vert < u_n^{1/2} + \Lambda n^{-1/2} \log^{1/2}(n) + 2 u_n^{1/2} = O(u^{1/2}_n)$. Let $r_{h,k} :=  2 \sumin \btghki\dxc \djt I_{A_i \cap C_i}$ and $s_{h,k} := 2 c u_n^{1/2}\sumin \btghki \djt I_{A_i \cap C_i}$. Then 
\begin{align}
 \p(\vert&\sumt q_{h,k} \ghk(t) \vert > \delta/2)\notag \\&  \le\p(\vert\sumt r_{h,k} \ghk(t) \vert > \delta/4) + \p(\vert\sumt s_{h,k} \ghk(t) \vert > \delta/4).\notag
  \label{}
\end{align}
Now consider that $\sumt r_{h,k} \ghk(t) \le  c H^n \sumin \dxc \djt I_{A_i \cap C_i}$, which implies that  
\begin{align}
  \p(&\vert\sumt r_{h,k} \ghk(t) \vert > \delta/4) \le \p(c H^n \vert \sumin \dxc \djt I_{A_i \cap C_i} \vert  > \delta/4)\notag\\& \le \p\left( \left(\sumin (\dxc)^2\right)^{1/2} \left(\sumin ( \djt I_{A_i \cap C_i})^2\right)^{1/2} > \delta (4H^nc)^{-1}\right).\notag
  \label{}
\end{align}
It is a well known fact that  $\sumin (\dxc)^2 (t)$ converges to $\int_0^t\sigma^2(s)\D s$ in probability uniformly on the unit interval. Hence, there is a sufficiently large $N$ such that if $n > N$ then $ \p(\vert (\sumin (\dxc)^2)^{1/2}  - (\int_0^1\sigma^2(s) \D s)^{1/2}\vert > \delta ) \le  \varepsilon/4$, and because integrated volatility is almost surely finite, there is a sufficiently large $K$ satisfying  $K/2 > \delta$ such that $\p( \int_0^1\sigma^2(s) \D s > K/2) \le \varepsilon/4.$ Hence, we may write
\begin{align}
\p(&\vert\sumt r_{h,k} \ghk(t) \vert > \delta/4) \notag \\
  &\le \p\left(\sumin ( \djt I_{A_i \cap C_i})^2 > \delta^2 (4K H^nc)^{-2}\right) + \varepsilon/2\notag\\
  &\le  \p\left( (x^2I_{\{\vert x\vert \le 1 \wedge 2 u_n^{1/2}\}})\ast\mu_1 > \delta^2 (4K H^nc)^{-2}\right) + \varepsilon/2\notag\\
&\le \delta^{-2} (4K H^nc)^{2}\e\left( (x^2I_{\{\vert x\vert \le 1 \wedge 2 u_n^{1/2}\}})\ast\mu_1  \right) + \varepsilon/2\notag\\
&\le \delta^{-2} (4K H^nc)^{2} (x^2I_{\{\vert x\vert \le 1 \wedge 2 u_n^{1/2}\}})\ast\nu_1   + \varepsilon/2\notag
  \label{}
\end{align}
which for sufficiently large $n$ is  less than $\varepsilon$ by \eqref{eq:smallo}.

Now it is easily seen that for sufficiently large $c$
\begin{align}
  \p(&\vert \sumt s_{h,k} \ghk(t) \vert > \delta/4) \le \p( \sumin  \djt I_{A_i \cap C_i} > (8 c u_n^{1/2})^{-1}\delta) \notag\\
  &\le (64 c^2 u_n)\delta^{-2} \e\left( (x^2I_{\{\vert x\vert \le 1 \wedge 2 u_n^{1/2}\}}\ast\mu)_1  \right)\notag \\
  & \le  (64 c^2 u_n)\delta^{-2} (x^2I_{\{\vert x\vert \le 1 \wedge 2 u_n^{1/2}\}}\ast\nu)_1
  \label{}
\end{align}
which, as above, is  less than $\varepsilon/4$ for sufficiently large $n$.
Hence, 
\begin{align}
  \notag
  \p(\sup_{t \in [0,1]}\vert \sumt y_{h,k} \ghk(t) \vert > \delta) \to 0.\label{eq:y}
\end{align}

Next,  write $z_{h,k} = a_{h,k} + b_{h,k}$ where $a_{h,k} :=  \sumin \btghki(\djt)^2I_{A_i \cap C_i}$ and  $b_{h,k} := \sumin \btghki(\djt)^2I_{A_i \cap C_i^c}$. Then 
\begin{align}
  \p(&\vert \sumt z_{h,k} \ghk(t) \vert   > \delta) \notag \\ &\le \p(\vert \sumt a_{h,k} \ghk(t) \vert > \delta/2) + \p(\vert \sumt b_{h,k} \ghk(t) \vert > \delta/2)\notag.
\end{align}
In the first instance,
\begin{align}
  \p(&\vert \sumt b_{h,k} \ghk(t) \vert > \delta/2) \notag \\ &\le
\p\left(\cup_i\{I_{\{\vert x \vert > 1\}}\ast \mu( (t_i, t_{i + 1}] \times \real) > 0   , (\djt)^2 > 4u_n\}\right) \notag \\
&\le n \p(I_{\{\vert x \vert > 1\}}\ast \mu( [0, 1/n] \times \real) > 0) \e((\djt)^2)(4u_n)^{-1}\notag \\
&\le cn^{-1}\kappa  u^{-1}_n.
\end{align}
which can be made as small as desired. Now consider
\begin{align}
  \p(&\vert \sumt a_{h,k} \ghk(t) \vert > \delta/2) \notag \\ &\le \p( \sumin (\djt)^2 I_{\{\vert \djt \vert \le 2u_n^{1/2} \}} > \delta (2 c H^n)^{-1} )\notag\\
  &\le\delta^{-1} (2 c H^n)\e\left( x^2I_{\{\vert x\vert \le 1 \wedge 2 u_n^{1/2}\}}\ast\mu_1  \right)\notag \\
  &\le \delta^{-1} (2 c H^n) (x^2I_{\{\vert x\vert \le 1 \wedge 2 u_n^{1/2}\}}\ast\nu)_1
\notag
  \label{}
\end{align}
which can be made arbitrarily small.
Hence, 
\begin{align}
  \p(\sup_{t \in [0,1]}\vert \sumt z_{h,k} \ghk(t) \vert > \delta) \to 0.\label{eq:z}
\end{align}
The result follows from \eqref{eq:w},\eqref{eq:x},\eqref{eq:y}, and \eqref{eq:z}.
\end{comment}
\end{proof}

