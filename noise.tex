\chapter{Market microstructure noise and spot volatility estimation }
The need to estimate the volatility of an asset is pervasive in finance. Volatility is the key component in portfolio selection, option pricing, and risk management.   Without further restriction on the price process the estimation of the volatility coefficient would be all but impossible. Thankfully financially theory provides some guidance in this regard. At relatively long observation intervals such as a week or a month, it is generally agreed that observed prices may be thought of as  discrete realizations from a  semimartigales. The theory of semimartingales provides a complete answer on what form we should expect the volatility coefficient of the price process to take.  Furthermore, the theory tells us that the usual realized variance estimator converges to the quadratic variation process  as the observation interval shrinks to zero \cite[Theorem 23]{Protter2004}. 

Unfortunately, this general concensus on the semimartingale status of observed prices breaks down when we consider prices sampled at  higher frequencies. The problem is that, at high frequencies, it is hard to justify the assumption that the efficient price process, which may be a semimartingale, is directly observable. Instead, what is observable is part efficient price and part noise  resulting from the established processes in the market.    The noisy component of observed prices has several sources; these include the so-called bid-ask bound, the release of asynchronous information, and  rounding error resulting from discrete prices etc. We elaborate on this sources below. The term \emph{market microstructure noise} was coined by \cite{Garman1976} to describe this type of price contamination. In short, due to the presence of market microstructure noice in  high frequency prices, it may no longer be justifiable to assume that observed prices are the discrete realizations of a semimartigale.



There is a large literature dealing with nonparametric estimation of the volatility process in the presence of market microstructure noise. Here too the focus of these efforts has been on obtaining estimates for the integrated volatility. Some of the proposed approaches such as the two-time scale estimator of \cite{Zhang2005} and the pre-averging estimator of \cite{Podolskij2007} have been extended to cover local spot volatility estimation; see for example \cite{Zu2014}. On the other hand,  the study of market-microstructure-robust global volatility estimators has thus far been rather scant. A notable contribution, is the wavelet-based estimator proposed by \cite{Hoffmann2012}, which works essentially by estimating the wavelet coefficient using pre-averaged market data. In this chapter, we propose an alternative estimator which combines the Gabor frame estimator from the previous chapter with the multi-time scale procedure popularized by \cite{Zhang2005}. We believe that frame-based estimators are very well-suited for dealing with high-frequency data. In a high-frequency setting, we may expect basis coefficient estimation error to be amplified by the presence of market microstructure noise in price data. As was explained in Chapter 1 (see subsection \ref{sub:why}) , the redundancy inherent in frames can be leveraged to efficiently reduce the effects of market microstructure noise on overall volatility estimates. 

The rest of this chapter is organized as follows: In Section \ref{sec:model2} we specify the price process and the structure of the noise process. In Section \ref{sec:estimator2} we give a specification of the noise robust spot volatility estimator. In Section \ref{sec:simulation2} we conduct a simulation exercise to verify the validity of the proposed estimator. In section \ref{sec:empirics2} we use the estimator to gain insight into the diurnal pattern of volatility in the bond market; Section \ref{sec:conclusion2} concludes the paper. A rigorous proof of the consistency of the estimator will be given in the Appendix. 
\section{Model}\label{sec:model2}
We consider the problem of making inference on the spot volatility of a  security price process using market data sampled discretely in the presence of market microstructure noise. The presence of the noise component implies that  the  price process of interest is unobservable directly; instead what we have  are discrete transaction or bid and ask price data  with market microstructure noise contamination. The usual way this market setup is modeled in the literature is via the Additive Market Microstructure (AMN) model, which as the name implies, states that prices are affected additively by microstructure noise. That is, for $i = 0,1,\cdots,n$  and $0 = t_0\le t_i \le t_n = T$,  the $i$-th observed price at time $t_i$  may be modeled as:
\begin{align}
  Y_{t_i} = X_{t_i} + \varepsilon_{t_i},
  \label{}
\end{align}
 where $\varepsilon_{t_i}$ is the $i$-th coordinate in an  \iid sequence of market microstructure noise. The noise component is assumed to be independent of the efficient price process. The unobserved efficient price of interest $X$  is the unique solution to the following stochastic differential equation:
\begin{align}
  \D X_t = \mu(t,X_t) \D t + \sigma(t) \D W_t, 
  \label{eq:sde}
\end{align}
where $W_t$ is a Brownian motion, $\sigma $ is strictly positive, and both $\mu$ and $\sigma$ are  continuous and bounded from above. The functions $\mu$ and $\sigma$ are referred to respectively as the drift and volatility functions.  This setup is similar to the  microstructure environment considered by \cite{Zhang2005} in their estimation of the integrated volatility. 
\subsection{Market microstructure noise}  
According to \cite{Garman1976}, market microstructure noise arises from the moment-to-moment aggregate exchange behavior. Some major sources of market microstructure noise include:
    \begin{enumerate}
      \item \emph{The bid-ask spread}. The price at which an investor can buy an asset, at any fixed point in time,   is almost always greater than the price at which he may sell the asset. The \emph{real} or efficient price of the asset is somewhere in between (in some cases, it could be outside the range if there is private information not available to the other participants in the market)
      \item \emph{The price impact of trade}. The idea is that each transaction releases information about the underlying asset. For instance, a buyer-initiated transaction tells the market that the asset is more valuable than its current price to somebody. Now, a really big buyer-initiated transaction tells the market that someone with a lot of money and, with no doubt a sophisticated knowledge of the market, thinks the asset is more valuable in the future than its current price. This type of information release can lead to a domino effect where the market goes through several rounds overbidding the price of the asset even though the fundamentals of the asset may not have changed. A pioneering work in the theoretical study of the price impact of trades is the paper by \cite{Roll1984};  \cite{Hasbrouck1991} provides an empirically-oriented treatment.
      \item \emph{Price round-off} Suppose the market valuation of IBM stock is CHF 19.95666, but because market prices are quoted up to a certain decimal place, the security may be exchanged  at say CHF 19.95. Economically, this seems like a small matter, but implementation-wise  this  a problem for any statistical procedure relying on the assumption that prices satisfy some form of \emph{recurrence or mixing} property. This is because with prices rounded at 2 decimal places, it is no longer the case that any possible value in the continuous range of the asset price will eventually show up in the data given enough time. Thus the vast majority of price information  will in fact \emph{never} be reported.
      \item\emph{Human error} This is especially a problem for prices resulting from trading pits. The chaos of the trading pit almost surely guarantees the occurence of data entry errors throughout the trading day.
\end{enumerate}
\section{Noise-robust estimator} \label{sec:estimator2}
We start by highlighting the difficulty with which we are faced when trying to estimate the volatility function in the presence of market microstructure noise. The main difficulty in Chapter 1 is that we are attempting to estimate an unobservable or latent volatility function on the basis of discretely observable price process; in the present setting, not even the price process $X$ is observable. What we have at our disposal is data $Y$, which is part efficient price $X$ and part noise $\varepsilon$. We have no idea how much of the observed price $Y$ is noise and how much is efficient price. It is thus safe to assume that we need to do something different in order to take care of the contamination. Motivated by the ideas proposed by \cite{Zhang2005} in the integrated volatility case, we propose to divide the sample $n$ into $R_n$ subsamples containing $m_n$ data points so that $R_n m_n$ is approximately $n$ or $R_n m_n \sim n$. For this approach to work, it is required that both $R_n$ and $m_n$ be large and tend asymptotically to infinity. A consequence of this assumption is that the current estimator does not merely collapse to the estimator proposed in Chapter 1 in the noiseless case. Once the subsamples are in place, we may compute coefficients of the Gabor frame expansion using each subsample (one set of basis coefficients for each subsample); next, we  take an average of the basis coefficient estimates over the $R_n$ subsamples; finally, we use  the coefficient estimates based on  the entire sample $n$ to bias-correct the average  coefficient estimates obtained in the previous step. In notation, we propose the following estimator of the spot volatility function in an environment with market microstructure noise:
\begin{align}
  &\svnb(t) = \sum_{(h,k) \in \Theta_n} \cnhk^b\;g_{h,k}(t),\qquad \forall t \in [0,T], \label{eq:noiseest}
\end{align}
where
\begin{align}
  &\cnhk^b = \cnhk^R - (m_n/n)\cnhk \\
  &\cnhk^R = (1/R_n)\sum_{i =0}^{n -R_n} \btghki (Y_{t_{i+R_n}} - Y_{t_i})^2 \label{eq:subset} \\
  &\cnhk = \sum_{i =0}^{n-1} \btghki (Y_{t_{i+1}} - Y_{t_i})^2.
  \label{}
\end{align}
Here, as in Chapter 1, $\{g_{h,k}, \tghk\}_{h,k \in \ints}$ denotes a pair of dual Gabor frames generated under the assumptions of Lemma \eqref{le:gabor}; $\Theta_n := \{(h,k) \in \ints^2 : |h| \le H_n \text{ and }|k| \le K_0\}$;  $H_n$ is an increasing sequence in $n$ representing the number of frame elements used along the frequency axis for each time axis shift; whereas  $K_0= \lceil (T + |s| + |r|)/b \rceil$  is a constant representing the number of frame elements used along the time axis for each frequency domain shift. 

We would like to mention that the similarities between \svn from Chapter 1 and \svnb are at best superficial. Note that \svn is constructed using the actual efficient price process $X$, whereas in the present context we have to make do with corrupted market data $Y$. Also, note that the coefficients \cnhk play a secondary role here; they merely serve as a devise for bias correction. On the other hand there are strong similarities between the two time scale estimators of \cite{Zhang2005} and \cite{Zu2014}. To see this note that we may express \eqref{eq:subset} as follows:
\begin{align}
  \cnhk^R = (1/R_n) \sum_{i=1}^{R_n -1} \sum_{j =1}^{m_n} \overline{\tghk(t_{i+ (j-1)R_n\Delta_n})}(Y_{i+ jR_n\Delta_n}-Y_{i+ (j-1)R_n\Delta_n})^2. \notag 
  \label{}
\end{align}
Now it is clear that $\cnhk^R$ is the average coefficient taken over the $R_n$  coefficients estimates obtained using the $R_n$ subsamples. In the sequel we show using a simulation study the validity of the proposed estimator. We also apply the estimator compute and study diurnal patterns in intra day volatility in the bond market.  A rigorous proof of the consistency of the estimator will be given in the Appendix.
\section{Simulation study} \label{sec:simulation2}
\section{Diurnal pattern in the bond market revisited}\label{sec:empirics2}
\section{Conclusion}\label{sec:conclusion2}
