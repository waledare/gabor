\begin{appendices}
  \section{Proofs}\label{ap:proof}
%  \begin{comment}
  We now give the proof of Lemma \eqref{lem:modtg}. \\ \\
  \begin{proof}
   $G$ is bounded away from zero. To see this, note that since $g$ has support in  $[r,s]$, the series on the left hand side of \eqref{eq:capg} has finitely many terms for each $t$.  In addition, it is straight forward to verify that $G(t) = G(t+b)$ for all $t$; so,  $G$ is periodic with period $b$. It is also clear that because $g$ is continuous, so is $G$. It follows that $G$ attains its min and max on any interval of length $b$. Let $I_b$ denote the interval $[(s+r -b)/2, (s+r +b)/2]$, then
  \begin{align}
    \min_{t\in\real} G(t) &=  \min_{t \in I_b} G(t) \notag\\
    & \ge  a^{-1}\min_{t \in I_b}\vert g(t)\vert^2.\notag
    \label{}
  \end{align}  
  Because $g$ is continuous and $g$ doesn't vanish in $(r,s)$, we conclude that $G_*:=\min_{t \in \real}G(t) > 0$. It is also straight forward that $G^* := \max_{t \in \real}G(t) < \infty$. Now, let $t, t' \in \real$, $ t > t'$,  such that $\vert t - t' \vert \le \delta$, then 
  \begin{align}
     \vert \tg(t) - \tg(t')\vert & = \vert (G(t)G(t'))^{-1}(g(t)G(t') - g(t')G(t))\vert  \notag\\
    & \le (G_*^{-2})\{\vert g(t)\vert \vert G(t) - G(t') \vert + \vert G(t)\vert \vert g(t) - g(t')\vert\}. 
    \label{eq:difftg}
  \end{align}
For a real number  $x$, denote $\lfloor x \rfloor$  the largest integer less than or equal to $x$ and $\lceil x \rceil$  the smallest integer that is greater  than or equal to $x$. Now,
Let $A$ denote the set of integers $i$ such that $r < t - ib < s$. By definition of $g$, $g(t - j b) = 0$, whenever $j \not\in A$. Since $b > 0$, $A$ contains at most $\lceil( 1 + \vert s \vert + \vert r \vert)/b \rceil$ number of elements.  Let $\tau := \min \{t - ib: i \in A\}$, i.e. $\tau$ is the smallest $t - ib$ such that $i \in A$. Because $A$ contains at most a finite number of elements, there exists an integer $k$ such that $\tau = t - kb$. Set $\tau' := t'-kb$.   
  
   It is straight forward to verify that  $\vert \tau - \tau' \vert \le \delta$ and 
\begin{align}
a  \vert G(t) - G(t') \vert & \le  \sum_{j = 0}^{\lceil ( 1 + \vert s \vert  +\vert r \vert )/b \rceil} \vert g(\tau + j b)^2 - g(\tau' + j b)^2\vert\notag\\
  & \le \sum_{j = 0}^{\lceil (1 + \vert s \vert  +\vert r\vert )/b \rceil} \vert g(\tau + j b) - g(\tau' + j b) \vert \vert g(\tau + j b) + g(\tau' + j b)\vert\notag\\
  & \le 2 \lceil (1 + \vert s \vert  + \vert r \vert)/b\rceil g^*\modc{g}{\delta},
  \label{}
\end{align}
where $g^* := \max_{t \in \real} \vert g(t)\vert $.
Returning to \eqref{eq:difftg}, we see that
\begin{align}
  \vert \tg(t) - \tg(t')\vert \le C_{\tg} \modc{g}{\delta},\notag
  \label{}
\end{align}
where $C_{\tg} = G_*^2(2a (\lceil ( 1 + \vert s \vert  +\vert r\vert)/b\rceil)(g^*)^2 + G^*)$. Now let \hkints, then
\begin{align}
  \vert \tghk(t) - \tghk(t')\vert & = \vert \mathrm{e}^{2 \pi \i h a t} (\tg(t - kb) -\tg(t'-kb)) \vert\notag\\
  &\le \vert \tg(t - kb) -\tg(t'-kb) \vert \le C_{\tg} \modc{g}{\delta}.
  \label{}
\end{align}
The last inequality follows because translating a function leaves its modulus of continuity unchanged.\\
\end{proof}
%\end{comment}
The following is a simple corollary to \levy's modulus of continuity theorem. 

\begin{lem} \label{lem:mylevy}
  Suppose $\int^1_0\sigma^2(s) \D s < \infty$, almost surely,   where $\sigma$ is adapted, strictly positive,  and continuous. If $W$ is an $\mcal{F}_t$-Brownian motion, then for sufficiently large $n$, it is almost surely the case that  
  \begin{align}
    \sup_{0 \le i \le n -1} \left \vert \int_{t_i}^{t_{i+1}} \sigma(s)  \D W_s   \right \vert \le C (n^{-1}  \log (n))^{1/2}, \notag
    \label{}
  \end{align}
  where $C$ is a finite-valued random variable.
\end{lem}
\begin{proof} Let $T_t := \inf\{s > 0: \int_0^s \sigma^2(u) \D u > t\}$ and  $\mcal{G}_t := \mcal{F}_{T_t}$. Then,  by Theorem 42 of \cite{Protter2004}, it is almost surely  the case that 
  \begin{align}
    \int_{0}^{t} \sigma(s)  \D W_s = B_{T_t},   \notag
    \label{}
  \end{align}
  where $B$ is a $\mcal{G}_t$-Brownian motion.
   Applying \levy's modulus of continuity result \citep[Theorem 10.32]{Rogers1994a} to $ B_{T_t} $, we have, almost surely,  for sufficiently large $n$
  \begin{align}
    \sup_{ 0 \le i \le n-1}\left\vert \int_{t_i}^{t_{i +  1}} \sigma(s)  \D W_s \right \vert  (2 \delta_i \log(1/\delta_i))^{-1/2} \le 1, 
    \label{}
  \end{align}
  where $\delta_i := T_{t_{i+1}} - T_{t_i} = \int^{t_{i + 1}}_{t_i} \sigma^2(s) \D s$. Because $\sigma$ is continuous and strictly positive on \domain, there are finite-valued random variables $c^* \ge c_* > 0$, such that $c^* \ge \sigma^2(s) \ge c_*$, $s \in \domain$. Hence, 
  \begin{align}
    (2 \delta_i \log(1/\delta_i))^{1/2} & \le  (2 c^*n^{-1} \log(n/c_*))^{1/2}\notag\\
    &= (2 c^*n^{-1} \log(n) - 2 c^*n^{-1} \log(c_*)  )^{1/2}\notag\\
    &\le (2 c^*n^{-1} \log(n))^{1/2}\notag.
    \label{}
  \end{align}
\end{proof}
\noindent Now we prove Proposition \eqref{pr:consistency}.\\\\
\begin{proof}
Without loss of generality let $X_0 = 0$, and  take $\alpha  \in (0, 1]$ and $c > 0$ as given. We begin with $B^2_n(\alpha, c)$, the integrated square bias component of $R_n(\alpha,c)$, which is defined as:
\begin{align}
  B^2_n(\alpha, c) = \int_0^1 \e[ (\svn(t) - \sigma^2(t))\indvol]^2 \D t. \label{eq:sqbapp}
\end{align}
We make the following notational simplification:
\begin{align}
  \eind [ X] := \e[ X \indvol], \notag
  \label{}
\end{align}
for all random variables $X$. We proceed by first obtaining an  upper bound  for the integrand in \eqref{eq:sqbapp}, i.e. the square bias at each fixed point $t$. To that end, let $t \in \domain$, and note that 
\begin{align}
  \eind[ \svn(t) - \sigma^2(t)] & = \sum_{(h,k) \in \Theta_n} \eind[\cnhk\ - \chk]\;g_{h,k}(t)\notag\\
  & \quad -\sum_{(h,k) \not\in \Theta_n} \eind[\chk]\;g_{h,k}(t),\label{eq:rone} 
\end{align}
  where 
\begin{align}  
  \cnhk &= \sum_{i =0}^{n-1} \btghki (X_{t_{i+1}} - X_{t_i})^2 \text{ and }\notag\\  
  \chk &= \int^1_0 \btghks \sigma^2(s) \D s. \notag
\end{align}
We tackle the summands in \eqref{eq:rone} in turn starting with the first one. But first let 
\begin{align}
  M_i := \int_{t_i}^{t_{i+1}} b(s) \D s, \quad \text{and} \quad  S_i := \int_{t_i}^{t_{i+1}} \sigma(s) \D W_s, \notag
\end{align}
and note that since $X_{t_{i+1}} - X_{t_i} = M_i + S_i$, it follows that
\begin{align}
  \eind[(X_{t_{i+1}} - X_{t_i})^2] &= \eind[M_i^2]  
  + 2\eind[M_iS_i] \notag\\ & \quad+ \eind[S_i^2].\notag 
\end{align}
So, \eqref{eq:rone} may be written as 
\begin{align}
  &\eind[\svn(t) - \sv(t)] = B_{1,n}(t) + B_{2,n}(t) + B_{3,n}(t) + B_{4,n}(t), \notag
\end{align}
where
\begin{align}
  &B_{1,n}(t) :=  \sumt g_{h,k}(t) \left(\eind\left[ \sum_{i=0}^{n-1} \btghki S_i^2  - \chk\right]\right), \notag\\
  &B_{2,n}(t) := 2 \sumt g_{h,k}(t)\left(\sum_{i=0}^{n-1} \btghki \eind\left[S_i M_i\right] \right), \notag\\
  &B_{3,n}(t) := \sumt g_{h,k}(t)\left( \sum_{i=0}^{n-1} \btghki \eind\left[M_i^2\right] \right), \notag\\
  &B_{4,n}(t) := - \sum_{\substack{(h,k) \not\in \Theta_n }}g_{h,k}(t)\eind[\chk]. 
  \label{}
\end{align}
We will estimate the summands starting with $B_{4,n}(t)$. Note the following:
\begin{align}
  \sum_{\substack{(h,k) \not\in \Theta_n }}g_{h,k}(t)\eind[\chk] &=  \sum_{\substack{(h,k) \not\in \Theta_n }}g_{h,k}(t)\e[\inner{\sigma^2}{ \tghk}\indvol] \notag\\ 
&\le  \e\left\vert\sum_{\substack{(h,k) \not\in \Theta_n }}g_{h,k}(t)\inner{\sigma^2 \indvol}{ \tghk}\right\vert \notag\\
&\le c\e[\modc{\tghk}{1/H_n}\log H_n] \notag \\
&\quad + c\e[\modc{\sigma^2\indvol}{1/H_n} \log H_n]\notag,
\end{align}
where the last  line follows from Lemma \ref{th:fourone} after swapping the label $g$ with $\tg$ in Lemma \ref{le:gabor}. But since  $\sigma^2 \indvol$ is bounded by $c$ if it is in the \holder ball \hball{\alpha}{c}  and $0$ otherwise, it follows that $\modc{\sigma^2\indvol}{1/H_n} \le c H_n^{-\alpha}$. Furthermore, by Lemma \eqref{lem:modtg} and the Lipschitz continuity of $g$ we have $\modc{\tghk}{1/H_n} \le c H^{-1}_n$.  So,  
\begin{align}
 B_{4,n}(t) =  O( H_n^{-\alpha}\log H_n).
  \label{eq:B4}
\end{align}
Note the generic use of the constant  $c$. In the sequel, we will use $c$ to denote the amalgamation of various constants resulting from multiple steps; this should be harmless since  constants are not asymptotically relevant. 

We now obtain an estimate for  $B_{3,n}(t)$. Note the following:
\begin{align}
  \eind[ M_i^2] & = \e\left[\left(\int_{t_i}^{t_{i+1}} b(s) \D s\right)^2\indvol\right] \notag\\
  & = \e\left[\left(\int_{t_i}^{t_{i+1}} b(s)\indvol \D s\right)^2\right] \notag\\
& \le \e\left[\left(\int_{t_i}^{t_{i+1}}\vert b(s)\indvol\vert \D s\right)^2\right] \notag
  \label{}
\end{align}
Note that $\vert b(s)\indvol\vert$ is either 0 or less than $c$, so   
\begin{align}
 \eind[ M_i^2] \le c \Delta_n^2. 
  \label{eq:msq}
\end{align}
 Now since  $g_{h,k}$ and \tghk  are bounded independently of $h$ and $k$, and $n\Delta_n = 1$, we have
\begin{align}
  B_{3,n}(t) = O( H_n \Delta_n).
  \label{}
\end{align}
We obtain an estimate for $B_{2,n}(t)$ next, but first let 
\begin{align}
&T_{\sigma^2}(c)   := \inf\{t \in  (0,1]: \sigma^2(t) > c\}, \\ 
  &T_{b}(c)  := \inf\{t \in (0,1]: b(t) > c\}. \notag
  \label{}
\end{align}
So, $T_{\sigma^2}(c)$ and $T_{b}(c)$ are the hitting times of the open set $(c, \infty)$ by $b$ and $\sigma^2$, respectively; and they record the instant just before these coefficients exceed $c$. Because both processes are adapted and at least \cadlag, both hitting times  are stopping times. Now, set 
\begin{align}
T_i(c)  :=  T_{\sigma^2}(c) \wedge  T_{b}(c) \wedge  t_i, \qquad i= 1, \cdots, n. \notag
\end{align}
Because  $T_{\sigma^2}(c)$ and   $T_{b}(c)$ are stopping times, so are the  $T_i(c)$'s. The important thing to note is that at all times before  $T_i(c)$, both $b$ and $\sigma^2$ are bounded by $c$.  

Returning to $B_{2,n}(t)$ note that $\indvol^2 = \indvol$ so that
$\eind[M_iS_i] = \e[(M_i \indvol) (S_i \indvol)]$. By the Cauchy-Schwarz inequality, 
\begin{align}
  \e[(M_i \indvol) (S_i \indvol)]\le \e[(M_i \indvol)^2]^{1/2}\e[ (S_i \indvol)^2]^{1/2}\label{eq:rfour}
\end{align}
Now, by repeating the same steps as in the case of  $B_{3,n}(t)$ above, we  may conclude that 
\begin{align} 
  \e[(M_i \indvol)^2]^{1/2} \le  c \Delta_n \label{eq:rtwo}.
\end{align}
Now consider the following: if $\omega \in \eball{\alpha}{c}$ then $\sigma^2(\omega) \in \hball{\alpha}{c}$ and $\Vert \sigma^2(\omega) \Vert_\infty \le \Vert \sigma^2(\omega)\Vert_\alpha \le c$ so  that $T_{\sigma^2(\omega)}(c) = 1$. Similarly $T_{b(\omega)}(c) = 1$ so that $T_i(c) = t_i$.  Hence, 
\begin{align}
  \e[ (S_i \indvol)^2] &= \e\left[\left(\int_{t_i}^{t_{i+1}} \sigma(s) \D W_s\right)^2\indvol\right]\notag\\
&= \e\left[\left(\int_{t_i}^{T_{i+1}(c)} \sigma(s) \D W_s\right)^2\indvol\right]\notag\\
&\le \e\left[\left(\int_{t_i}^{T_{i+1}(c)} \sigma(s) \D W_s\right)^2\right],
\end{align}
Note the role played by the $T_i(c)$'s; they serve to eliminate the factor $\indvol$ from the computations.  An application of the Burkholder-Davis-Gundy (BDG) inequality now has the effect of eliminating the Wiener process $W$. So,
\begin{align}
  \notag\e[ (S_i \indvol)^2]^{1/2} &\le c\e\left[\left(\int_{t_i}^{T_{i+1}(c)} \sigma^2(s) \D s\right)\right]^{1/2}\notag \\
  &\le (c \Delta_n)^{1/2}.\label{eq:rthree}
\end{align}
 Now, substituting \eqref{eq:rtwo} and \eqref{eq:rthree} into \eqref{eq:rfour} yields the estimate
\begin{align}
  \eind[M_i S_i] \le (c\Delta_n)^{3/2}.\label{eq:ms}
\end{align}
Since  $g_{h,k}$ and \tghk  are bounded independently of $h$ and $k$, and $n\Delta_n = 1$, we have
\begin{align}
  B_{2,n}(t) = O( H_n \Delta_n^{1/2}).
  \label{}
\end{align}
Now we tackle the final piece $B_{1,n}(t)$. Let 
\begin{align}
  A  := \eind\left[\sum_{i=0}^{n-1}\btghki S_i^2 -\int_{0}^{1} \sigma^2(s) \btghks \D s\right].
\end{align}
We will first obtain an upper bound for $A$; we proceed by adding and subtracting $\eind [\sumin\int_{t_i}^{t_{i+1}}\btghki \sigma^2(s)  \D s]$  from  $A$ to yield: 
\begin{align}
  A & = \eind\left[\sumin\btghki \left(S_i^2 -\int_{t_i}^{t_{i+1}} \sigma^2(s) \D s\right)\right ]\notag \\
  &\quad +  \eind\left[\sumin\left(\int_{t_i}^{t_{i+1}} \sigma^2(s) \{\btghki - \btghks\} \D s\right) \right]\notag\\
  &=:A_{1} + A_{2} \notag.
  \label{}
\end{align}
We obtain estimates in turn for the summands. By linearity of expectation 
\begin{align}
  A_{2} &= \sumin \int_{t_i}^{t_{i+1}} \e[\sigma^2(s) \indvol] \{\btghki - \btghks\} \D s\notag\\
  &\le c\modc{\tghk}{\Delta_n},\notag
  \label{}
\end{align}
where $\modc{\tghk}{\Delta_n}$ is the modulus of continuity of $\tghk$ on an interval of length $\Delta_n$. By  Lemma \eqref{lem:modtg} and the Lipschitz continuity of $g$ we have, 
\begin{align}
  A_{2} \le c\modc{g}{\Delta_n} \le c \Delta_n.\notag \\
  \label{}
\end{align}
Now, we obtain an estimate for $A_{1}$. First, let $D_i: \Omega\times[0,1] \to \real$ for $i = 0, \cdots, n-1$ be defined as follows:  
\begin{align}
&D_{i}(t) := \btghki\left(\int_{t_i}^{t} \sigma(u)\D W_u\right)I_{(t_i, t_{i+1}]}(t).
  \label{eq:Di}\\
&D_0(0) := 0.
\end{align}
So, $D_i(t)$ is 0 on $[0,1]$ except when $t$ is in $(t_i, t_{i+1}]$. Now, using the integration by parts formula for semimartingales, we may write
\begin{align}
  S^2_i - \int^{t_{i+ 1}}_{t_i} \sigma^2(s) d s = 2\int_{t_i}^{t_{i+1}}\left(\int_{t_i}^{s} \sigma(u)\D W_u\right)\sigma(s) \D W_s\notag
  \label{}
\end{align}
so that 
\begin{align}
  A_1 &= 2\,\eind\left[\sumin\int_{t_i}^{t_{i+1}}\btghki\left(\int_{t_i}^{s} \sigma(u)\D W_u\right)\sigma(s) \D W_s\right ]\notag \\
  &=2\,\eind\left[\int_0^1\sumin D_i(s) \sigma(s) \D W_s \right]\notag\\
  &=2\,\e\left[\left(\int_0^1\sumin D_i(s) \sigma(s) \D W_s\right)\indvol \right]\notag.
  \label{}
\end{align}
Using the same stopping time argument as above, we may replace the upper limit of integration with $T_n(c)$ so that 
\begin{align}
  A_1 & \le 2\,\e\left[\left(\int_0^{T_n(c)}\sumin D_i(s) \sigma(s) \D W_s\right)\indvol \right]\notag\\ 
  & \le 2\,\e\left[\left\vert\int_0^{T_n(c)}\sumin D_i(s) \sigma(s) \D W_s\right\vert \right].\notag 
  \label{}
\end{align}
Now using the fact that $\int_0^{T_n(c)}\sumin D_i(s) \sigma(s) \D W_s$ is a  martingale, we may make another appeal to the BDG inequality to yield:
\begin{align}
  A_1 & \le c\e\left[\left\vert\int_0^{T_n(c)}\left(\sumin D_i(s) \sigma(s)\right)^2 \D s\right\vert^{1/2} \right]\notag \\
  & \le c\e\left[\left\vert\int_0^{T_n(c)}\sumin \{D_i(s) \sigma(s)\}^2 \D s\right\vert^{1/2} \right]\notag,
  \notag
  \label{}
\end{align}
where the last line follows because $D_i (s) D_j(s) = 0$ whenever $i \not= j$. Now if  we define $D^*_i := \sup_{t_i <s \le T_n(c)} D_i(s)$, and use the fact that $\sigma$ is less than  $c$ before $T_n(c)$ then
\begin{align}
  A_1 & \le c\e\left[\left\vert\sumin \Delta_n(D_i^*)^2 \right\vert^{1/2} \right] \notag\\
  & \le c\e\left[\left\vert\sumin \Delta_n(D_i^*)^2 \right\vert \right] \notag\\
  & \le c\Delta_n\sumin\e[(D_i^*)^2]   
  \label{eq:ra1}
\end{align}
where $c$ is a generic constant representing the bound on $\sigma^2$ and the BDG constant.  Note  from the definition of  $D_i$ \eqref{eq:Di} that it  is itself a martingale, so we may bound  $D^*_i$  with yet another application of the BDG inequality. That is
\begin{align}
  D^*_i &\le \left(\int^{T_{i+1}(c)}_{t_i} \sigma^2(s) \D s\right)^{1/2} \notag\\
  &\le c \Delta_n^{1/2}. 
  \label{}
\end{align}
Plugging the above into the estimate in \eqref{eq:ra1} yields: $A_1 \le c \Delta_n $. Combining  the estimates for $A_1$ and $A_2$,  it may be seen that
\begin{align}
  B_{1,n}(t) = O(H_n\Delta_n ).
  \label{}
\end{align}
Collecting the estimates for $B_{j,n}(t)$ for $j =1,\cdots,4$, it is easily seen that $\e[\svn(t) - \sigma^2(t)] = O(H_n \Delta_n^{1/2}  +H_n^{-\alpha}\log H_n)$ for all $t \in \domain$. So that 
\begin{align}
  B^2(\alpha,c) =  O(H_n^2 \Delta_n  + H_n^{-2\alpha}\log^2 H_n).\notag
  \label{}
\end{align}
Next, we obtain a  bound for the variance term $V_n(\alpha, c)$. Recall that
\begin{align}
V_n(\alpha,c)  & =  \int_0^1 \eind[\{\svn(t) - \eind[\svn(t)]\}^2]  \D t.
\notag
\end{align}
So that by the definition of the estimator, we may write
\begin{align}
 V_n(\alpha,c) & =   \int_0^1\eind\left[\left\{\sumt (\chk -\eind[\chk])\ghk(t)\right\}^2 \right]\D t \notag\\
& =\sum_{(h,k) \in \Theta_n}\varind[\cnhk]\left( \int_0^1 g_{h,k}^2(t) \D t\right)\notag \\
& \quad + \sum_{(h,k) \ne (h',k') \in \Theta_n} \covind[\cnhk, \hat{c}_{h',k'}]\left( \int_0^1 g_{h,k}(t) g_{h',k'}(t)\D t\right) \notag\\
& =: V_1 + V_2,
\end{align}
 We will estimate these quantities in turn starting with $V_1$, but first let  
\begin{align}
  &Y_i := \left(\int^{t_{i+1}}_{t_i} \sigma(s) \D W_s\right)^2 ,\notag\\
  &Z_i := \left(\int^{t_{i+1}}_{t_i} b(s) \D s\right)^2 + 2\left(\int^{t_{i+1}}_{t_i} b(s) \D s\right)\left(\int^{t_{i+1}}_{t_i} \sigma(s) \D W_s\right), \notag\\ 
  &\beta_{1,i} := \sum_{(h,k) \in \Theta_n} g^2_{h,k}(t_i)\left( \int_0^1 g_{h,k}^2(t) \D t\right),\notag
  \label{}
\end{align}
for $i = 0,\cdots, n-1$. Now note that 
\begin{align}
  \cnhk = \sumin \ghk(t_i) (X_{t_{i=1}} - X_{t_i})^2 = \sum_{i = 0}^{n-1} g_{h,k}(t_i) (Y_i + Z_i),\notag
  \label{}
\end{align}
and  since  increments of the Brownian motion are independent, we have
\begin{align}
  V_{1} = \sum_{i = 0}^{n-1} \beta_{1,i} (\varind[Y_i] + \varind[Z_i] + 2 \covind[Y_i,Z_i]). \notag
  \label{}
\end{align}
We will estimate the first two moments of $Y_i$ and $Z_i$ in turn. Note that  
\begin{align}
  \eind[Y_i] &= \e[Y_i \indvol] = \e\left[ \left(\int^{t_{i+1}}_{t_i} \sigma(s) \D W_s\right)^2\indvol\right]\notag\\
&=\e\left[ \left(\int^{T_i(c)}_{t_i} \sigma(s) \D W_s\right)^2\indvol\right] \notag\\
&\le\e\left[ \left(\int^{T_i(c)}_{t_i} \sigma(s) \D W_s\right)^2\right]\notag\\
&\le c \e\left[ \left(\int^{T_i(c)}_{t_i} \sigma^2(s) \D s\right)\right]\notag\\
&\le c\Delta_n.
  \label{}
\end{align}
where the fourth line results from an application of the BDG inequality.  Repeating the exact same steps, it may be seen that $\eind[Y^2] \le c \Delta_n^2$. Thus,
\begin{align}
  \varind[Y_i] = \eind[Y_i^2] - \eind[Y_i]^2 \le c \Delta_n^2.
\end{align}
Next we obtain estimates for $Z_i$. From \eqref{eq:msq} and \eqref{eq:ms} we may conclude
\begin{align}
  \eind[Z_i] = \eind[M_i^2] + 2 \eind[M_i S_i] \le c \Delta_n^{3/2}. \notag
  \label{}
\end{align}
Using similar computations as above, it may be seen that $\eind[Z_i^2] \le c\Delta^3_n$ so that 
\begin{align}
  \varind[Z_i] \le c \Delta_n^3.\notag
  \label{}
\end{align}
Now by  the Cauchy-Schwarz inequality  we may write
\begin{align}
    \covind[Y_i, Z_i] \le (\varind[Z_i] \varind[Y_i])^{1/2} \le c \Delta_n^{5/2}. \notag
\end{align}
Now because \ghk is bounded, it follows that $\beta_{1,i} = O(H_n)$ so  
\begin{align}
  V_{1} = O(H_n\Delta_n)\notag
  \label{}
\end{align}
It is  straight forward to see  that $V_{2}$ may be estimated in a similar fashion. Indeed, let
\begin{align}
  \beta_{2,i} := \sum_{ (h,k) \ne (h',k') \in \Theta_n} g_{h',k'}(t_i)g_{h,k}(t_i)\left( \int_0^1 g_{h,k}(t) g_{h',k'}(t)\D t\right),\notag
  \label{}
\end{align}
then we may write
\begin{align}
  V_{2} = \sum_{i = 0}^{n-1} \beta_{2,i} (\varind[Y_i] + \varind[Z_i] + 2 \covind[Y_i,Z_i]), \notag
  \label{}
\end{align}
the computations will then proceed identically as before from this point. Again, by the boundedness of \ghk, it follows that   $ \beta_{2,i} = O(H_n^2)$ so that
\begin{align}
  V_{2} = O(H^2_n\Delta_n).\notag
  \label{}
\end{align}
Therefore,
\begin{align}
  V(\alpha,c)  = O(H^2_n\Delta_n).\notag
  \label{}
\end{align}
\end{proof}
\end{appendices}
%\chapter{Extensions} \label{ap:ext}
