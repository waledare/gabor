\section{Volatility estimation } \label{sec:estimator}
We make the following assumptions about the drift and volatility coefficients explicit.
\begin{ass}\label{as:vol}\mbox{} 
  \begin{enumerate}
    \item The volatility function is strictly positive, bounded, and  continuous. 
    \item The modulus of continuity  of the volatility coefficient,   $\omega_\sigma(\Delta_n)$, is equal to  $ o(1/\log(1/\Delta_n))$ as $n \to \infty$.  
    \item There is $0< C_T <\infty $ such that $\vert \mu(t , x) \vert \le C_T (1 + \vert x \vert)$, for all $t \in [0,T]$ and $x \in \real$.
  \end{enumerate}
\end{ass}
\noindent Since $\sigma^2$  may not necessarily be defined on the entire real line, we proceed as in \cite{GenonCatalot1992} by  constructing an extension over the real line that has support in $[0,T]$; this way we are able to apply the Hilbert space machinery in \LtwoR. We let \bsv denote the extension of $\sigma^2$, i.e.  \bsv coincides exactly with $\sigma^2$ on $[0,T]$ but vanishes outside of $[0,T]$. We summarize these notions as follows:
\begin{ass} \label{as:barvol}
  \bsv is in  $\LtwoR$,   has support in   $[0, T]$, and coincides with $\sigma^2$ on $[0,T]$.
\end{ass}
\noindent  With this substitution, we end up with a new process $\bar{X}$  coinciding with $X$ on $[0,T]$ such that 
\begin{align}
  \D\bar{X} = \mu(t, \bar{X}_t) \D t + \bar{\sigma}(t) \D W_t, \qquad \bar{X}_0 = x.\notag
  \label{}
\end{align}
Now we may avail ourselves of the Gabor frame representation on \LtwoR. Let $\{g, \tg\}$ be a dual Gabor pair constructed as in Lemma \eqref{le:gabor}, then   \bsv admits a Gabor frame expansion given by:  
\begin{align}
  &\bsv(t)  = \sum_{h,k \in \ints} \chk \;g_{h,k}(t), \text{ where } 
\\
&\chk = \langle \bsv, \tilde{g}_{h,k} \rangle.
\end{align}
Note that both $\bsv$ and $\tg$ have compact support. Indeed \bsv has support in $[0,T]$, whereas  \tg has support in $[s,r]$. So, $\chk \ne 0$ only if  the supports of \bsv and \tghk overlap.  Furthermore, we note from \eqref{eq:dualghk} that $\tilde{g}_{h,k+1}$ is simply \tghk shifted by $b$ units; so, $\chk = 0$ if $|k| > K_0$ with 
\begin{align}
  K_0:= \lceil (T + |s| + |r|)/b \rceil,
\end{align}
where $\lceil x\rceil$ is the smallest positive integer larger than $x \in \real$.  Thus \bsv admits a  representation of the form: 
\begin{align}
  &\bsv(t) =  \sum_{\substack{(h,k) \in \ints^2\\\vert k \vert \le K_0}} \chk\;g_{h,k}(t).
\end{align}
Now, suppose $n$ observations of the log price process are available, and let 
\begin{align}
  \Theta_n := \{(h,k) \in \ints^2 : |h| \le H_n \text{ and }|k| \le K_0\},\notag
  \label{}
\end{align}
where $H_n$ is an increasing sequence in $n$. 
We propose the following estimator of the volatility coefficient in $[0,T]$:
\begin{align}
  &\svn(t) = \sum_{(h,k) \in \Theta_n} \cnhk\;g_{h,k}(t), \text{ where}\\
  &\cnhk = \sum_{i =0}^{n-1} \btghki (X_{t_{i+1}} - X_{t_i})^2.
  \label{}
\end{align}
In the next section we show that the estimator converges to $\sigma^2$ on $[0,T]$ in a mean integrated square error sense.
\section{Asymptotic properties} \label{sec:deviation}
Let $R_n$ denote the average integrate square deviation of \svn from \bsv, i.e.  
\begin{align}
R_n = \e \int_\real \{\svn(t) - \bsv(t)\}^2 \lambda(t) \D t,
  \label{eq:remeainder}
\end{align}
where $\lambda$ is a positive and continuous weight function with support in $(0,T)$. The weight function allows us to emphasis different time windows when estimating the volatility. For instance, we may wish to emphasize the recent past in real-time applications. 
We show that $R_n$ tends to 0 as a function of the sample size, $n$. Note that $R_n$ is the sum of a bias and a variance component, which we write as follows:
\begin{align}
  &R_n = B_n^2 + V_n, \notag \qquad \text{ } 
\end{align}
where
\begin{align}
  &B^2_n := \int_\real (\e[\svn(t) - \bsv(t)])^2 \lambda(t) \D t \notag \\
&V_n :=  \int_\real \e[\{\svn(t) - \e[\svn(t)]\}^2] \lambda(t) \D t. \notag
  \label{}
\end{align}
\begin{prop} \label{pr:consistency}
  Let $\{g, \tg\}$ be pair of dual Gabor generators constructed as in Lemma \eqref{le:gabor}.   Suppose the conditions in  Assumptions \eqref{as:vol} and \eqref{as:barvol} hold. If  $H_n^2 \Delta_n, H_n\omega_g(\Delta_n)$, and  $\omega_{\sigma^2}(1/H_n) \log H_n\to 0$, then the mean integrated square error
  $R_n$ tends to 0 as $n$ tends to infinity, with 
  \begin{align}
    & B_n^2 = O(H_n^2\Delta_n + \{H_n\omega_g(\Delta_n)\}^2 + \{\omega_{\sigma^2}(1/H_n) \log H_n\}^2)\notag \\
    & V_n = O(H_n^2 \Delta_n),
    \label{}
  \end{align}
  where $\omega_g(\delta) := \sup \{ |g (t) - g(t')|: t, t' \in \real \text{ and } |t -t'| < \delta\}$. 
\end{prop}
\begin{proof}
  See Appendix \ref{ap:proof}.
\end{proof}
\begin{remark}\mbox{}
  \begin{enumerate}  
    \item First, the above bounds are remarkably similar to those achievable using an orthonormal basis such as wavelets \citep{GenonCatalot1992}. The variance component is slower by a factor of $H_n$. This comes about because the vectors in a frame need not be orthogonal. The bias term is slower by a logarithmic factor. Intuitively, the logarithmic term shows up because we are expanding \bsv using a frame, which may be thought of as containing some redundant term. In practical implementations, this may be a small price to pay for the added flexibility and robustness gained by using frames. 
    \item Second, this result shows that the variance component of the MISE does not depend on the smoothness properties of either $\sigma^2$ and $g$.  
  \end{enumerate}
\end{remark}

