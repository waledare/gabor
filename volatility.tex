\section{Volatility estimation: continuous prices } \label{sec:estimator}
In this section we specify a consistent estimator of  spot volatility within a framework of continuous prices. That is, we simplify the general setup of \eqref{eq:semimartingale} to:
\begin{align}
  X_t & = X_0 + \int_0^t b_s \D s + \int_0^t \sigma_s \D W_s,    \qquad t \ge 0 
  \label{eq:contsemimartingale}.
\end{align}
We further restrict the processes $b$ and $\sigma$ as follows: 
\begin{ass}\label{as:vol}\mbox{} 
  \begin{enumerate}
    \item Spot volatility $\sigma$ is a strictly positive and adapted stochastic process  with continuous paths on \domain.  
    \item The drift coefficient $b$ is adapted and \cadlag on \domain. 
  \end{enumerate}
\end{ass}
\noindent The asymptotic results we shall establish  rely heavily on the modulus of continuity of realized spot variance, so, it is not immediately clear that the continuity assumption can be further weakened.     
The right-continuity assumption (\cadlag is a French abbreviation for right continuous, with finite left limits)  on the drift ensures that the drift coefficient is pathwise bounded on \domain. Likewise, the spot variance is pathwise bounded as a result of the continuity assumption.   

Let $\{g, \tg\}$ be a pair of dual Gabor frame generators constructed as in Lemma \ref{le:gabor}, then   \sv admits a Gabor frame expansion given by:  
\begin{align}
  &\sv(t)  = \sum_{h,k \in \ints} \chk \;g_{h,k}(t), \text{ where } 
\\
&\chk = \langle \sv, \tilde{g}_{h,k} \rangle. \label{eq:chk}
\end{align}
Note that both $\sv$ and $\tg$ have compact support. Indeed \sv has support in $[0,1]$, whereas  \tg has support in $[s,r]$. So, $\chk \ne 0$ only if  the supports of \sv and \tghk overlap.  Furthermore, we note from \eqref{eq:dualghk} that $\tilde{g}_{h,k+1}$ is simply \tghk shifted by $b$ units; so, $\chk = 0$ if $|k| \ge K_0$ with 
\begin{align}
  K_0:= \lceil ( 1 + |s| + |r|)/b \rceil,
\end{align}
where $\lceil x\rceil$, $x \in \real$, is the least  integer that is greater than or equal to   $x$.  Thus \sv admits a  representation of the form: 
\begin{align}
  \sv(t) =  \sum_{\substack{(h,k) \in \ints^2\\\vert k \vert \le K_0}} \chk\;g_{h,k}(t) ,\notag
\end{align}
and  for sufficiently large positive integer $H$, 
 \begin{align}
 \sv(t) \approx \sum_{\substack{\vert h \vert \le H \\\vert k \vert \le K_0}} \chk\;g_{h,k}(t). \notag
  \notag
   \label{}
 \end{align}
 Now, suppose $n$ observations of the price process are available, and let 
\begin{align}
  \Theta_n := \{(h,k) \in \ints^2 : |h| \le H_n \text{ and }|k| \le K_0\},
  \label{eq:theta}
\end{align}
where $H_n$ is an increasing sequence in $n$.    
We propose the following estimator of the volatility coefficient:
\begin{align}
  \label{eq:contvolestimator}
  &\svnx := \sum_{(h,k) \in \Theta_n} \cnhk\;g_{h,k}(t), \qquad  t \in [0,1], \text{ where}\\
  &\cnhk := \sum_{i =0}^{n-1} \btghki (X_{t_{i+1}} - X_{t_i})^2.
\end{align}
So $\vert \Theta_n \vert $ is the number of frame elements included in the expansion. Specifically,   $\vert \Theta_n\vert = (2K_0 + 1)(2H_n+1)$; and since $K_0$ is a finite quantity, it follows that $\vert\Theta_n\vert = O(H_n)$, i.e. the number of estimated coefficients is proportional to $H_n$, and therefore, will grow with the number of observations, $n$. 
In the next section we show that the estimator converges to $\sigma^2$ on $[0,1]$ in probability.
\subsection{Asymptotic properties} \label{sec:deviation}
In this section we obtain an estimate of the rate of convergence of  the Gabor frame estimator on compact intervals of the real line. We will take \domain as the prototype for such intervals. The usual way to think about  $\sigma^2$ is as a stochastic process, but it is just as natural to think of it as a random element. A random element is an extension  of the familiar concept of a random variable to include  situations where the state space can be any metric space $E$. Because our interest is in studying the convergence of the estimator on \domain,  $E$ will be set equal to \Ltwo, the set of real-valued, square integrable function on \domain. Now, due to the path continuity assumption on $\sigma^2$, we will restrict our attention to \czero,  the set of real-valued, continuous  functions on \domain equipped with the \Ltwo norm. So, given  an outcome $\omega$ in the sample space  $\Omega$, the restriction of realized volatility $\sigma^2(\omega)$ to \domain is a  continuous function defined on \domain. Now, with regards smoothness, \state  is  a very diverse class, comprising    functions that are infinitely differentiable,  those that are nowhere differentiable, and everything in-between. For a random volatility coefficient it may very well be the case that for some outcome $\omega$, realized volatility $\sigma^2(\omega)$ is very smooth with finite derivatives of all orders, whereas for outcome $\omega'$,  $\sigma^2(\omega')$  is very rough with kinks everywhere. For instance, with probability one, the one-dimesional Brownian motion  maps outcomes $\omega$  to continuous functions in  \state, but it is  almost surely  the case that none of these functions will be  differentiable. 

Now,  a good estimator should  yield successively better approximations with increasing observation frequency, regarless of the degree of smoothness  of the realized volatility function. But in all probability, the \emph{rate} of convergence of the approximation would depend on the smoothness of the realized volatility, with faster rates achieved for smooth functions. That is, for outcomes $\omega, \omega'$ in $\Omega$, if $\sigma^2(\omega)$ is  smoother as a function of time than $\sigma^2(\omega')$ then the number of observations required to achieve a given level of accuracy when $\sigma^2(\omega)$ is realized should not exceed  the required number when $\sigma^2(\omega')$ is realized.  So, while an estimator might eventually converge regardless of the regularity of volatility, the convergence rate may be outcome- or state-dependent. 

To develop an asymptotic theory for the Gabor frame estimator which can account for state-dependency in convergence rates, we characterize the effective  state space of the volatility coefficient, viewed as a random element in \state, according to a smoothness criterion.
A simple way to achieve this characterization is via the H\"older continuity criterion. Let $0 < \alpha \le 1$, a function $f$ in \state is said to be H\"older continuous with exponent $\alpha$ if  there is a finite constant $K$ such that whenever $x$ and $y$ are distinct numbers in \domain then 
\begin{align}
  \homo{f} := \frac{\vert f(x) - f(y)\vert}{\vert x - y \vert^\alpha} \le K. 
  \label{eq:holder}
\end{align}
The set of \holder continuous functions with exponent $\alpha$ is denoted by \calpha. The \holder class with $\alpha = 1$ is the familiar class of Lipschitz continuous functions. The \holder classes admit a natural ordering relation whereby if $\alpha$ is larger than  $\beta$ then every function that is \holder-continuous with exponent $\alpha$ is also  \holder-continuous with exponent $\beta$. Note that with regards regularity (smoothness), the ordering is reversed: the smoother the function the larger the \holder exponent. Consequently, the Lipschitz class ($ \alpha = 1$) is contained in every \holder class, and it is also the class with the smoothest (most regular) functions. As mentioned earlier, Brownian paths are nowhere differentiable, but using \holder classes the regularity of Brownian paths can be further qualified; this is a consequence of  the well-known L\'evy's Modulus of Continuity Theorem \citep[Theorem I.10.2]{Williams2000}, which states that, with probability one, no Brownian path is \holder-continuous with exponent larger than 1/2 on \domain.   
Now for a fixed exponent $\alpha$, the following norm may be defined on \calpha:
\begin{align}
  \hono{f} := \sup_{t \in \domain} |f(t)| + \homo{f},\notag
  \label{}
\end{align}
where \homo{f} is defined in \eqref{eq:holder}. The norm is obviously well-defined since $f$ is a continuous function defined on a compact set. Now using \hono{\cdot}  the state space may be characterized in a such a way that functions with similar regularity propoerties can be grouped together. We accomplish this via \holder balls: A \holder ball  of radius $c > 0$ is given by:
\begin{align}
  \hball{\alpha}{c}  := \{f \in \calpha : \hono{f} \le c\}.\notag
  \label{}
\end{align}
With this device it is possible to obtain convergence rates that take into account the regularity of realized volatility. While this is already quiet satisfying,  it is also of some interest to achieve  flexibility with regards the drift coefficient. Where the drift is concerned, regularity or even continuity for that matter is irrelevant; what is key is pathwise boundedness. The natural way to achieve flexibility in this respect is to group realized drift according to membership in  balls of radius $c$ in $\Linf$, that is,   
\begin{align} 
  \uball{c} := \{f \in \Linf: \Vert f \Vert_\infty \le c \}.
\end{align}
In this way we are able to  characterize the sample space according to the regularity of realized volatility  and the boundedness of realized drift.  This leads to the consideration of the asymptotic behavior of the estimator when  $\mu \in \uball{c}$ and $\sigma^2 \in  \hball{\alpha}{c}$ for  $c, \alpha > 0$\footnote{It is not essential to use different $c$'s for $\mu$ and $\sigma^2$.}. We denote such events by $\eball{\alpha}{c}$, i.e., 
\begin{align}
  \eball{\alpha}{c}  := \{ \omega : \mu(\omega) \in \uball{c}\} \cap \{\omega: \sigma^2(\omega) \in \hball{\alpha}{c}\}.\notag
  \label{}
\end{align}
So, an outcome $\omega$  is in \eball{\alpha}{c} if realized drift is caught between $-c$ and $c$ on \domain; realized volatility $\sigma^2(\omega)$  is \holder continuous with exponent $\alpha$,  and $\Vert \sigma^2(\omega) \Vert_\alpha \le c$. Note that the implication of the last statement is that there is some $c' \le c$ such that realized volatility is cought between $0$ and $c'$.     

We are now only left with the task of making the  obvious  modification to the usual  integrated mean square error criterion :
\begin{align}
  R_n(\alpha, c) := \e[\Vert \svn - \sigma^2\Vert^2\indvol],\notag
  \label{}
\end{align} 
where  $\indvol$ is the indicator function of $\eball{\alpha}{c}$, $\Vert \cdot \Vert$ is the \Ltwo norm, and $n$ is the observation frequency. Note that if $\eball{\alpha}{c} = \Omega$,  the expression above will just be the usual integrated mean square error criterion. By restricting the  volatility and the drift according to  events \eball{\alpha}{c}, the asymptotic properties of the estimator may be studied with full flexibility. That is we may  obtain results of the form  $\limsup_{n \to \infty}\tilde{n}_{n,\alpha,c} R_n(\alpha,c) < \infty$, where the rate may vary for different values of $\alpha$ and $c$. 

Much like the usual integrated mean square error risk, $R_n(\alpha,c)$ admits a decomposition in terms of an integrated square bias component and an integrated variance component. To see this, note the following:
\begin{align}
  R_n(\alpha, c ) & = \e \int_0^1  \{(\svn(t) - \sigma^2(t))\indvol\}^2 \D t \\ 
  & = \int_0^1 \e[ \{(\svn(t) - \sigma^2(t))\indvol\}^2] \D t \label{eq:fubini}\\ 
& = \int_0^1 \e[ (\svn(t) - \sigma^2(t))\indvol]^2 \D t \notag \\
& \quad + \int_0^1 \var[\svn(t)\indvol] \D t. \label{eq:msedecomp}
\end{align}
The equality in \eqref{eq:fubini} results from an interchange of the expectectation and integration operators justified by Fubini's theorem. The decomposition in line  \eqref{eq:msedecomp} results from the usual mean square error decomposition  into a square bias and a variance component for each  $t \in \domain$. The two summands in the last line are the bias and variance components and will be denoted $B_n^2(\alpha,c)$ and $V_n(\alpha,c)$, respectively; we obtain estimates for their rates of convergence below.
\begin{prop} \label{pr:consistency}
  Let $\{g, \tg\}$ be pair of dual Gabor generators constructed as in Lemma \eqref{le:gabor}.   Suppose the conditions in  Assumption \eqref{as:vol}  hold. If $g$ is Lipschitz continuous and $H_n \uparrow \infty$ satisfies  
  \begin{align}
    H_n^2 \Delta_n  = o(1) \notag
    \label{}
  \end{align}
   then $R_n(\alpha,c)$ converges to 0, with 
  \begin{align}
    & B_n^2(\alpha,c)  = O(H_n^2\Delta_n  + H_n^{-2\alpha} \log^2 H_n)\notag \\
    & V_n(\alpha,c)  = O(H_n^2 \Delta_n),
    \label{}
  \end{align}
  where  $\Delta_n = 1/n$ is the step size, and $H_n$ is the order of magnitude of the number of estimated frame coefficients. 
\end{prop}
\proof{ See the appendix.}
\begin{remark}\mbox{}
  \begin{enumerate}  
    \item First, the above bounds are remarkably similar to those achievable using an orthonormal basis such as wavelets \citep{GenonCatalot1992}. The variance component is slower by a factor of $H_n$. This comes about because the vectors in a frame need not be orthogonal. The bias term is slower by a logarithmic factor. Intuitively, the logarithmic term shows up because we are expanding \sv using a frame, which may be thought of as containing some redundant term. Overall, the rate of convergence of the integrated mean square error differ only by logarithmic term. In practical implementations, this is an insignificant price to pay for the added flexibility and robustness gained by using frames. 
    \item Second, this result shows that the variance component of the MISE does not depend on the smoothness properties of either $\sigma^2$ and $g$.  
  \end{enumerate}
\end{remark}
\begin{prop}\label{pro:finite}
  Suppose the  price process is specified as in \eqref{eq:contsemimartingale}. Let $\{g, \tg\}$ be pair of dual Gabor generators satisfying the conditions of Lemma \eqref{le:gabor} such that $g$ is Lipschitz continuous on the unit interval. 
If $H_n \uparrow \infty$ satisfies 
  \begin{align}
    H_n (n^{-1} \log(n))^{1/2} = o(1),\notag
    \label{}
  \end{align}
  then
  \svnx, defined in \eqref{eq:contvolestimator}, converges in \Ltwo to \sv in probability.
\end{prop}
\begin{proof}
\begin{align}
  \svnx - \sigma^2(t) & = \sum_{(h,k) \in \Theta_n} (\cnhk\ - \chk)\;g_{h,k}(t)\notag\\
  & \quad -\sum_{(h,k) \not\in \Theta_n} \chk\;g_{h,k}(t),\label{eq:rone} 
\end{align}
  where 
\begin{align}  
  \cnhk &= \sum_{i =0}^{n-1} \btghki (X_{t_{i+1}} - X_{t_i})^2 \text{ and }\notag\\  
  \chk &= \int^1_0 \btghks \sigma^2(s) \D s. \notag
\end{align}
We tackle the summands in \eqref{eq:rone} in turn starting with the first one. But first let 
\begin{align}
  M_i := \int_{t_i}^{t_{i+1}} b(s) \D s, \quad \text{and} \quad  S_i := \int_{t_i}^{t_{i+1}} \sigma(s) \D W_s, \notag
\end{align}
and note that since $X_{t_{i+1}} - X_{t_i} = M_i + S_i$, it follows that
\begin{align}
  (X_{t_{i+1}} - X_{t_i})^2 &= M_i^2  
  + 2M_iS_i +   S_i^2.\notag 
\end{align}
So, \eqref{eq:rone} may be written as 
\begin{align}
  &\svnx - \sv(t) = B_{1,n}(t) + B_{2,n}(t) + B_{3,n}(t) + B_{4,n}(t), \notag
\end{align}
where
\begin{align}
  &B_{1,n}(t) :=  \sumt g_{h,k}(t) \left(\sum_{i=0}^{n-1} \btghki S_i^2  - \chk\right), \notag\\
  &B_{2,n}(t) := 2 \sumt g_{h,k}(t)\left(\sum_{i=0}^{n-1} \btghki S_i M_i \right), \notag\\
  &B_{3,n}(t) := \sumt g_{h,k}(t)\left( \sum_{i=0}^{n-1} \btghki M_i^2 \right), \notag\\
  &B_{4,n}(t) := - \sum_{\substack{(h,k) \not\in \Theta_n }}g_{h,k}(t)\chk. 
  \label{}
\end{align}
We will estimate the summands starting with $B_{4,n}(t)$. Note the following:
\begin{align}
  \sum_{\substack{(h,k) \not\in \Theta_n }}g_{h,k}(t)\chk &=  \sum_{\substack{(h,k) \not\in \Theta_n }}g_{h,k}(t)\inner{\sigma^2}{ \tghk} \notag\\ 
&\le c\modc{\tghk}{1/H_n}\log H_n  + c\modc{\sigma^2}{1/H_n} \log H_n\notag,
\end{align}
where the last  line follows from Theorem \eqref{th:fourone}. It from the \holder continuity of $\sigma$ that  that $\modc{\sigma^2}{1/H_n} \le c H_n^{-\alpha}$. Furthermore, by Lemma \eqref{lem:modtg} and the Lipschitz continuity of $g$ we have $\modc{\tghk}{1/H_n} \le c H^{-1}$.  So,  
\begin{align}
 B_{4,n}(t) =  O( H_n^{-\alpha}\log H_n).
  \label{eq:B4}
\end{align}
Note the generic use of the constant  $c$. In the sequel, we will use $c$ to denote the amalgamation of various constants resulting from multiple steps; this should be harmless since  constants are not asymptotically relevant. 

We now obtain an estimate for  $B_{3,n}(t)$. Note the following:
\begin{align}
  M_i^2 & = \left(\int_{t_i}^{t_{i+1}} b(s) \D s\right)^2 \notag\\
  & \le \left(M^* n^{-1}\right)^2 
  \label{eq:m}
\end{align}
 Now since  $g_{h,k}$ and \tghk  are bounded independently of $h$ and $k$, and $n\Delta_n = 1$, we have
\begin{align}
  B_{3,n}(t) = O( H_n \Delta_n).
  \label{}
\end{align}
  Now consider the following, 
\begin{align}
  S_i & = \int_{t_i}^{t_{i+1}} \sigma(s) \D W_s\notag \\
  & = O (\sqrt{n^{-1} \log(n)}),\label{eq:rthree}
\end{align}
for sufficiently large $n$ by \levy's modulus of continuity theorem. Hence
\begin{align}
  M_i S_i = O(\sqrt{n^{-3} \log(n)}).\label{eq:ms}
\end{align}
Since  $g_{h,k}$ and \tghk  are bounded independently of $h$ and $k$, and $n\Delta_n = 1$, we have
\begin{align}
  B_{2,n}(t) = O( H_n \sqrt{n^{-1} \log(n)}).
  \label{}
\end{align}
Now we tackle the final piece $B_{1,n}(t)$. Let 
\begin{align}
  A  := \sum_{i=0}^{n-1}\btghki S_i^2 -\int_{0}^{1} \sigma^2(s) \btghks \D s.
\end{align}
We will first obtain an upper bound for $A$; we proceed by adding and subtracting $\sumin\int_{t_i}^{t_{i+1}}\btghki \sigma^2(s)  \D s$  from  $A$ to yield: 
\begin{align}
  A & = \sumin\btghki \left(S_i^2 -\int_{t_i}^{t_{i+1}} \sigma^2(s) \D s\right)\notag \\
  &\quad +  \sumin\left(\int_{t_i}^{t_{i+1}} \sigma^2(s) \{\btghki - \btghks\} \D s\right) \notag\\
  &=:A_{1} + A_{2} \notag.
  \label{}
\end{align}
We obtain estimates in turn for the summands. By linearity of expectation 
\begin{align}
  A_{2} &= \sumin \int_{t_i}^{t_{i+1}} \sigma^2(s)  \{\btghki - \btghks\} \D s\notag\\
  &\le c\modc{\tghk}{\Delta_n},\notag
  \label{}
\end{align}
where $\modc{\tghk}{\Delta_n}$ is the modulus of continuity of $\tghk$ on an interval of length $\Delta_n$. By  Lemma \eqref{lem:modtg} and the Lipschitz continuity of $g$ we have, 
\begin{align}
  A_{2} \le c\modc{g}{\Delta_n} \le c \Delta_n.\notag 
  \label{}
\end{align}

Now, we obtain an estimate for $A_{1}$. First, let 
\begin{align}
  T_c = \inf \{ t >0 : \sigma^2(t) > c \}\notag.
  \label{}
\end{align}
By the regularity assumption on $\sigma$, $\p(T_c > t)$ can be made arbitrarily small by taking $c$ sufficiently large.  Hence, by the usual stopping time argument, it may be assumed that $\sigma^2(t)$ is bounded above by $c$ almost surely.  Now, let $D_i: \Omega\times[0,1] \to \real$ for $i = 0, \cdots, n-1$ be defined as follows:  
\begin{align}
&D_{i}(t) := \btghki\left(\int_{t_i}^{t} \sigma(u)\D W_u\right)\mathbbm{1}_{(t_i, t_{i+1}]}(t).
  \label{eq:Di}\\
&D_0(0) := 0.
\end{align}
So, $D_i(t)$ is 0 on $[0,1]$ except when $t$ is in $(t_i, t_{i+1}]$.


Now, using the integration by parts formula for semimartingales, we may write
\begin{align}
  S^2_i - \int^{t_{i+ 1}}_{t_i} \sigma^2(s) \D W_s = 2\int_{t_i}^{t_{i+1}}\left(\int_{t_i}^{s} \sigma(u)\D W_u\right)\sigma(s) \D W_s\notag
  \label{}
\end{align}
so that 
\begin{align}
  \e(\vert A_1 \vert )  &= 2\,\e\left[\sumin\int_{t_i}^{t_{i+1}}\btghki\left(\int_{t_i}^{s} \sigma(u)\D W_u\right)\sigma(s) \D W_s\right ]\notag \\
  & \le 2\,\e\left[\left\vert\int_0^{1}\sumin D_i(s) \sigma(s) \D W_s\right\vert \right].\notag 
  \label{}
\end{align}
Now using the fact that $\int_0^{t}\sumin D_i(s) \sigma(s) \D W_s$ is a  martingale, we may make an appeal to the BDG inequality to yield:
\begin{align}
  \e(\vert A_1 \vert) & \le c\e\left[\left\vert\int_0^{1}\left(\sumin D_i(s) \sigma(s)\right)^2 \D s\right\vert^{1/2} \right]\notag \\
  & \le c\e\left[\left\vert\int_0^{1}\sumin \{D_i(s) \sigma(s)\}^2 \D s\right\vert^{1/2} \right]\notag,
  \notag
  \label{}
\end{align}
where the last line follows because $D_i (s) D_j(s) = 0$ whenever $i \not= j$. Now if  we define $D^*_i := \sup_{t_i <s \le t_{i+1}} D_i(s)$, and use the fact that $\sigma^2$ is less than  $c$ before $T_c$ then
\begin{align}
  \e(\vert A_1\vert ) & \le c\e\left[\left\vert\sumin \Delta_n(D_i^*)^2 \right\vert^{1/2} \right] \notag\\
  & \le c\e\left[\left\vert\sumin \Delta_n(D_i^*)^2 \right\vert \right] \notag\\
  & \le c\Delta_n\sumin\e[(D_i^*)^2]   
  \label{eq:ra1}
\end{align}
where $c$ is a generic constant representing the bound on $\sigma^2$ and the BDG constant.  Note  from the definition of  $D_i$  that it  is itself a martingale, so we may bound  $D^*_i$  with yet another application of the BDG inequality. That is
\begin{align}
  \e((D^*_i)^2) &\le \e\left(\int^{t_{i+1}}_{t_i} \sigma^2(s) \D s\right) \notag\\
  &\le c \Delta_n. 
  \label{}
\end{align}
Hence, given any $\varepsilon > 0$,  $\p( \sup_{t \in [0,1]} \vert B_{j,n}(t) \vert > \varepsilon)  = O(H_n n^{-1}) + \p(T_c < 1)$.

Collecting the estimates for $B_{j,n}(t)$ for $j =1,\cdots,4$, it is easily seen that $\vert \svnx - \sigma^2(t) \vert$ tends to zero in probability uniformly  for all $t \in \domain$. 
\end{proof}

\section{Volatility estimation: discontinuous prices} 
In this section we specify a global spot volatility estimator for possibly discontinuous \ito semimartingale price processes. That is, for $t \ge 0$,
  \begin{align}
    &X_t = X_0 + \int^t_0 b_s ds + \int^t_0 \sigma_s d W_s  + x  I_{\{\vert x \vert > 1\}} \ast \mu_t  + x  I_{\{\vert x \vert \le  1\}} \ast (\mu - \nu)_t 
  \notag \end{align}
with  $\nu(dt, dx) = F(dx) dt$ for a determinsistic  and constant-in-time $\sigma$-finite measure $F$. We assume $\sigma$ and $b$ satisfy the requirements of Assumption \ref{as:vol}, and we further restrict the \levy system of $X$ as follows:
\begin{ass} \label{as:nu}
  \mbox{}
 %   \item  There is a sequence of stopping times $T_m \uparrow \infty$, a.s., and a sequence $\{F^m\}$ of deterministic $\sigma$-finite measures  on the real line satisfying 
  %    \begin{align} 
   %     \int_\real (x^2 \wedge \vert x \vert)  F^m(dx) <\infty\end{align}
   %     such that $\sup_{t \wedge T_m} F_t (B) \le F^m(B)$ almost surely for all Borel sets $B$.
    %\item $x^2  \ast \nu_t = \int_0^t \int_\real x^2 F_t(dx) dt < \infty$,
  The \levy measure $F$ satisfies the following condition
  $(x^2I_{\{\vert x \vert \le u\}})\ast \nu_t = \int_0^t \int_{-u}^{u} x^2 F( \D x) \D t = O(u)$ as $u \to 0$.
\end{ass}
\begin{remark}
% The first requirement implies the existence of $T_m \uparrow \infty$, a.s., such that $X_{t \wedge T_m} - X_0$ has square integrable small jumps and large jumps  of integrable variataion for each $m$.  The requirement is met by price processes with locally bounded jumps. %It also implies that the jumps of $X$  in excess of one are of locally integrable varaition, so that $X$ is a special martingale.

  The  requirement  is satisfied if $F$ is absolutely continuous with bounded density $f$, as is the case with the Gaussian distribution; more generally, it is satisfied  if  $f(x) = O(x^{-2})$ as $x \to 0$; these include the \levy$(\gamma, \delta)$ distribution with density 
  \begin{align}
    f(x) = (\gamma/2 \pi)^{1/2} (x - \delta)^{-3/2} \exp(-\gamma/ 2(x - \delta)),  \qquad x \in \real, \notag
    \label{}
  \end{align}
  and the Cauchy$(\gamma, \delta)$ distribution with density
  \begin{align}
    f(x) = (\gamma/\pi) (\gamma^2 + (x - \delta)^2)^{-1}, \qquad x > \delta.\notag
    \label{}
  \end{align}

  We also remark that for general semimartingales $(x^2 \wedge 1)\ast  \nu_t$ is increasing and locally integrable. By the \levy assumption, we simply have that  $(x^2 \wedge 1) *  \nu$ is finite. In addition, it is a consequence of the \levy assumption that the price process has no fixed time of discontinuity \citep[II.4.3]{Jacod2003}. Hence, by \ito's integration by parts formula 
  \begin{align}
    \e((x^2\wedge 1)\ast \mu_t) = t(x^2 \wedge 1) *  \nu = O(t), \qquad t \ge 0. \label{eq:secmo}
  \end{align}
\end{remark}

\begin{comment}
 Let $\tau: \real \to \real$ be bounded and satisfy $\tau(x) = x$ in a neighborhood of 0.  Let $\iota$ be the identity  function on the real line, i.e.  $\iota(x) = x$ for $x \in \real$. The price process $X$ admits the following  representation:
\begin{align}
  X_t & = X_0 + \int_0^t b_s \D s + \int_0^t \sigma_s \D W_s +  \tau(x)\ast (\pme  - \nu)_t  + (\iota - \tau)(x)\ast \pme_t ,   
  \label{eq:generalsemimartingale}
\end{align}
for $t \ge 0$  where  \sbm is a standard Brownian motion;  $X_0$ is either known or observable at time 0;  both $b$ and $\sigma$ are adapted; $b$ is \cadlag, and $\sigma$ is continuous;  \pme is a Poisson random measure on $\real_+ \times \real$ with intensity $\nu$, where $\nu$ is a  $\sigma$-finite L\'evy  measure on $\real_+ \times \real$. Note that because $\mu$ is a Poisson measure, if $A$ and $B$ are disjoint Borel sets on $\real_+ \times \real$, then the random measures $\mu(A)$ and $\mu(B)$ are Poisson distributed, independent, and  and have intensity, $\nu(A)$ and $\nu(B)$, respectively. Moreover, because of the \levy assumption on $\nu$, it is the case that $\nu$ does not charge 0 and 
\begin{align}
  (x^2 \wedge 1) \ast\nu_t < \infty, \qquad t \in [0,1],\notag
  \label{}
\end{align}
where $a \wedge b$, with  $a,b \in \real$, denotes the minimum of $a$ and $b$. The  notation $``\ast"$ denotes integration with respect to a random measure. So that 
\begin{align}
  &J_{t}^l :=  \tau\ast (\pme  - \nu)_t = \int_0^t\int_\real \tau(x) [\pme(\D s, \D x)   -  \nu(\D s,\D x)], \notag\\
  & J_{t}^s:= (\iota - \tau)\ast \pme_t = \int_0^t\int_\real [\iota(x) - \tau(x)] \pme(\D s, \D x), \notag
  \label{}
\end{align}
for $t \ge 0$. Both  $J^l$ and $J^s$ are purely discontinuous in the sense that they are orthogonal to all continuous semimartingales. $J^s$ accounts for  small jumps; it is a square-integrable martingale with possibly infinite activity. $J^l$ accounts for large jumps, i.e. jumps with magnitude exceeding the bound on $\tau$; it neccessarily has finite activity so it is a process with finite variation. In the sequel, we will specify $\tau$ as follows:
\begin{align}
  \tau(x) = xI_{\{\vert x \vert \le 1\}}, \qquad x \in \real. \notag
  \label{}
\end{align}
\end{comment}
As in the preceeding section, we observe a realization of the price process at $n + 1$ equidistant points $t_i$,  $i = 0, 1, \cdots, n$. The observation interval is normalized to \domain with  no loss of generality.  The estimator proposed in the previous section, where there is no jump activity, will not do here. It is inconsistent on account of the presence of jumps; its quality deteriorates as a function of how active the jumps of $X$ are. We will counter this phenomenon with a modified spot variance estimator, but first we introduce the following notation. Let $\dx$ denote $X_{t_{i+1}} - X_{t_i}$ for $i = 0, 1,\cdots, n-1$, and let $u_n$  be a positive decreasing sequence such that 
\begin{align}
  u_n  = O(\Delta_n^\beta), \text{ where }\quad 0< \beta < 1.  
  \label{}
\end{align}
 We specify the jump-robust global estimator of  spot volatility as follows: 
\begin{align}
  \label{eq:jumpvolestimator}
  &\jvn(t) := \sum_{(h,k) \in \Theta_n} \anhk\;g_{h,k}(t), \qquad \forall t \in [0,1], \text{ where}\\
  &\anhk := \sum_{i =0}^{n-1} \btghki (\dx)^2 \indx,
\end{align}
where $\{\ghk, \tghk\}$ is a pair of dual Gabor frames constructed as in Lemma \eqref{le:gabor}; $\Theta_n$ retains its meaning from \eqref{eq:theta}; and \indx is one if $(\dx)^2$ is less than or equal to  $u_n$ and zero otherwise.  


There are obvious similarities between \svnx, defined at  \eqref{eq:contvolestimator},  and \jvn with the key difference being that \jvn discards realized squared increments over intervals that likely contain jumps; $u_n$ determines the threshold for what is included in the computation and what is not. This determination becomes more accurate as the observation interval becomes infinitessimally small. Clearly it makes sense to use \svnx if we have reason to believe that the price process is not subject to jumps; \svnx will always employ   all available data and therefore may be assumed to produce more accurate results.  



\begin{comment} \subsection{Finite activity \levy jumps}
In order to demonstrate that the global estimator of spot volatility is consistent, we will proceed in stages.  First suppose the price process specified in all generality in \eqref{eq:generalsemimartingale} experiences at most a finite number of \levy jumps in any finite time interval. That is we assume that $X$ has finite activity  \levy jumps, which is equivalent to $\nu$ being finite on the complement of   $\{0\}$. The finite activity assumption  also implies that the price process may be expressed as  
\end{comment}
We now proceed to prove the consistency of the estimator. First we introduce the following notation and prove an intermediate lemma. Set 
\begin{align}
  &X_t^c :=  X_0 + \int_0^t b_s \D s + \int_0^t \sigma_s \D W_s, \notag\\ 
  &A_t :=  xI_{\{\vert x \vert > 1 \}} \ast \mu_t, \notag \\
  &X^f_t := X^c_t + A_t.
  \label{jumpfa}
\end{align}
We now prove the following:
\begin{comment}
for $t \in \domain$ where the $Y_i$'s are  \iid jump sizes; $N$ is a Poisson process with intensity $\lambda$, independent of each $Y_i$. Under this conditions, we have the following:
\end{comment}
\begin{lem}\label{lem:finite}
  Let $X^f$ be specified as in \eqref{jumpfa} with $\sigma$ and $b$ satisfying Assumption \ref{as:vol}. Let $\{g, \tg\}$ be pair of dual Gabor generators satisfying the conditions of Lemma \eqref{le:gabor} such that $g$ is Lipschitz continuous on the unit interval. If 
  \begin{comment}
  \begin{enumerate}[label=\emph{(}\roman*\emph{)}]
    \item 
  the drift of $X$ satisfies with probability 1:
  \begin{align}
    \limsup_{\Delta_n \to 0} \frac{M^*}{(\Delta_n \log(1/\Delta_n))^{1/2}} \le  C< \infty, \notag
    \label{}
  \end{align}
where $M^* : = \sup_{1 \le i <n} \vert \int^{t_{i+1}}_{t_i} b(s) \D s\vert$;
\item the diffusion coefficient satisfies with probability 1:  $\int^1_0 \sigma^2(s) \D s < \infty$ and 
  \begin{align}
\limsup_{\Delta_n \to 0} \frac{S^*}{\Delta_n} \le B <\infty, \notag
    \label{}
  \end{align}
  where $S^* := \sup_{1 \le i <n} \vert \int^{t_{i+1}}_{t_i} \sigma^2(s) \D s\vert$;
  \end{enumerate}
  \end{comment}
 $u_n = O(\Delta_n^\beta), 0 <\beta <1,$ and $H_n \uparrow \infty$ are sequences satisfying  
\begin{align}
  u_n^{-1/2} (H^n)^2 \Delta_n^{1/2} = o(1),\notag
    \label{}
  \end{align}
  then
  $V_n(X^f, t)$ as defined in \eqref{eq:jumpvolestimator}  converges in \Ltwo in probability to \sv.
\end{lem}
\begin{proof} 
We have
\begin{align}
V_n(X^f,t)  - \sv(t) & = \{V_n(X^f,t)  - V_n(X^c, t)\}  + \{V_n(X^c,t)  - \svn{X^c}\} \notag \\
& \quad +  \{ \svn{X^c} - \sv(t)\}. \label{eq:now}
\end{align}
That the third summand on the right converges to 0 in \Ltwo in  probability is the content of Proposition \ref{pro:finite}. 
Set
    $\hat{b}_{h,k} := \sum_{i =0}^{n-1} \btghki (\dxc)^2I_{\{(\dxc)^2 \le  u_n\}}$ and  
  $\dnhk := \sum_{i =0}^{n-1} \btghki (\dxc)^2$.
Now note that
$V_n(X^c,t)  - \svn{X^c} = \sum_{(h,k) \in \Theta_n} (\hat{b}_{h,k}  - \dnhk)\;g_{h,k}(t)$ 
with 
\begin{align}
  \hat{b}_{h,k}  - \dnhk &= \sum_{i =0}^{n-1} \btghki\{ (\dxc)^2 \indxc- (\Delta_i X^c)^2\}\notag\\
  &= \sum_{i =0}^{n-1} \btghki (\dxc)^2 \indxcn \notag.
  %& = \sum_{i =0}^{n-1} \btghki\{ (\dxc)^2 (\indxc- I_{\{(\dxc)^2 \le 4 u_n\}}\}\notag \\
  %& \quad +\sum_{i =0}^{n-1} \btghki\{(\dxc)^2 I_{\{(\dxc)^2 \le 4 u_n\}} - (\Delta_i X^c)^2\}\notag\\
  %& =: E^1_n + E_n^2.
  \label{}
\end{align} 
Without loss of generality, suppose $b_0 = \sigma_0 = 0$; let $\{T_m\}$ be a localizing sequence for $b$ and $\sigma$.   
Set $\Delta_i M_m := \int_{t_i}^{t_{i+1}} \sigma_{s\wedge T_m} \D W_s$, $\Delta_i S_m := \int_{t_i}^{t_{i+1}} b_{s\wedge T_m} \D s$, and $\dxc_m :=  \Delta_i M_m + \Delta_i S_m$. Define $\hat{b}^m_{h,k}  - \dnhk^m$ as above by substituting $\dxc_m$ for \dxc. Now note the following
\begin{align}
  \e(\vert \hat{b}^m_{h,k}  - \dnhk^m \vert ) &\le c n \e( (\dxc_m)^2 I_{\{( \dxc_m)^2 > u_n\}})\notag\\
  & \le c n \e( (\dxc_m)^4)^{1/2}  \p(( \dxc_m)^2 > u_n)^{1/2}\notag\\
  & \le c n u_n^{-1/2} \e( (\dxc_m)^4)^{1/2}  \e(( \dxc_m)^2)^{1/2}\notag.
  \label{}
\end{align}
Arguing as in Proposition \ref{pro:finite}, it is easily verified that $\e( (\dxc_m)^4) \le  c (\Delta_n^4 + \Delta_n^3 + \Delta_n^2)$ and $ \e( (\dxc_m)^2)  \le  c(\Delta_n^2 + \Delta_n^{3/2} + \Delta_n)$. Hence, $\e(\vert \hat{b}^m_{h,k}  - \dnhk^m \vert ) \le c n u_n^{-1/2}\Delta^{3/2}_n = c u_n^{-1/2}\Delta^{1/2}_n $. Because \tghk is bounded, this allows us to conclude by way of Markov's inequality that given $\eta > 0$, 
\begin{align}
  \p(\sup_{t \in \domain} \vert V_n(X^c,t)  - \svn{X^c} \vert > \eta) \le \p(T_m \le 1) + c u^{-1/2}_n H^n \Delta_n^{1/2},\notag
  \label{}
\end{align}
which becomes arbitrarily small as $m$ and $n$ tend to infinity simultaneously.

To obtain an estimate for the first summand in \eqref{eq:now}, denote
    $\hat{e}_{h,k} := \sum_{i =0}^{n-1} \btghki (\dxf)^2I_{\{(\dxf)^2 \le  u_n\}}$ 
  and observe that 
  $V_n(X^f,t)  - V_n(X^c, t) = \sum_{(h,k) \in \Theta_n} (\hat{e}_{h,k} - \hat{b}_{h,k})\;g_{h,k}(t)$ 
with 
\begin{align}
  \hat{e}_{h,k} - \hat{b}_{h,k}  &= \sum_{i =0}^{n-1} \btghki\{ (\dxf)^2 \indxf- (\dxc)^2 \indxc\}.\notag
  \label{}
\end{align}
By definition $X^f = X^c + A$, where $A$ represents the jumps of $X$ in excess of $1$.  
We may write $ (\dxf)^2 \indxf- (\dxc)^2 \indxc = \gamma^1_i + \gamma^2_i + \gamma^3_i$ with 
  \begin{align}
  &\gamma^1_i := (\dxc)^2 (\indxf-  \indxc), \notag \\
  &\gamma^2_i := (\dxc \Delta_i A) \indxf, \notag\\ 
  &\gamma^3_i := (\Delta_i A)^2 \indxf. 
  \end{align}
Because, $X$ is \cadlag, there is at most a finite number of  jumps in excess of 1  per outcome in \domain. For sufficiently large $n$, each interval $(t_{i} , t_{i + 1}]$  contains at most one  jump.  If the $i$-th interval does not contains a jump    then $\gamma^2_i = \gamma^3_i = 0$ because $\Delta_i A = 0$.   If the $i$-th interval contains a jump, we have  
\begin{align}
  \vert\Delta_i X^f\vert = \vert\Delta_i A+ \Delta_i X^c\vert \ge 1 - \vert \dxc \vert.\label{eq:well1}
\end{align}
Now observe that because $X^c$ has  continuous paths, it is uniformly continuous on the compact domain \domain, so that   as $n$ tends to infinity, $1 - \sup_{i < n}\vert \dxc \vert \uparrow 1$;  meanwhile, $u_n^{1/2} \downarrow 0$. Hence, for $n$ large enough, we have $\vert \dxf \vert > u_n^{1/2}$ so that, almost surely,  $\gamma^2_i$ and $\gamma_i^3$, for all $i$,  are uniformly  eventually zero.

To pin down $\gamma^1_i$, we introduce the following events
\begin{align}
&\Omega_n^1 := \{\omega : \mu(\omega,  (t_i, t_{i+1}] \times \{\vert x \vert > 1\} ) \le 1, \text{ for all } i <n  \}, & n \in \nats, \notag\\
&\Omega_n^2 := \{ \omega: \vert\Delta_i X^c(\omega)\vert <   1-  u_n^{1/2} , \text{ for all }i < n   \},& n \in \nats,\notag\\
&\Omega_k := \{\omega : \mu(\omega, \domain  \times \{\vert x \vert > 1\} ) \le k \}.  & k \in \nats.\notag
  \label{}
\end{align}
Set $\Omega_n := \Omega_n^1 \cap \Omega_n^2$.  As previously argued (see \eqref{eq:well1}), $\p(\Omega^2_n) \to 1$ as $n \to \infty$.  Because $X$ is \cadlag, $\mu(  \domain \times \{\vert x \vert > 1\} ) $ is almost surely finite, so that  $\p(\Omega^1_n) \to 1$ as $n \to \infty$.    Hence, $\p(\Omega_n) \to 1$ as $n \to \infty$. It is also the case that $\p(\Omega_k) \to 1$ as $k \to \infty$ since $X$ is \cadlag and the number of jumps larger than one in any bounded interval must be finite almost surely. Now, recall that $\{T_m \}$ is a localizing sequence for $b$ and $\sigma$;  set $\Omega(m,n,k) := \Omega_n \cap \Omega_k \cap \{T_m > 1\}$ and note that $\p(\Omega(m,n,k)) \to 1$ as $n,m, k \to \infty$.  Thus, on $\Omega(m,n,k)$ there is at most $k$ jumps larger than  one with no more than one  jump per interval; the increments of $X^c$ are small enough to ensure the increments of $X^f$ exceed $u_n^{1/2}$; and the processes $\sigma^4$ and $b^4$ are integrable. 

Set $\gamma^1_i(n, m, k) = \gamma^1_i I_{\Omega(m,n,k)}$  and  denote  $G_i := \{\vert \Delta_i A\vert  > 0\}$. By triangle inequality,  $E(\vert \gamma_i^1(n, m, k) \vert) \le E(\vert \gamma_i^1(n, m, k)  I_{G_i}\vert)  + E(\vert \gamma_i^1(n,m, k) I_{G^c_i}\vert)$. Clearly, $\gamma^1_i(n,m, k) = 0$ on $G^c_i$  so that 
\begin{align}
\sum_{i =0}^{n-1} \btghki E(\vert \gamma_i^1(n,m, k) \vert) & \le\sum_{i =0}^{n-1} \btghki E(\vert \gamma_i^1(n,m, k) I_{G_i}\vert) \notag\\
 & =  \sum_{i =1}^{k} \btghki\e( (\dxc_m)^2 I_{\{(\dxc_m)^2 \le u_n\}}I_{G_i})\notag\\
 &\le \sum_{i =1}^{k} \btghki\e((\dxc_m)^2)\notag \\
 &\le c k \Delta_n\notag.
  \label{}
\end{align}
Hence, given $\eta > 0$, 
\begin{align}
  \p(\sup_{t \in \domain} \vert V_n(X^f,t)  - V_n(X^c, t) \vert > \eta) \le \p(\Omega(m,n,k)^c) + c H^n k \Delta_n.\notag
  \label{}
\end{align}
By taking $m,n,k$ large enough, the first term can be made as small as required; for fixed $m,k$, letting $n\to \infty$ will make the second term as small as desired. This completes the proof. 
\end{proof}

\begin{comment} \subsection{Infinite activity \levy jumps}
We now turn to the case of a price process specified in full generality by \eqref{eq:generalsemimartingale}, that is the price process is a sum of a continuous and a discontinuous process with possibly  infinite activity. The infinite activity assumption is equivalent to  the statement that $\nu$ assigns  infinite measure to the complement of the singleton containing zero.    The following is the consistency Proposition in this more general framework:
\end{comment}
We now prove consistency for the estimator when the price process admits both large and small jumps. That is 
\begin{align}
  X_t = X_0 + X^c_t + J^l_t + J^s_t. \notag
  \label{}
\end{align}
where $J^l_t := (xI_{\{\vert x \vert > 1\}}) \ast \mu_t$ and  $J^s_t := (xI_{\{\vert x \vert \le 1\}}) \ast (\mu - \nu)_t$.
\begin{comment}
First, we make some obvious statements:
\begin{lem}\label{lem:est}
  Let $X$ be an \ito semimartingale meetig the requirements of Assumption \ref{as:vol} and \ref{as:nu}. Let $\{T_m, F^m\}$ be a localizing sequence for $(b, \sigma, F)$. Denote $(X^c_m)_t := X^c_{t \wedge T_m}$,  $(J^l_m)_t := J^l_{t \wedge T_m}$, and $(J^s_m)_t := J^s_{t \wedge T_m}$. Then,
  \begin{enumerate}
    \item $\e((\dxc_m)^p) = O(\Delta_n^{p/2}), \qquad p \in \{2, 4\}$,
    \item $\e((\djt_m)^2) = O(\Delta_n)$,
    \item $\e(\vert \djl_m \vert ) = O(\Delta_n)$.
  \end{enumerate}
\end{lem}
\begin{proof}
That $\e((\dxc_m)^p) = O(\Delta_n^{p/2})$  for $p = 2$ and 4  holds has already been demonstrated in Proposition \ref{pro:finite}. Now, since $J^s$ is a local martingale, we have by  the BDG inequality that,   for $t \in (t_i, t_{i + 1}]$, 
\begin{align}
\e( ( J^s_{t\wedge T_m} - J^s_{t_{i}\wedge T_m})^2) &= \e(  [J^s_{t\wedge T_m} - J^s_{t_{i}\wedge T_m}]) \notag \\
& = \e(\int_{t_i \wedge T_m}^{t\wedge T_m}\int_{\{ \vert x \vert \le 1\}} x^2 F_t(dx) dt)\notag \\
& \le \e(\int_{t_i\wedge T_m}^{t\wedge T_m}\int_\real (x^2 \wedge \vert x \vert ) F_t(dx) dt)\notag \\
& \le  \int_{t_i\wedge T_m}^{t\wedge T_m}\int_\real (x^2 \wedge \vert x \vert) F^m(dx) dt\notag\\
%& \le  (t - t_i)\int_\real x^2 F^m(dx) \notag\\
& =  O(\Delta_n)\notag.
\end{align}
Similarly, 
\begin{align}
\e( \vert J^l_{t\wedge T_m} - J^l_{t_{i}\wedge T_m} \vert) &\le \e( (\vert x\vert I_{\{\vert x \vert > 1\}}) \ast \mu((t_i\wedge T_m, t_{i+1}\wedge T_m] ) \notag\\
& = \e(\int_{t_i\wedge T_m}^{t\wedge T_m}\int_{\{ \vert x \vert > 1\}} \vert x\vert F_t(dx) dt)\notag \\
& \le  \int_{t_i\wedge T_m}^{t\wedge T_m}\int_\real (x^2 \vee \vert x \vert) F^m(dx) dt\notag\\
& =  O(\Delta_n)\notag.
  \label{}
\end{align}
\end{proof}
\end{comment}
\begin{comment}
First, we state some obvious results.
\begin{lem}
 Let the price process  $X$ be specified as in  \eqref{eq:semimartingale}. Then,
 \begin{enumerate}
   \item $(x^2 \wedge 1) \ast \mu$ is locally integrable.
   \item $(\vert x \vert I_{\{\vert x \vert > 1\}}) \ast \mu$ is locally integrable.
 \end{enumerate}
\end{lem}
The first statement follows because $(x^2 \wedge 1) \ast \mu_t$ is dominated by $ (x^2I_{\{\vert x \vert \le 1\}}) \ast \mu_t$ and $ (I_{\{\vert x \vert > 1\}}) \ast \mu_t$, which  are increasing processes with bounded jumps and, therefore, locally integrable.    
\end{comment}
We now give the main result of the paper.
\begin{prop} \label{pro:infinity}
  Let the price process  $X$ be specified as in  \eqref{eq:semimartingale}. We assume that the requirements of Assumption \ref{as:vol} and \ref{as:nu} are met. Let $\{g, \tg\}$ be pair of dual Gabor generators satisfying the conditions of Lemma \eqref{le:gabor} with $g$  Lipschitz continuous on the unit interval. Let  $\{H_n\}$ be an increasing sequence and   $\{u_n\}$  a decreasing sequence statisfying $u_n = O(\Delta_n^\beta)$ with   $0 <\beta<1$. If  
  \begin{align}
    &u_n^{-1/2}(H^n)^2\Delta_n^{1/2} = o(1), \notag\\
    &(H^n)^2u^{1/2}_n = o(1)\notag
    \label{}
  \end{align} 
  \begin{comment}and   that $\nu$ satisfies 
   \begin{align}
    (x^2 \wedge u_n^{1/2}) \ast \nu_1 = o(H_n^{-2}).
    \label{eq:smallo}
  \end{align}
  \end{comment}
  then  \jvn defined in \eqref{eq:jumpvolestimator} converges in \Ltwo in probability to \sv.
\end{prop}
\begin{proof}
  \begin{comment}  We wish to show that the random variable
  $\int_0^1 (\jvn - \sigma^2(t))^2\D t$ tends to zero in probability. The regularity conditions on $X$ and $\sigma^2$ imply that  $\sup_{t \in [0,1]} (\jvn - \sigma^2(t))^2$ is a random variable and that the previous claim would follow as soon as $\sup_{t \in [0,1]} (\jvn - \sigma^2(t))^2$ is shown  to converge to  zero in probability.\end{comment}
  We argue along the lines of Theorem 4 of  \cite{Mancini2009}. First,  consider the following decomposition of the process $X$:
  \begin{align}
    &X = X^f + J^s\label{eq:xj},\\
    &X^f = X^c + J^l\label{eq:xjc},
  \end{align}
  where 
    $X^c_t = \int^t_0 b_s \D s + \int^t_0 \sigma_s \D W_s$, 
    $J^l_t = (xI_{\vert x \vert > 1}) \ast \mu_t,$
    and $J^s_t = (xI_{\vert x \vert \le  1} )\ast (\mu - \nu)_t$. By localization, it is enough to assume $\sigma^4$ and  $b^4$   are integrable. 
    Let $t$ be a point in the unit interval, then 
    \begin{align}
      \jvn -  \sigma^2_t &= \sumt (\ceen{X} - \cee) \ghk(t) - \sumnt \cee \ghk(t),
      \label{eq:open}
    \end{align}
    with $\ceen{X}$ and $\cee$ defined by \eqref{eq:jumpvolestimator} and \eqref{eq:chk}, respectively. The last term tends to zero, almost surely, in \Ltwo as $n \to \infty$ because Gabor frames converge unconditionally. 
    
     To obtain a bound on the first item on the right of \eqref{eq:open}, we may use \eqref{eq:xj} to write
    \begin{align}
      \sumt &(\ceen{X} - \cee)\ghk(t)   = \sumt (w_{h,k} + x_{h,k} + y_{h,k} + z_{h,k}) \ghk(t),\label{eq:summands} 
    \end{align}
    where 
    \begin{align}
      &w_{h,k} :=  \sumin \btghki  (\dxf)^2 \indxff - \int^{1}_{0} \sigma^2(s) \btghks \D s \notag\\
      &x_{h,k} :=  \sumin \btghki(\dxf)^2 (\indx - \indxff) \notag\\
      &y_{h,k}  := 2 \sumin \btghki\dxf \djt \indx \notag \\
      &z_{h,k} := \sumin \btghki(\djt)^2\indx.
      \label{}
    \end{align}
    By Lemma \ref{lem:finite}, if $\delta > 0$ then %\begin{align}  
      $\p(\sup_{t \in [0,1]}\vert \sumt w_{h,k} \ghk(t) \vert > \delta) \to 0$ %\label{eq:w} \end{align} 
    as $n$ tends to infinity.  It remains to show that the last three terms on the right of \eqref{eq:summands} converge to zero in probability. Starting with the second summand, denote $A_i := \{(\dx)^2 \le u_n\}$,  $B_i   := \{(\dxf)^2 \le 4 u_n\}$ and note that $I_{A_i} - I_{B_i} = I_{A_i \cap B_i^c} - I_{A^c_i\cap B_i}$.  Hence, we may write 
    \begin{align}
      \sumt x_{h, k} \ghk(t) = \sumt \sumin \btghki (x^{i, 1} - x^{i, 2}) \ghk(t)  \notag
      \label{}
    \end{align}
   where $ x^{i, 1}  := (\dxf)^2 I_{A_i \cap B_i^c}$ and  $x^{i, 2} := (\dxf)^2I_{A^c_i\cap B_i}$.  
    It is now easily verified using the  reverse triangle inequality that   $A_i \cap B_i^c \subset \{\vert \djt \vert > u_n^{1/2}\}$. So that, 
    \begin{align}
      (\dxf)^2 I_{A_i \cap B_i^c} &\le (\dxf)^2 I_{\{(\djt)^2 > u_n\}}\\
       & \le 2 (\dxc)^2 I_{\{(\djt)^2 > u_n\}}
       + 2 (\djl)^2 I_{\{(\djt)^2 > u_n\}}\notag\\
       & =: v_i + w_i. 
      \label{eq:longineq}
    \end{align}
    It thus follows that   
    \begin{align}
      \sumt \sumin \btghki x^{i,1} \ghk(t) \le \sumt \left(\sumin \btghki (v_i + w_i) \right) \ghk(t). \notag 
  \end{align}
    \begin{comment}
    where \begin{align} &v_n :=  2c H_n \Lambda n^{-1} \log(n) \sumin I_{\{(\djt)^2 > u_n\}} \notag \\ & w_n: = c H_n \sumin  (\djl)^2 I_{\{(\djt)^2 > u_n\}}\notag \end{align}  where $c$ is a sufficiently large constant, and $\Lambda$ is a finite-valued random variable satisfying  $\Lambda \ge  \sup_{t \in \domain} \vert b(t)\vert  + C $, where $C^{1/2}$ is the finite-valued random variable from Lemma \ref{lem:mylevy}.  Let  $\delta > 0$ be given, put  $x_n(t)  :=  \sumt\left(\sumin  \btghki(\dxf)^2 I_{A_i \cap B_i^c})\right)\ghk(t)$  and note that 
    \begin{align}
      \p &\left(  \sup_{t \in \domain}\vert x_n (t) \vert  > \delta\right) \le \p(v_n > \delta/2) + \p(w_n > \delta/2) \notag. 
      \label{}
    \end{align}
  Now let  $\varepsilon > 0$ be given and note that because  $\Lambda$ is almost surely finite,   there is a sufficiently large $K > 0$ such that $\p(\Lambda > K) \le \varepsilon/2$. Hence, 
\end{comment}
We  proceed by using \holder's inequality and \eqref{eq:secmo}   to write
\begin{align} 
  \e(v_i) & \le c (\e( (\dxc)^4))^{1/2} \p( (\djt)^2 > u_n)^{1/2} \notag \\ 
  & \le c u_n^{-1/2} \e( (\dxc)^4)^{1/2} \e( (\djt)^2)^{1/2} \notag \\ 
  & \le c u_n^{-1/2}\Delta_n^{3/2} \label{eq:vi}.
\end{align}
Hence,  by Markov's inequality and the boundedness of $\ghk$ \begin{align} \sup_{t \in \domain}\sumt \sumin \btghki v_i \ghk(t) = O_P(u_n^{-1/2}H^n \Delta_n^{1/2}),\end{align} 
which by assumption tends to zero in probability.

As for the term involving $w_i$,   recall  that because $\mu$ is a Poisson random measure, if $A$ and  $B$ are disjoint measurable sets  in $\real^+\times \real$    then $\mu(A)$ is independent of $\mu(B)$. Using this fact, we may write  given $\eta > 0$    
\begin{align} 
  \p( \sup_{t \in \domain}\vert &\sumt \sumin \btghki w_i \ghk(t)\vert > \eta) \notag \\
  &\le \p\left(\cup_i\{\mu( (t_i, t_{i + 1}] \times \{\vert x \vert > 1\}) > 0   , (\djt)^2 > u_n\}\right) \notag \\ 
  & \le n \p(\mu( [0, t_1] \times \{\vert x \vert > 1\}) > 0 ) E( ( J^s_{t_1})^2) u_n^{-1} \notag \\ 
  & \le c \Delta_n u_n^{-1}\notag,
\end{align} 
which clearly tends to zero in $n$. This concludes the demonstration that $ \sumt \sumin \btghki x^{i,1} \ghk(t)$ tends to zero in probability.  
  \begin{comment}
\p(v_n > \delta /2) & \le 2cKH_n\delta^{-1}E(n^{-1} \log(n) \sumin I_{\{(\djt)^2 > u_n\}}) + \p(\Lambda > K) \notag  \\ & = 2cK H_n\delta^{-1}\log(n) \p( (\Delta_{1/n}J^s)^2 > u_n)  + \varepsilon/2 \notag \\ &\le  2cK H_n\delta^{-1}\log(n) E((\Delta_{1/n} J^s)^2))u_n^{-1}   + \varepsilon/2 \notag \\ &\le 2cKH_n\delta^{-1} \log(n) n^{-1}\kappa u_n^{-1} +  \varepsilon/2\label{eq:asabove} \end{align} where $\kappa := E((\Delta_1 J^s)^2)) < \infty$.  Obviously there is a large enough $n$ such that the first expression above is less than or equal to $\varepsilon/2$.
\end{comment}
\begin{comment} Moreover, because $\delta >  0$,  \begin{align} \p(w_n > \delta/2) & \le \p\left(\cup_i\{I_{\{\vert x \vert > 1\}}\ast \mu( (t_i, t_{i + 1}] \times \real) > 0   , (\djt)^2 > u_n\}\right) \notag \\ & \le n \p(\mu( [0, 1/n] \times \{\vert x \vert > 1\}) > 0 ) E( (\Delta_1 J^s)^2) u_n^{-1} \notag \\ & \le c n^{-1} \kappa u_n^{-1}\notag,\end{align} which clearly tends to zero in $n$. 
\end{comment}
To tackle the term $ \sumt \sumin \btghki x^{i,2} \ghk(t)$, we start with the following definitions:
\begin{align}
&\Omega_n^1 := \{ \omega: \vert\Delta_i X^c(\omega)\vert <   1-  2u_n^{1/2} , \text{ for all }i < n   \},\notag \\
%\{\omega : \vert \dxf(\omega) \vert > 2u^{1/2}_n, \forall  i < n\},\notag \\ 
&\Omega_n^2 := \{\omega : \mu(\omega,  (t_i, t_{i+1}] \times \{\vert x \vert > 1\} ) \le 1, \forall i <n  \},\notag 
\end{align}
$\forall n \in \nats$. These sets are clearly measurable. Denote $\Omega_n   := \Omega_n^1 \cap \Omega_n^2 $. Since there can be at most a finite number of jumps  larger than 1  in magnitude    on \domain,  and  $1 - 2u_n^{1/2} \uparrow 1$ while $ \Delta_iX^c \downarrow  0 $  uniformly  on \domain, it follows that $ \p(\Omega_n) \to 1$ as $n \to \infty$. Now note that 
\begin{align}
  \notag
  A_i^c \cap B_i \cap \Omega_n & \subset  \{ (\dxc + \djt)^2 > u_n \} \notag\\
  &\subset \{(\dxc)^2 > u_n/4\} \cup \{(\djt)^2 > u_n/4\}\notag .
  \label{}
\end{align}
Hence, by successive applications of \holder and Markov inequalities,   
\begin{align}
  \notag
 \e(&(\dxf)^2 I_{A_i^c \cap B_i \cap \Omega_n} ) = \e((\dxc)^2 I_{A_i^c \cap B_i \cap \Omega_n} )\notag \\
 &\le \e((\dxc)^2 I_{\{(\dxc)^2 > u_n/4\}}) + \e((\dxc)^2 I_{\{(\djt)^2 > u_n/4\}}) \label{} \notag\\
 & \le c \Delta_n^{3/2} u_n^{-1/2}.\notag
\end{align}
Let $\eta$ be a given positive number; it is now clear that
\begin{align}
  \p(\sup_{t \in \domain} \vert \sumt\sumin  \btghki x^{i,2} \ghk(t)\vert  > \eta) \le  \p(\Omega_n^c)  + c u^{-1/2}_nH^n\Delta_n^{1/2}, \notag
  \label{}
\end{align}
which tends to zero.  This completes the demonstration that \begin{align}\label{eq:x} \p(\sup_{t \in [0,1]}\vert \sumt x_{h,k} \ghk(t) \vert > \eta) \to 0. \end{align}
Now we show the third summand in  \eqref{eq:summands} tends to zero. First, denote $C_i := \{( \djt )^2 \le 4 u_n\}$, $p_{h,k} : = 2 \sumin \btghki\dxf \djt I_{A_i \cap C_i}$, and $q_{h,k} := 2 \sumin \btghki\dxf \djt I_{A_i \cap C_i^c}$. Clearly, 
\begin{align} 
  \sumt y_{h,k} \ghk(t)  = \sumt (p_{h,k} + q_{h,k}) \ghk(t).\notag
\end{align}
Treating the term involving $q_{h,k}$ first, note that by the reverse triangle inequality, we may  write $A_i \cap C_i^c \subset \{ u_n^{1/2} < \vert \dxf \vert  \} \subset \{ u_n^{1/2}/2 < \vert \dxc \vert  \} \cup  \{ u_n^{1/2}/2 < \vert \djl \vert  \} =: G^1_i \cup G^2_i\notag$. So that 
\begin{align}
  \dxf \djt I_{A_i \cap C_i^c} &\le \dxf \djt  (I_{G^1_i} +  I_{G^2_i}) \notag\\
  & \le \dxc \djt(I_{G^1_i} +  I_{G^2_i}) + \djl \djt(I_{G^1_i} +  I_{G^2_i}) \notag \\
  & =: \gamma_i^1 + \gamma_i^2 + \gamma_i^3 + \gamma^4_i.\notag 
\end{align}
Hence, \begin{align} \notag \sumt q_{h,k} \ghk(t) \le \sumt (\sumin \btghki (\gamma_i^1 + \gamma_i^2 + \gamma_i^3 + \gamma^4_i)) \ghk(t). \end{align} We show in turn that each summand converges to zero. First, observe that
\begin{align}
  \e(\gamma_i^1) &\le  \e( (\dxc  I_{G^1_i})^2)^{1/2} \e( (\djt)^2)^{1/2} \notag\\
  &\le  \e( (\dxc)^4)^{1/4}  E(I_{G^1_i})^{1/4} \e( (\djt)^2)^{1/2} \notag\\
  & \le c \Delta_n^{1/2} (u^{-1/2}_n \Delta_n^{1/2}) \Delta_n^{1/2} \notag \\
  & \le c u^{-1/2}_n \Delta_n^{3/2}.
  \label{}
\end{align}
Hence, given positive $\eta$, 
\begin{align} 
  \p( \sup_{t \in \domain}\vert \sumt \sumin \btghki \gamma^1_i \ghk(t)\vert > \eta) \le cH^n (u^{-1}_n\Delta_n)^{1/2}. \label{eq:g1} \end{align}
Secondly, we have 
\begin{align}
  \e(\gamma_i^2) & =  \e(\dxc \djt I_{G^2_i}) \notag\\
  & \le \e((\dxc)^2 I_{G^2_i})^{1/2} \e((\djt)^2)^{1/2}\notag \\
  &\le \e((\dxc)^4)^{1/4} \p(\djl > u_n^{1/2}/2)^{1/4} \e((\djt)^2)^{1/2}\notag\\
  & \le  c u^{-1/8}_n \Delta_n^{5/4}. \notag
\end{align}
So that given positive $\eta$,
\begin{align} 
  \label{eq:g2}
  \p( \sup_{t \in \domain}\vert \sumt \sumin \btghki \gamma^2_i \ghk(t)\vert > \eta) \le cH^n (u^{-1/2}_n\Delta_n)^{1/4}.  \end{align}
Moreover, 
\begin{align}
\p( \sup_{t \in \domain}\vert &\sumt \sumin \btghki \gamma^3_i  \ghk(t)\vert > \eta) \notag \\
&\le \p(    \cup_i\{ \mu( (t_i, t_{i+1}] \times \{\vert x \vert > 1 \} ) > 0,  (\dxc)^2 > u_n/4\}) \notag \\ & \le c \Delta_n u_n^{-1}. \label{eq:g3}\end{align}
Finally,
\begin{align}
\p( \sup_{t \in \domain}\vert &\sumt \sumin \btghki \gamma^4_i  \ghk(t)\vert > \eta) \notag \\
&\le \p(    \cup_i\{ \mu( (t_i, t_{i+1}] \times \{\vert x \vert > 1 \} ) > 0,  (\djt)^2 > u_n/4\}) \notag \\ & \le c \Delta_n u_n^{-1}.\label{eq:g4}\end{align}
\begin{comment}

$\Omega_q := \{\omega : \mu(\omega, \domain  \times \{\vert x \vert > 1\} ) \le q \}$, for     $q \in \nats$. It is easily seen that $\p(\Omega_q) \to 1$ as $q \to \infty$. Set $\Omega(n, q) := \Omega_n \cap \Omega_q$. Now note that   $\e(\gamma^3_i I_{\Omega(n,q)}) \le q \e(\djt I_{G^1_i}) \le c q \Delta_n^{3/2} u_n^{-1/2}$. So that given positive $\eta$,
\begin{align} 
  \p( \sup_{t \in \domain}&\vert \sumt \sumin \btghki \gamma^3_i \ghk(t)\vert > \eta) \notag \\ &\le \p(\Omega(n,q)^c) + cqH^n (u^{-1}_n\Delta_n)^{1/2}, \notag \end{align}
which can be made arbitrarily small by  choosing $n,q$ large enough and letting $n \to \infty$.  Similarly note that $\e(\gamma^4_i I_{\Omega(n,q)}) \le q \e(\djt I_{G^2_i}) \le c q \Delta_n u_n^{-1/4}$. So that 
\begin{align} 
  \p( \sup_{t \in \domain}\vert \sumt \sumin \btghki \gamma^4_i \ghk(t)\vert > \eta) \le \p(\Omega(n,q)^c) + cqH^n u^{-1/4}_n\Delta_n, \notag \end{align}
which can be made as small as desired. This completes the demonstration  that 
\begin{align}
  \p(\sup_{t \in \domain} \vert \sumt q_{h,k} \ghk(t) \vert > \eta) \to 0.\notag
  \label{}
\end{align}
%\begin{comment}
Now, arguing as in Theorem 4.1 of \cite{Mancini2009}, note that on $ A_i \cap C_i^c$, it is the case that  $2u_n^{1/2} - \vert \dxf \vert < \vert \djt\vert - \vert \dxf\vert \le \vert \dx \vert \le u_n^{1/2}$, so that $u_n^{1/2} < \vert \dxf \vert < \vert \djl \vert + \vert \dxc \vert $. In turn, the last inequality implies that either $\vert \djl \vert > u^{1/2}_n/2$ or  $\vert \dxc \vert > u^{1/2}_n/2$. Now, for sufficiently large $n$, it is almost surely never the case that $\vert \dxc \vert > u^{1/2}_n/2$ for some $i$,  $0 \le  i  \le n -1$. Hence,  for positive $\delta$, \begin{align} \p(\vert&\sumt q_{h,k} \ghk(t) \vert > \delta/2) \notag \\ &\le \p(    \cup_i\{ \mu( (t_i, t_{i+1}] \times \{\vert x \vert > 1 \} ) > 0,  (\djt)^2 > u_n\}) \notag \\ & \le c n^{-1} \kappa u_n^{-1}.\end{align}
\end{comment}
%\end{comment}
We conclude by  reference to the estimates in \eqref{eq:g1}, \eqref{eq:g2}, \eqref{eq:g3}, and \eqref{eq:g4} that  $\sup_{t \in \domain} \vert \sumt q_{h,k} \ghk (t) \vert $ tends to zero in probability.

We now show that $\sumt p_{h,k} \ghk (t)$ tends to zero uniformly  in probability.
To that end, let $\Psi_n := \{\omega: \vert \dxc(\omega) \vert > u_n^{1/2} \text{ for some } i < n\}$.  It now follows by Markov's inequality that 
\begin{align}
  \p(\Psi_n ) &\le \sumin \p(\vert \dxc \vert > u_n^{1/2}) \notag \\
  & \le u_n^{-3/2(1 - \beta) } \sumin \e( (\dxc)^{ 3/(1 - \beta)} ) \notag\\
  & \le  c \Delta_n^{1/2}.
  \label{}
\end{align}
Hence, $\p(\Psi_n ) \to 0$. On $A_i \cap C_i \cap \Psi_n^c$, it is easily seen that $\vert \djl \vert - \vert \dxc + \djt \vert < \vert \dx \vert \le u_n^{1/2}$, so that $\vert \djl\vert \le u^{1/2}_n + \vert \dxc \vert + \vert \djt\vert$. It is therefore the case that  $ \vert \djl \vert  = O(u^{1/2}_n)$. Let $r_{h,k} :=  2 \sumin \btghki\dxc \djt I_{A_i \cap C_i \cap \Psi_n^c}$ and $s_{h,k} := 2 c u_n^{1/2}\sumin \btghki \djt I_{A_i \cap C_i \cap \Psi_n^c}$. Then given  $\delta > 0$ and $\varepsilon > 0$, 
\begin{align}
  &\p(\sup_{t \in \domain}\vert\sumt p_{h,k} \ghk(t) \vert > \delta)\notag \\&  \le \p(\Psi_n) + \p(\sup_{t \in \domain} \vert\sumt r_{h,k} \ghk(t) \vert > \delta/2)  \notag \\
  &\quad + \p(\sup_{t \in \domain} \vert\sumt s_{h,k} \ghk(t) \vert > \delta/2).
  \label{eq:phk}
\end{align}
Now consider that $\sumt r_{h,k} \ghk(t) \le  c H_n \sumin \vert \dxc \djt  I_{A_i \cap C_i} \vert$; this implies that given $\varepsilon > 0$  
\begin{align}
  \p(&\vert\sumt r_{h,k} \ghk(t) \vert > \delta/2) \le \p(c H_n  \sumin \vert \dxc \djt I_{A_i \cap C_i} \vert  > \delta/2)\notag\\& \le \p\left( \left(\sumin (\dxc)^2\right)^{1/2} \left(\sumin ( \djt I_{A_i \cap C_i})^2\right)^{1/2} > \delta (2H_nc)^{-1}\right).\notag
  \label{}
\end{align}
We now use the well-known fact  that  $\sumin (\dxc)^2 (t)$ converges to $\int_0^t\sigma^2(s)\D s$ in probability uniformly on compact intervals \citep[Theorem II.22]{Protter2004}. That is, there is a sufficiently large $N$ such that if $n$ is larger than or equal to $N$ then   $ \p(\vert (\sumin (\dxc)^2)^{1/2}  - (\int_0^1\sigma^2(s) \D s)^{1/2}\vert > \delta ) \le  \varepsilon/4$, and because integrated volatility is almost surely finite, there is a sufficiently large $K$ satisfying  $K/2 > \delta$ such that $\p( \int_0^1\sigma^2(s) \D s > K/2) \le \varepsilon/4.$ Hence, we may write
\begin{align}
\p(&\vert\sumt r_{h,k} \ghk(t) \vert > \delta/4) \notag \\&\le \p\left(\sumin ( \djt I_{A_i \cap C_i})^2 > \delta^2 (4K H_nc)^{-2}\right) + \varepsilon/2\notag\\&\le  \p\left( (x^2I_{\{\vert x\vert \le 1 \wedge 2 u_n^{1/2}\}})\ast\mu_1 > \delta^2 (4K H_nc)^{-2}\right) + \varepsilon/2\notag\\
&\le \delta^{-2} (4K H_nc)^{2}\e\left( (x^2I_{\{\vert x\vert \le 1 \wedge 2 u_n^{1/2}\}})\ast\mu_1  \right) + \varepsilon/2\notag\\
&\le \delta^{-2} (4K H_nc)^{2} (x^2I_{\{\vert x\vert \le 1 \wedge 2 u_n^{1/2}\}})\ast\nu_1   + \varepsilon/2\notag
  \label{}
\end{align}
which for sufficiently large $n$ is  less than $\varepsilon$.
Now it is easily seen that for sufficiently large $c$
\begin{align}
  \p(&\vert \sumt s_{h,k} \ghk(t) \vert > \delta/4) \le \p( \sumin  \djt I_{A_i \cap C_i} > (8 c u_n^{1/2})^{-1}\delta) \notag\\
  &\le (64 c^2 u_n)\delta^{-2} \e\left( (x^2I_{\{\vert x\vert \le 1 \wedge 2 u_n^{1/2}\}})\ast\mu_1  \right)\notag \\
  & \le  (64 c^2 u_n)\delta^{-2} (x^2I_{\{\vert x\vert \le 1 \wedge 2 u_n^{1/2}\}})\ast\nu_1
  \label{}
\end{align}
which, as above, is  less than $\varepsilon/4$ for sufficiently large $n$.
Hence, 
\begin{align}
  \notag
  \p(\sup_{t \in [0,1]}\vert \sumt y_{h,k} \ghk(t) \vert > \delta) \to 0.\label{eq:y}
\end{align}
Next,  write $z_{h,k} = a_{h,k} + b_{h,k}$ where $a_{h,k} :=  \sumin \btghki(\djt)^2I_{A_i \cap C_i}$ and  $b_{h,k} := \sumin \btghki(\djt)^2I_{A_i \cap C_i^c}$. Then 
\begin{align}
  \p(&\vert \sumt z_{h,k} \ghk(t) \vert   > \delta) \notag \\ &\le \p(\vert \sumt a_{h,k} \ghk(t) \vert > \delta/2) + \p(\vert \sumt b_{h,k} \ghk(t) \vert > \delta/2)\notag.
\end{align}
Note,
\begin{align}
  \p(&\vert \sumt b_{h,k} \ghk(t) \vert > \delta/2) \notag \\ &\le
\p\left(\cup_i\{I_{\{\vert x \vert > 1\}}\ast \mu( (t_i, t_{i + 1}] \times \real) > 0   , (\djt)^2 > 4u_n\}\right) \notag \\
&\le n \p(I_{\{\vert x \vert > 1\}}\ast \mu( [0, 1/n] \times \real) > 0) \e((\Delta_{1/n} J^s)^2)(4u_n)^{-1}\notag \\
&\le cn^{-1}\kappa  u^{-1}_n.
\end{align}
which can be made as small as desired. Now consider
\begin{align}
  \p(&\vert \sumt a_{h,k} \ghk(t) \vert > \delta/2) \notag \\ &\le \p( \sumin (\djt)^2 I_{\{\vert \djt \vert \le 2u_n^{1/2} \}} > \delta (2 c H_n)^{-1} )\notag\\
  &\le\delta^{-1} (2 c H_n)\e\left( x^2I_{\{\vert x\vert \le 1 \wedge 2 u_n^{1/2}\}}\ast\mu_1  \right)\notag \\
  &\le \delta^{-1} (2 c H_n) (x^2I_{\{\vert x\vert \le 1 \wedge 2 u_n^{1/2}\}}\ast\nu)_1\notag \\
  &\le c H^n u^{1/2}_n \notag
\notag
\end{align}
which can be made arbitrarily small by the constraints on $H^n$.
Hence, 
\begin{align}
  \p(\sup_{t \in [0,1]}\vert \sumt z_{h,k} \ghk(t) \vert > \delta) \to 0.\label{eq:z}
\end{align}
\begin{comment}
Meanwhile, on $A_i \cap C_i$, it is easily seen that $\vert \djl \vert - \vert \dxc + \djt \vert < \vert \dx \vert \le u_n^{1/2}$, so that
\begin{align}
  \notag
  \dxf \djt I_{A_i \cap C_i} &\le \dxf \djt I_{\{\vert \djl \vert \le 4 u^{1/2}_n\}}\\
   &\le \dxc \djt I_{\{\vert \djl \vert \le 4 u^{1/2}_n\}} +  \djl \djt I_{\{\vert \djl \vert \le 4 u^{1/2}_n\}}\notag\\
   & =: \theta_i^1 + \theta_i^2\notag. 
  \label{}
\end{align}
Arguing as above, it is easily verified that $\e(\theta_i^1)  \le c \Delta_n^{5/4} u_n^{-1/8}$  so that 
\begin{align} 
  \p( \sup_{t \in \domain}\vert \sumt \sumin \btghki \theta^1_i \ghk(t)\vert > \eta) \le cH^n (u^{-1/2}_n\Delta_n)^{1/4} \notag \end{align}
Similarly,  $\e(\theta_i^2 I_{\Omega(n,q)}) \le q c \Delta_n u^{-1/4}_n$ so that  
\begin{align} 
  \p( \sup_{t \in \domain}\vert \sumt \sumin \btghki \theta^2_i &\ghk(t)\vert > \eta) \notag \\ &\le \p(\Omega(n,q)^c) + cH^n (u^{-1/2}_n\Delta_n)^{1/4} \notag \end{align}
Hence, 
\begin{align}
  \p(\sup_{t \in \domain} \vert \sumt p_{h,k} \ghk(t) \vert > \eta) \to 0.\notag
  \label{}
\end{align}
This completes the demonstration  that 
\begin{align}
  \p(\sup_{t \in \domain} \vert \sumt y_{h,k} \ghk(t) \vert > \eta) \to 0.\notag
  \label{}
\end{align}

$\vert \djl\vert \le u^{1/2}_n + \vert \dxc \vert + \vert \djt\vert \le 4u^{1/2}_n$. So that  On the other hand, $ \vert \djl \vert < u_n^{1/2} + \Lambda n^{-1/2} \log^{1/2}(n) + 2 u_n^{1/2} = O(u^{1/2}_n)$. Let $r_{h,k} :=  2 \sumin \btghki\dxc \djt I_{A_i \cap C_i}$ and $s_{h,k} := 2 c u_n^{1/2}\sumin \btghki \djt I_{A_i \cap C_i}$. Then 
\begin{align}
 \p(\vert&\sumt q_{h,k} \ghk(t) \vert > \delta/2)\notag \\&  \le\p(\vert\sumt r_{h,k} \ghk(t) \vert > \delta/4) + \p(\vert\sumt s_{h,k} \ghk(t) \vert > \delta/4).\notag
  \label{}
\end{align}
Now consider that $\sumt r_{h,k} \ghk(t) \le  c H_n \sumin \dxc \djt I_{A_i \cap C_i}$, which implies that  
\begin{align}
  \p(&\vert\sumt r_{h,k} \ghk(t) \vert > \delta/4) \le \p(c H_n \vert \sumin \dxc \djt I_{A_i \cap C_i} \vert  > \delta/4)\notag\\& \le \p\left( \left(\sumin (\dxc)^2\right)^{1/2} \left(\sumin ( \djt I_{A_i \cap C_i})^2\right)^{1/2} > \delta (4H_nc)^{-1}\right).\notag
  \label{}
\end{align}
It is a well known fact that  $\sumin (\dxc)^2 (t)$ converges to $\int_0^t\sigma^2(s)\D s$ in probability uniformly on the unit interval. Hence, there is a sufficiently large $N$ such that if $n > N$ then $ \p(\vert (\sumin (\dxc)^2)^{1/2}  - (\int_0^1\sigma^2(s) \D s)^{1/2}\vert > \delta ) \le  \varepsilon/4$, and because integrated volatility is almost surely finite, there is a sufficiently large $K$ satisfying  $K/2 > \delta$ such that $\p( \int_0^1\sigma^2(s) \D s > K/2) \le \varepsilon/4.$ Hence, we may write
\begin{align}
\p(&\vert\sumt r_{h,k} \ghk(t) \vert > \delta/4) \notag \\
  &\le \p\left(\sumin ( \djt I_{A_i \cap C_i})^2 > \delta^2 (4K H_nc)^{-2}\right) + \varepsilon/2\notag\\
  &\le  \p\left( (x^2I_{\{\vert x\vert \le 1 \wedge 2 u_n^{1/2}\}})\ast\mu_1 > \delta^2 (4K H_nc)^{-2}\right) + \varepsilon/2\notag\\
&\le \delta^{-2} (4K H_nc)^{2}\e\left( (x^2I_{\{\vert x\vert \le 1 \wedge 2 u_n^{1/2}\}})\ast\mu_1  \right) + \varepsilon/2\notag\\
&\le \delta^{-2} (4K H_nc)^{2} (x^2I_{\{\vert x\vert \le 1 \wedge 2 u_n^{1/2}\}})\ast\nu_1   + \varepsilon/2\notag
  \label{}
\end{align}
which for sufficiently large $n$ is  less than $\varepsilon$ by \eqref{eq:smallo}.

Now it is easily seen that for sufficiently large $c$
\begin{align}
  \p(&\vert \sumt s_{h,k} \ghk(t) \vert > \delta/4) \le \p( \sumin  \djt I_{A_i \cap C_i} > (8 c u_n^{1/2})^{-1}\delta) \notag\\
  &\le (64 c^2 u_n)\delta^{-2} \e\left( (x^2I_{\{\vert x\vert \le 1 \wedge 2 u_n^{1/2}\}}\ast\mu)_1  \right)\notag \\
  & \le  (64 c^2 u_n)\delta^{-2} (x^2I_{\{\vert x\vert \le 1 \wedge 2 u_n^{1/2}\}}\ast\nu)_1
  \label{}
\end{align}
which, as above, is  less than $\varepsilon/4$ for sufficiently large $n$.
Hence, 
\begin{align}
  \notag
  \p(\sup_{t \in [0,1]}\vert \sumt y_{h,k} \ghk(t) \vert > \delta) \to 0.\label{eq:y}
\end{align}

Next,  write $z_{h,k} = a_{h,k} + b_{h,k}$ where $a_{h,k} :=  \sumin \btghki(\djt)^2I_{A_i \cap C_i}$ and  $b_{h,k} := \sumin \btghki(\djt)^2I_{A_i \cap C_i^c}$. Then 
\begin{align}
  \p(&\vert \sumt z_{h,k} \ghk(t) \vert   > \delta) \notag \\ &\le \p(\vert \sumt a_{h,k} \ghk(t) \vert > \delta/2) + \p(\vert \sumt b_{h,k} \ghk(t) \vert > \delta/2)\notag.
\end{align}
In the first instance,
\begin{align}
  \p(&\vert \sumt b_{h,k} \ghk(t) \vert > \delta/2) \notag \\ &\le
\p\left(\cup_i\{I_{\{\vert x \vert > 1\}}\ast \mu( (t_i, t_{i + 1}] \times \real) > 0   , (\djt)^2 > 4u_n\}\right) \notag \\
&\le n \p(I_{\{\vert x \vert > 1\}}\ast \mu( [0, 1/n] \times \real) > 0) \e((\djt)^2)(4u_n)^{-1}\notag \\
&\le cn^{-1}\kappa  u^{-1}_n.
\end{align}
which can be made as small as desired. Now consider
\begin{align}
  \p(&\vert \sumt a_{h,k} \ghk(t) \vert > \delta/2) \notag \\ &\le \p( \sumin (\djt)^2 I_{\{\vert \djt \vert \le 2u_n^{1/2} \}} > \delta (2 c H_n)^{-1} )\notag\\
  &\le\delta^{-1} (2 c H_n)\e\left( x^2I_{\{\vert x\vert \le 1 \wedge 2 u_n^{1/2}\}}\ast\mu_1  \right)\notag \\
  &\le \delta^{-1} (2 c H_n) (x^2I_{\{\vert x\vert \le 1 \wedge 2 u_n^{1/2}\}}\ast\nu)_1
\notag
  \label{}
\end{align}
which can be made arbitrarily small.
Hence, 
\begin{align}
  \p(\sup_{t \in [0,1]}\vert \sumt z_{h,k} \ghk(t) \vert > \delta) \to 0.\label{eq:z}
\end{align}
The result follows from \eqref{eq:w},\eqref{eq:x},\eqref{eq:y}, and \eqref{eq:z}.
\end{comment}
\end{proof}


\begin{comment}
\begin{remark}\mbox{}
  \begin{enumerate}
    \item The adaptive approach is not the only way to specify this estimator. We could have instead said something like the volatility paths are all Holder continuous with exponent at least $\alpha$ and are all contained in a Holder ball of at most $c$. This would give us a minimal rate. The holder ball is actually not too bad either. 
    \item The approach taken here is a little more theoretical and frankly does not add much more in terms approachability of friendliness. But it is the more mature approach. It lacks usefulness but it is precise.
    \item The difficulties here are of serveral order. 1. The thing to be estimated \sv is itself random. 2. the realisations of the thing to be estimated are paths not single numbers. 3. the paths of the thing to be estimated can vary very widely with regards smoothness. For example, the Brownian process can produce paths of anysmoothness imaginable; of course some of this paths have zero probability. The point is there is no single smoothness criterion that captures all the possibilities adequately. The best we can do is to restrict the type of processes to be considered by say something like only processes whose paths is Holder continuous with exponent at least $\alpha > 0$, but this will exclude the Brownian motion as a possible driver. In fact all non-deterministic semimartingales.
    \item the adaptive approach seems like the less of two evils.
    \item Under the assumption that \sv is compactly supported, could it be taken  for granted that the modulus of continuity is proportional to the step size? 
    \item Zigmonds theorem says $|\hat{f}(\rho)| \le (1/2)\omega(\pi/|\rho|)$. This works for $2\pi$ periodic functions defined on the entire real line. It possibly holds when the subset of the real line considered is compact.
  \end{enumerate}
\end{remark}
\end{comment}





\begin{comment}
\section{Asymptotic properties} \label{sec:deviation}

Let $R_n$ denote the average integrate square deviation of \svn from \bsv, i.e.  
\begin{align}
R_n = \e \int_\real \{\svn(t) - \bsv(t)\}^2 \lambda(t) \D t,
  \label{eq:remeainder}
\end{align}
where $\lambda$ is a positive and continuous weight function with support in $(0,1)$. The weight function allows us to emphasis different time windows when estimating the volatility. For instance, we may wish to emphasize the recent past in real-time applications. 
We show that $R_n$ tends to 0 as a function of the sample size, $n$. Note that $R_n$ is the sum of a bias and a variance component, which we write as follows:
\begin{align}
  &R_n = B_n^2 + V_n, \notag \qquad \text{ } 
\end{align}
where
\begin{align}
  &B^2_n := \int_\real (\e[\svn(t) - \bsv(t)])^2 \lambda(t) \D t \label{eq:ibias}\\
  &V_n :=  \int_\real \e[\{\svn(t) - \e[\svn(t)]\}^2] \lambda(t) \D t. \label{eq:ivar}
\end{align}
\begin{prop} \label{pr:consistency}
  Let $\{g, \tg\}$ be pair of dual Gabor generators constructed as in Lemma \ref{le:gabor}.   Suppose the conditions in  Assumption \ref{as:vol} and \eqref{as:barvol} hold. If  $H_n^2 \Delta_n, H_n\omega_g(\Delta_n)$, and  $\omega_{\sigma^2}(1/H_n) \log H_n\to 0$, then the mean integrated square error
  $R_n$ tends to 0 as $n$ tends to infinity, with 
  \begin{align}
    & B_n^2 = O(H_n^2\Delta_n + \{H_n\omega_g(\Delta_n)\}^2 + \{\omega_{\sigma^2}(1/H_n) \log H_n\}^2)\notag \\
    & V_n = O(H_n^2 \Delta_n),
    \label{}
  \end{align}
  where $\omega_g(\delta) := \sup \{ |g (t) - g(t')|: t, t' \in \real \text{ and } |t -t'| < \delta\}$. 
\end{prop}
\begin{proof}
  See Appendix \ref{ap:proof}.
\end{proof}

\end{comment}
