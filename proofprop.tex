\begin{prop}\label{pro:finite}
  Suppose the  price process is specified as in \eqref{eq:contsemimartingale} and satisfies the conditions of Assumption \ref{as:vol}. Let $\{g, \tg\}$ be pair of dual Gabor generators satisfying the conditions of Lemma \ref{le:gabor} such that $g$ is Lipschitz continuous on the unit interval. 
If $H_n \uparrow \infty$ satisfies 
  \begin{align}
    (H_n)^2 \Delta_n^{1/2} = o(1),\notag
    \label{}
  \end{align}
  then
  \svnx, defined in \eqref{eq:contvolestimator}, converges in \Ltwo to \sv in probability.
\end{prop}
\begin{proof} We begin by noting that 
\begin{align}
  \svnx - \sigma^2(t) & = \sum_{(h,k) \in \Theta_n} (\cnhk\ - \chk)\;g_{h,k}(t)\notag\\
  & \quad -\sum_{(h,k) \not\in \Theta_n} \chk\;g_{h,k}(t),\label{eq:ronep} 
\end{align}
  where 
\begin{align}  
  \cnhk &= \sum_{i =0}^{n-1} \btghki (X_{t_{i+1}} - X_{t_i})^2 \text{ and }\notag\\  
  \chk &= \int^1_0 \btghks \sigma^2(s) \D s. \notag
\end{align}
We tackle the summands in \eqref{eq:ronep} in turn starting with the first one. But first let 
\begin{align}
  M_i := \int_{t_i}^{t_{i+1}} b(s) \D s, \quad \text{and} \quad  S_i := \int_{t_i}^{t_{i+1}} \sigma(s) \D W_s, \notag
\end{align}
and note that since $X_{t_{i+1}} - X_{t_i} = M_i + S_i$, it follows that
\begin{align}
  (X_{t_{i+1}} - X_{t_i})^2 &= M_i^2  
  + 2M_iS_i +   S_i^2.\notag 
\end{align}
So, \eqref{eq:ronep} may be written as 
\begin{align}
  &\svnx - \sv(t) = B_{1,n}(t) + B_{2,n}(t) + B_{3,n}(t) + B_{4,n}(t), \notag
\end{align}
where
\begin{align}
  &B_{1,n}(t) :=  \sumt g_{h,k}(t) \left(\sum_{i=0}^{n-1} \btghki S_i^2  - \chk\right), \notag\\
  &B_{2,n}(t) := 2 \sumt g_{h,k}(t)\left(\sum_{i=0}^{n-1} \btghki S_i M_i \right), \notag\\
  &B_{3,n}(t) := \sumt g_{h,k}(t)\left( \sum_{i=0}^{n-1} \btghki M_i^2 \right), \notag\\
  &B_{4,n}(t) := - \sum_{\substack{(h,k) \not\in \Theta_n }}g_{h,k}(t)\chk. 
  \label{}
\end{align}
We start by recalling the well-known fact that  frame expansions converge unconditionally in \Ltwo, that is, the expansion converges regardless of the order of summation \citep[Theorem 5.1.7]{Christensen2008}, so that  
\begin{align}
  \Vert B_{4, n}  \Vert_{L^2[0.1]} = o_{a.s.}(1).\notag
  \label{}
\end{align}
We now obtain an estimate for  $B_{3,n}(t)$. Suppose without loss of generality that $b_0 = \sigma =  0$ and let $\{T_m\}$ be a localizing sequence for $b$ and $\sigma$.  Then, by Jensen's inequality
\begin{align}
 E\left(\int_{t_i}^{t_{i+1}} b_{s \wedge T_m}  \D s\right)^2  &\le \Delta_n E\left(\int_{t_i}^{t_{i+1}} b_{s \wedge T_m}^2  \D s\right) \notag \\
&\le \Delta_n \int_{t_i}^{t_{i+1}} E(b_{s \wedge T_m}^2)  \D s \notag\\
&\le \Delta_n \int_{t_i}^{t_{i+1}} E(\sup_{u \le T_m}b_{u}^4)^{1/2}  \D s \notag\\
& \le c \Delta_n^2,
  \label{eq:m}
\end{align}
where the change in the order of integration is justified by Fubini's theorem.  Set $M_i^m = \int_{t_i}^{t_{i+1}} b_{s \wedge T_m}^2  \D s$ and 
\begin{align}
  B^m_{3,n}(t) = \sumt g_{h,k}(t)\left( \sum_{i=0}^{n-1} \btghki (M_i^m)^2 \right)\notag
  \label{}
\end{align}
and note that given $\eta > 0$,
\begin{align}
 \p(\sup_{t \in \domain} \vert B_{3,n}(t)\vert >  \eta ) \le \p(T_m \le 1) + \p( \sup_{t \in \domain} \vert B_{3,n}^m(t)\vert > \eta). \notag
  \label{}
\end{align}
Since $T_m \uparrow \infty$ a.s., the first term on the right becomes arbitrarily small as $m$ tends to infinity. Now since  $g_{h,k}$ and \tghk  are bounded independently of $h$ and $k$, and $n\Delta_n = 1$, it follows by Markov's inequality and \eqref{eq:m} that 
\begin{align}
  \p( \sup_{t \in \domain} \vert B_{3,n}^m(t)\vert > \eta) \le  c H^n \Delta_n.  \notag
  \label{}
\end{align}
Hence,
\begin{align}
  \sup_{t \in \domain} \vert B_{3,n}(t)\vert = o_{P}( 1).
  \label{}
\end{align}
Now set $S_i^m := \int_{t_i}^{t_{i + 1}} \sigma_{s \wedge T_m} d W_s$. Note the following  
\begin{align}
  E((S_i^m)^2)   & = E\left(\int_{t_i}^{t_{i+1}} \sigma^2_{s \wedge T_m} \D s \right)\notag \\
                & = \int_{t_i}^{t_{i+1}} E(\sigma^2_{s \wedge T_m}) \D s \notag \\
                & = \int_{t_i}^{t_{i+1}} (E(\sup_{u \wedge T_m} \sigma^4_{u})^{1/2}) \D s \notag \\
  & \le c \Delta_n.\label{eq:news}
\end{align}
By \holder's inequality, \eqref{eq:m}, and \eqref{eq:news},  we have
\begin{align}
  E(M_i^m S_i^m) & = (E(M_i^m)^2 E( S^m_i)^2)^{1/2} \notag\\
  & \le c\Delta_n^{3/2}.\label{eq:ms}
\end{align}
Set 
\begin{align}
 B_{2,n}^m(t) := 2 \sumt g_{h,k}(t)\left(\sum_{i=0}^{n-1} \btghki S_i^m M_i^m \right) \notag.
  \label{}
\end{align}
Then for each $m$,  because  $g_{h,k}$ and \tghk  are bounded independently of $h$ and $k$, and $n\Delta_n = 1$, we conclude by an appeal to Markov's inequality  that  $\p( \sup_{t \in \domain} \vert B_{2,n}^m(t)\vert > \eta) \le  c H^n \Delta_n^{1/2}$. By the previously used localization argument, 
\begin{align}
  \sup_{t\in\domain}\vert B_{2,n}(t)\vert = o_{P}(1).
  \label{}
\end{align}
Now we tackle the final piece $B_{1,n}(t)$. Let 
\begin{align}
  A^n  := \sum_{i=0}^{n-1}\btghki S_i^2 -\int_{0}^{1} \sigma^2(s) \btghks \D s.
\end{align}
We will first obtain an upper bound for $A^n$; we proceed by adding and subtracting $\sumin\int_{t_i}^{t_{i+1}}\btghki \sigma^2(s)  \D s$  from  $A$ to yield: 
\begin{align}
  A^n & = \sumin\btghki \left(S_i^2 -\int_{t_i}^{t_{i+1}} \sigma^2(s) \D s\right)\notag \\
  &\quad +  \sumin\left(\int_{t_i}^{t_{i+1}} \sigma^2(s) \{\btghki - \btghks\} \D s\right) \notag\\
  &=:A_{1}^n + A_{2}^n \notag.
  \label{}
\end{align}
We obtain estimates in turn for the summands. By Assumption \ref{as:vol}, $\sigma$ is \cadlag so that it is almost surely  bounded on \domain;  by the continuity of \tghk   and Lemma \eqref{lem:modtg}, we have 
\begin{align}
  A_{2}^n &= \sumin \int_{t_i}^{t_{i+1}} \sigma^2(s)  \{\btghki - \btghks\} \D s\notag\\
  &\le c\modc{\tghk}{\Delta_n},\quad a.s.,\notag
  \label{}
\end{align}
where $\modc{\tghk}{\Delta_n}$ is the modulus of continuity of $\tghk$ on an interval of length $\Delta_n$.  By the Lipschitz continuity of $g$ we have, 
\begin{align}
  A_{2}^n = O_{a.s.}(\modc{g}{\Delta_n}) = O_{a.s.}( \Delta_n). \notag 
  \label{}
\end{align}
Now, we obtain an estimate for $A_{1}^n$.
First, let $D_i^n: \Omega\times[0,1] \to \real$ for $i = 0, \cdots, n-1$ be defined as follows:  
\begin{align}
&D_{i}^n(t) := \btghki\left(\int_{t_i}^{t} \sigma_{u \wedge T_m}\D W_u\right)\mathbbm{1}_{(t_i, t_{i+1}]}(t).
  \label{eq:Di}\\
&D_0^n(0) := 0.
\end{align}
So, $D_i^n(t)$ is 0 on $[0,1]$ except when $t$ is in $(t_i, t_{i+1}]$. Moreover,  
\begin{align}
  D_i^n (t) D_j^n(t) = 0, \qquad i \not = j.\notag
  \label{}
\end{align}
Now, for $t \in (t_i, t_{i+1}]$,
\begin{align}
E(D_i^n(t)^4) &= \btghki^4\mathbbm{1}_{(t_i, t_{i+1}]}(t)\e\left( \left(\int_{t_i}^{t} \sigma_{u\wedge T_m}\D W_u\right)^4\right)& \notag\\
& \le c\mathbbm{1}_{(t_i, t_{i+1}]}(t)\e\left( \left(\int_{t_i}^{t} \sigma_{u\wedge T_m}^2\D u\right)^2\right) &\text{B.D.G}\notag\\
& \le c(t - t_i)\mathbbm{1}_{(t_i, t_{i+1}]}(t) \e \left(\int_{t_i}^{t} \sigma_{u\wedge T_m}^4\D u\right)&\text{Jensen}\notag\\
  & \le c\Delta_n\mathbbm{1}_{(t_i, t_{i+1}]}(t)\int_{t_i}^{t_{i+1}} \e\left(\sigma_{u\wedge T_m}^4\right) \D u &\text{Fubini}\notag \\
  %& \le c \Delta_n \mathbbm{1}_{(t_i, t_{i+1}]}(t)\int_{t_i}^{t_{i+1}} \e\left(\sup_{s \le T_m}\sigma_s^4\right) \D u& \notag \\
  & \le c \mathbbm{1}_{(t_i, t_{i+1}]}(t)\Delta_n^2 &
  \label{eq:dz}
\end{align}
where the application of Fubini's theorem \citep[Theorem VII.36.B]{Halmos1950} is justified by the fact that $\sigma^4$ is non negative and measurable with respect to the product $\sigma$-algebra on $\domain\times\Omega$.
Now, using \ito's integration by parts formula, we may write
\begin{align}
  \e((A_1^n)^2)  &= \e\left(2\,\sumin\int_{t_i}^{t_{i+1} }\btghki\left(\int_{t_i}^{s} \sigma_{u \wedge T_m}\D W_u\right)\sigma_{s\wedge T_m} \D W_s\right)^2\notag \\
  & = 4 \e\left(\int_0^{1 }\sumin D_i^n(s) \sigma_{s\wedge T_m} \D W_s\right)^2 \notag \\
  %& = 4 \e\left(\int_0^{1 }\sumin D_i^n(s)^2 \sigma_{s\wedge T_m}^2 \D s\right) \notag \\
  & \le c \int_0^{1 }\sumin\e( D_i^n(s)^2 \sigma_{s\wedge T_m}^2) \D s  \notag \\
  & \le c \int_0^{1 }\sumin\e( D_i^n(s)^4)^{1/2} \e( \sigma_{s\wedge T_m}^4)^{1/2} \D s  \notag \\
  & \le  c \int_0^{1 }\sumin \mathbbm{1}_{(t_i, t_{i+1}]}(s)\Delta_n  \D s  \notag \\
  %& \le  c \Delta_n\int_0^{1 }\sumin \mathbbm{1}_{(t_i, t_{i+1}]}(s)  \D s  \notag \\
  & \le  c \Delta_n.  \notag 
  \label{}
\end{align}
\begin{comment}
We now note that since $4 \int_0^{1 }\left(\sumin D_i^n(s) \sigma_s\right)^2 \D s$ is increasing finite and continuous, it is locally bounded and therefore locally integrable. It is thus the case that the process defined by, $A_1^n(t) := 2 \int_0^{t}\sumin D_i^n(s) \sigma_s \D W_s$, $t \in \domain$, is a continuous local martingale. As such  $A_1^n(\cdot)$ is locally square integrable with predictable quadratic variation given by    $\langle A_1^n(\cdot) \rangle := 4 \int_0^{\cdot }\left(\sumin D_i^n(s) \sigma_s\right)^2 \D s$. Now let $\{T_n\}$ be a localizing sequence for $A_1^n(\cdot)$, then for any bounded stopping time $T$, we have
$  \e((A_1^n(T\wedge T_n) )^2) = \e(\langle A_1^n (\cdot) \rangle_{T \wedge T_n})$. By Fatou's Lemma $  \e( (A_1^n(T) )^2)  \le \e(\langle A_1^n(\cdot) \rangle_T)$. In other words $(A_1^n(\cdot))^2$ is L-dominated\footnote{The Lenglart domination property is defined in \citet[I.3.29]{Jacod2003}. } by $\langle A_1^n(\cdot) \rangle$.   

Now given $\varepsilon >  0$, we have by Lenglart's inequality for predictable majorants that 
  \begin{align}
    P(A_1^n \ge \varepsilon)  & \le P(\sup_s (A_1^n(s))^2 \ge \varepsilon^2)\notag\\
    & \le \varepsilon^{-2} \e(\langle A_1^n(\cdot) \rangle_1 \wedge \gamma) + P(\langle A_1^n(\cdot) \rangle_1 \le \gamma),
  \end{align}
where $\gamma > 0$ is arbitrary. Because $\langle A_1^n(\cdot) \rangle_1$ is almost surely finite, letting $\gamma \downarrow 0$ simultaneousy as $n \uparrow \infty$, it is easily seen that $A^n_1$ tends to zero in probability.
\end{comment}
\begin{comment}
\begin{align}
  T_c = \inf \{ t >0 : \sigma^2(t) > c \} \wedge 1\notag.
  \label{}
\end{align}
By the regularity assumption on $\sigma$, $\p(T_c < 1)$ can be made arbitrarily small by taking $c$ sufficiently large.  

Now using the fact that $\int_0^{t \wedge T_c}\sumin D_i(s) \sigma(s) \D W_s$ is a  martingale, we may make an appeal to the Burkholder-Davis-Gundy (BDG) inequality \citep[Theorem 10.36][]{He1992} to yield:
\begin{align}
  \e(\vert A_1^{T_c} \vert) & \le c\e\left[\left\vert\int_0^{1\wedge T_c}\left(\sumin D_i(s) \sigma(s)\right)^2 \D s\right\vert^{1/2} \right]\notag \\
  & \le c\e\left[\left\vert\int_0^{1 \wedge T_c}\sumin \{D_i(s) \sigma(s)\}^2 \D s\right\vert^{1/2} \right]\notag,
  \notag
  \label{}
\end{align}
where the last line follows because $D_i (s) D_j(s) = 0$ whenever $i \not= j$. Now if  we define $D^*_i := \sup_{t_i <s \le t_{i+1} \wedge T_c} D_i(s)$, and use the fact that $\sigma^2$ is less than  $c$ before $T_c$ then
\begin{align}
  \e(\vert A_1^{T_c}\vert ) & \le c\e\left[\left\vert\sumin \Delta_n(D_i^*)^2 \right\vert^{1/2} \right] \notag\\
  & \le c\Delta_n^{1/2}\e\left[\left\vert\sumin (D_i^*)^2 \right\vert^{1/2} \right] \notag\\
  & \le c\Delta_n^{1/2}\sumin\e[D_i^*]   
  \label{eq:ra1}
\end{align}
where $c$ is a generic constant representing the bound on $\sigma^2$ and the BDG constant.  Note  from the definition of  $D_i$  that it  is itself a martingale, so we may bound  $D^*_i$  with yet another application of the BDG inequality. That is
\begin{align}
  \e((D^*_i)^2) &\le \e\left(\int^{t_{i+1} \wedge T_c}_{t_i} \sigma^2(s) \D s\right) \notag\\
  &\le c \Delta_n. 
  \label{}
\end{align}
Hence, given any $\varepsilon > 0$,  $\p( \sup_{t \in [0,1]} \vert B_{1,n}(t) \vert > \varepsilon)  = O(H_n n^{-1}) + \p(T_c < 1)$.
\end{comment}
By Chebyshev's inequality and the previously used stopping time argument, we have
$A^n = O_P(\Delta_n)$. By the boundedness of \ghk, we have
\begin{align}
  \sup_{t \in \domain} \vert B_{1,n}(t)\vert = o_P(1).\notag
  \label{}
\end{align}
Hence, $B_{j,n}(t)$ for $j =1,\cdots,4$, tends to zero in \Ltwo in probability.
\end{proof}
